% \iffalse meta-comment
%
% Copyright (C) 1997-2003 by Michael J. Downes
% Copyright (C) 2007-2008 by Morten Hoegholm
% Copyright (C) 2007-2014 by Lars Madsen
% Copyright (C) 2007-2020 by Will Robertson
% Copyright (C) 2010-2017 by Joseph Wright
%
% This work may be distributed and/or modified under the
% conditions of the LaTeX Project Public License, either
% version 1.3 of this license or (at your option) any later
% version. The latest version of this license is in
%    http://www.latex-project.org/lppl.txt
% and version 1.3 or later is part of all distributions of
% LaTeX version 2005/12/01 or later.
%
% This work has the LPPL maintenance status "maintained".
%
% The Current Maintainer of this work is Will Robertson.
%
% This work consists of the main source file breqn.dtx
% and the derived files
%    breqn.sty, breqn.pdf, breqn.ins.
%
% Distribution:
%    CTAN:macros/latex/contrib/mh/breqn.dtx
%    CTAN:macros/latex/contrib/mh/breqn.pdf
%
% Unpacking:
%           tex breqnbundle.ins
%
% Documentation:
%    The class ltxdoc loads the configuration file ltxdoc.cfg
%    if available. Here you can specify further options, e.g.
%    use A4 as paper format:
%       \PassOptionsToClass{a4paper}{article}
%
%    Programm calls to get the documentation (example):
%       pdflatex breqn.dtx
%       makeindex -s gind.ist breqn.idx
%       pdflatex breqn.dtx
%       makeindex -s gind.ist breqn.idx
%       pdflatex breqn.dtx
%
% Installation:
%    TDS:tex/latex/breqn/breqn.sty
%    TDS:doc/latex/breqn/breqn.pdf
%    TDS:source/latex/breqn/breqn.dtx
%
%<*driver>
\NeedsTeXFormat{LaTeX2e}
\documentclass{ltxdoc}
\CodelineIndex
\EnableCrossrefs
\setcounter{IndexColumns}{2}
\usepackage{color,verbatim,xspace,varioref,listings,amsmath,trace}

\usepackage{geometry}

\def\partname{Part}

\definecolor{hilite}{rgb}{0.2,0.4,0.7}
\def\theCodelineNo{\textcolor{hilite}{\sffamily\tiny\arabic{CodelineNo}}}

\lstloadlanguages{[AlLaTeX]TeX}

\lstnewenvironment{literalcode}
  {\lstset{gobble=2,columns=fullflexible,basicstyle=\color{hilite}\ttfamily}}
  {}
\makeatletter

{\catcode`\%=12
 \long\gdef\@gobble@percent@space#1{\ifx
   #1%\expandafter\@gobble\else\expandafter#1\fi}}

\AtBeginDocument{\def\verbatim@processline{\expandafter\check@percent
  \the\verbatim@line\par}}
\newwrite\tmp@out
\newcounter{xio}
\newenvironment{xio}{% example input and output
  \par\addvspace\bigskipamount
  \hbox{\itshape
    \refstepcounter{xio}\kern-\p@ Example \thexio}\@nobreaktrue
  \immediate\openout\tmp@out\jobname.tmp \relax
  \begingroup
  \let\do\@makeother\dospecials\catcode`\^^M\active
  \def\verbatim@processline{
    \immediate\write\tmp@out{\expandafter\@gobble@percent@space
      \the\verbatim@line}}%
  \verbatim@start
}{%
  \immediate\closeout\tmp@out
  \@verbatim\frenchspacing\@vobeyspaces
  \@@input \jobname.tmp \relax
  \endgroup
  \par\medskip
  \noindent\ignorespaces
  \@@input \jobname.tmp \relax
  \par\medskip
}

\providecommand*\pkg[1]{\textsf{#1}}
\providecommand*\cls[1]{\textsf{#1}}
\providecommand*\opt[1]{\texttt{#1}}
\providecommand*\env[1]{\texttt{#1}}
\providecommand*\fn[1]{\texttt{#1}}

\providecommand*\cn[1]{\cs{#1}}
\providecommand*\csarg[1]{\texttt{\char`\{#1\char`\}}}

\providecommand*\tex{\TeX\xspace}
\providecommand*\latex{\LaTeX\xspace}
\providecommand*\dbldollars{\texttt{\detokenize{$$}}}%$$
\providecommand*\arg{}
\edef\arg{\expandafter\@gobble\string\#}

\newenvironment{aside}{\begin{quote}\bfseries}{\end{quote}}
\newenvironment{dn}{\begin{quote}\bfseries}{\end{quote}}

\newcommand\dash{\textemdash}
\newcommand\dbslash[1]{\texttt{\string\\}}
\newcommand\thepkg{the \pkg{breqn} package\xspace}

\providecommand*\texbook{\textsl{The \protect\TeX{}book}\xspace}

\providecommand*\dotsc{\ldots}
\providecommand*\eqref[1]{(\ref{#1})}

\providecommand*\qq[1]{\textquotedblleft#1\textquotedblright}
\providecommand*\quoted[1]{\textquoteleft#1\textquoteright}
\providecommand*\dquoted[1]{\textquotedblleft#1\textquotedblright}

\providecommand*\ie{i.e.,\xspace}
\providecommand*\eg{e.g.,\xspace}
\providecommand*\etc{etc.\xspace}
\providecommand*\cf{cf.\xspace}

\providecommand*\ndash{\unskip\textendash\ignorespaces}
\providecommand*\mdash{\unskip\textemdash\ignorespaces}

\makeatother

\usepackage{breqn}

\begin{document}
  \DocInput{breqn.dtx}
\end{document}
%</driver>
% \fi
%
%
% \title{The \pkg{breqn} package}
% \def\fileversion{0.98g}
% \def\filedate{2019/10/15}
% \date{\pkg{breqn} bundle: \filedate\space\fileversion}
% \author{Authors: Michael J. Downes, Morten H\o gholm\\ Maintained by Morten H\o gholm, Will Robertson\\ Feedback: \texttt{https://github.com/wspr/breqn/issues}}
%
% \maketitle
% \begin{abstract}
%   The \pkg{breqn} package facilitates automatic line-breaking of
%   displayed math expressions.
% \end{abstract}
%
%
% \tableofcontents
%
%
% \part{User's guide}
%
% \section{A bit of history}
%
% Originally \pkg{breqn}, \pkg{flexisym}, and \pkg{mathstyle} were
% created by Michael J.~Downes from the American Mathematical Society
% during the 1990's up to late 2002. Sadly---and much to the shock of
% the \TeX\ world---Michael passed away in early 2003 at the age of
% only~44.
%
% The American Mathematical Society kindly allowed Morten H\o gholm to
% assume maintainership of this part of his work and we wish to
% express our gratitude to them and to Barbara Beeton in particular for
% providing the files needed.
%
% MH brought Michael's work to a wider audience, thereby allowing users to create
% more \emph{masterpieces of the publishing art} as we think he would
% have wanted.
%
% Following the July 2008 breqn release, \pkg{breqn} was left in the hands
% of a maintenance team, while MH moved on with other projects.
%
% \section{Package loading}
%
%
% The recommended way of loading the \pkg{breqn} package is to load it
% \emph{after} other packages dealing with math, \ie, after
% \pkg{amsmath}, \pkg{amssymb}, or packages such as \pkg{mathpazo} or
% \pkg{mathptmx}.
%
% The \pkg{flexisym} package (described in section \vref{flexisym}) is
% required by \pkg{breqn} and ensures the math symbols are set up
% correctly. By default \pkg{breqn} loads it with support for Computer
% Modern but if you use a different math package requiring slightly
% different definitions, it must be loaded before \pkg{breqn}. Below
% is an example of how you enable \pkg{breqn} to work with the widely
% used \pkg{mathpazo} package.
% \begin{verbatim}
%\usepackage{mathpazo}
%\usepackage[mathpazo]{flexisym}
%\usepackage{breqn}
% \end{verbatim}
% Currently, the packages \pkg{mathpazo} and \pkg{mathptmx} are
% supported. Despair not: Chances are that the package will work using
% the default settings. If you find that a particular math font
% package doesn't work then please see implementation in
% \fn{flexisym.dtx} for how to create a support file---it is easier
% than one might think. Contributions welcome.
%
% The documentation for the package was formerly found in
% \fn{breqndoc}. It has now been added to this implementation
% file. Below follows the contents of the original \pkg{breqn}
% documentation. Not all details hold anymore but I have prioritized
% fixing the package.
%
% \section{Introduction}
%
% The \pkg{breqn} package for \LaTeX\ provides solutions to a number of
% common difficulties in writing displayed equations and getting
% high-quality output. For example, it is a well-known inconvenience that
% if an equation must be broken into more than one line, \cs{left} \dots\
% \cs{right} constructs cannot span lines. The \pkg{breqn} package makes
% them work as one would expect whether or not there is an intervening
% line break.
%
% The single most ambitious goal of the \pkg{breqn} package, however, is
% to support automatic linebreaking of displayed equations. Such
% linebreaking cannot be done without substantial changes under the hood
% in the way math formulas are processed. For this reason, especially in
% the alpha release, users should proceed with care and keep an eye out
% for unexpected glitches or side effects.
%
% \section{Principal features}
% The principal features of the \pkg{breqn} package are:
% \begin{description}
%
% \item[semantically oriented structure] The way in which compound
% displayed formulas are subdivided matches the logical structure more
% closely than, say, the standard \env{eqnarray} environment. Separate
% equations in a group of equations are written as separate environments
% instead of being bounded merely by \dbslash/ commands. Among other
% things, this clears up a common problem of wrong math symbol spacing at
% the beginning of continuation lines. It also makes it possible to
% specify different vertical space values for the space between lines of a
% long, broken equation and the space between separate equations in a
% group of equations.
%
% \item[automatic line breaking] Overlong equations will be broken
% automatically to the prevailing column width, and continuation lines
% will be indented following standard conventions.
%
% \item[line breaks within delimiters] Line breaks within \cs{left} \dots\
% \cs{right} delimiters work in a natural way. Line breaks can be
% forbidden below a given depth of delimiter nesting through a package
% option.
%
% \item[mixed math and text] Display equations that contain mixed
% math and text, or even text only, are handled naturally by means of a
% \env{dseries} environment that starts out in text mode instead of math
% mode.
%
% \item[ending punctuation] The punctuation at the end of a displayed
% equation can be handled in a natural way that makes it easier to promote
% or demote formulas from\slash to inline math, and to apply special
% effects such as adding space before the punctuation.
%
% \item[flexible numbering] Equation numbering is handled in a natural
% way, with all the flexibility of the \pkg{amsmath} package and with no
% need for a special \cs{nonumber} command.
%
% \item[special effects] It is easy to apply special effects to individual
% displays, e.g., changing the type size or adding a frame.
%
% \item[using available space] Horizontal shrink is made use of
% whenever feasible. With most other equation macros it is frozen when it
% occurs between \cs{left} \dots\ \cs{right} delimiters, or in any sort of
% multiline structure, so that some expressions require two lines that would
% otherwise fit on one.
%
% \item[high-quality spacing] The \cs{abovedisplayshortskip} is used when
% applicable (other equation macros fail to apply it in equations of more
% than one line).
%
% \item[abbreviations] Unlike the \pkg{amsmath} equation environments, the
% \pkg{breqn} environments can be called through user-defined abbreviations
% such as \cs{beq} \dots\ \cs{eeq}.
%
% \end{description}
%
% \section{Shortcomings of the package}
% The principal known deficiencies of the \pkg{breqn} package are:
%
% \subsection{Incompatibilities} As it pushes the envelope
% of what is possible within the context of \LaTeXe, \thepkg will tend
% to break other packages when used in combination with them, or to fail
% itself, when there are any areas of internal overlap; successful use may
% in some cases depend on package loading order.
%
%
%
% \subsection{Indention of delimited fragments} When line breaks within
% delimiters are involved, the automatic indention of continuation lines
% is likely to be unsatisfactory and need manual adjustment. I don't see
% any easy way to provide a general solution for this, though I have some
% ideas on how to attain partial improvements.
%
% \subsection{Math symbol subversion}
% In order for automatic line breaking to work, the operation of all the
% math symbols of class 2, 3, 4, and 5 must be altered (relations, binary
% operators, opening delimiters, closing delimiters). This is done by an
% auxiliary package \pkg{flexisym}. As long as you stick to the advertised
% \LaTeX\ interface for defining math symbols (\cs{DeclareMathSymbol}),
% things should work OK most of the time. Any more complex math symbol
% setup is quite likely to quarrel with the \pkg{flexisym} package.
% See Section~\vref{flexisym} for further information.
%
% \subsection{Subscripts and superscripts}
%
% Because of the changes to math symbols of class 2--5, writing certain
% combinations such as \verb'^+' or \verb'_\pm' or \verb'^\geq' without
% braces would lead to error messages; (The problem described here
% already exists in standard \LaTeX\ to a lesser extent, as you may know
% if you ever tried \verb'^\neq' or \verb'^\cong'; and indeed there are
% no examples in the \LaTeX\ book to indicate any sanction for omitting
% braces around a subscript or superscript.)
%
% The \pkg{flexisym} package therefore calls, as of version 0.92, another
% package called \pkg{mathstyle} which turns \verb'^' and \verb'_' into
% active characters. This is something that I believe is desirable in any
% case, in the long run, because having a proper mathstyle variable
% eliminates some enormous burdens that affect almost any
% nontrivial math macros, as well as many other things where the
% connection is not immediately obvious, e.g., the \LaTeX\ facilities for
% loading fonts on demand.
%
% Not that this doesn't introduce new and interesting problems of its
% own---for example, you don't want to put usepackage statements
% after flexisym for any package that refers to, e.g., \verb'^^J' or
% \verb'^^M'
% internally (too bad that the \LaTeX\ package loading code does not
% include automatic defenses to ensure normal catcodes in the interior of
% a package; but it only handles the \verb'@' character).
%
% But I took a random AMS journal article, with normal end-user kind of
% \LaTeX\ writing, did some straightforward substitutions to change all
% the equations into dmath environments, and ran it with active math
% sub/sup: everything worked OK. This suggests to me that it can work in
% the real world, without an impossible amount of compatibility work.
%
% \section{Incomplete}
% In addition, in the \textbf{alpha release [1997/10/30]} the following
% gaps remain to be filled in:
% \begin{description}
% \item[documentation]
% The documentation could use amplification, especially more
% illustrations, and I have undoubtedly overlooked more than a few errors.
%
% \item[group alignment] The algorithm for doing alignment
% of mathrel symbols across equations in a \env{dgroup} environment
% needs work. Currently the standard and \opt{noalign} alternatives
% produce the same output.
%
% \item[single group number] When a \env{dgroup} has a group number and
% the individual equations are unnumbered, the handling and placement of
% the group number aren't right.
%
% \item[group frame] Framing a group doesn't work, you might be able to
% get frames on the individual equations at best.
%
% \item[group brace] The \opt{brace} option for \env{dgroup} is intended
% to produce a large brace encompassing the whole group. This hasn't been
% implemented yet.
%
% \item[darray environment] The \env{darray} environment is unfinished.
%
% \item[dseries environment] The syntax and usage for the \env{dseries}
% environment are in doubt and may change.
%
% \item[failure arrangements] When none of the line-breaking passes for a
% \env{dmath} environment succeeds\dash i.e., at least one line is
% overfull\dash the final arrangement is usually rather poor. A better
% fall-back arrangement in the failure case is needed.
%
% \end{description}
%
% \section{Package options}
%
% Many of the package options for \thepkg are the same as options of the
% \env{dmath} or \env{dgroup} environments, and some of them require an
% argument, which is something that cannot be done through the normal
% package option mechanism. Therefore most of the \pkg{breqn} package
% options are designed to be set with the \cs{breqnsetup} command after the
% package is loaded. For example, to load the package and set the
% maximum delimiter nesting depth for line breaks to~1:
% \begin{verbatim}
% \usepackage{breqn}
% \breqnsetup{breakdepth={1}}
% \end{verbatim}
%
% See the discussion of environment options, Section~\vref{envopts}, for
% more information.
%
% Debugging information is no longer available as a package
% option. Instead, the tracing information has been added in a fashion
% so that it can be enabled as a docstrip option:
% \begin{verbatim}
% \generate{\file{breqn.sty}{\from{breqn.dtx}{package,trace}}}
% \end{verbatim}
%
%
% \section{Environments and commands}
% \subsection{Environments}
% All of the following environments take an optional argument for
% applying local effects such as changing the typesize or adding a
% frame to an individual equation.
%
% \begin{description}
% \item[\env{dmath}] Like \env{equation} but supports line breaking and variant
% numbers.
%
% \item[\env{dmath*}] Unnumbered; like \env{displaymath} but supports line
% breaking
%
% \item[\env{dseries}] Like \env{equation} but starts out in text mode;
% intended for series of mathematical expressions of the form `A, B, and
% C'. As a special feature, if you use
% \begin{verbatim}
% \begin{math} ... \end{math}
% \end{verbatim}
% for each expression in the series, a suitable amount of inter-expression
% space will be automatically added. This is a small step in the direction of
% facilitating conversion of display math to inline math, and vice versa: If
% you write a display as
% \begin{verbatim}
% \begin{dseries}
% \begin{math}A\end{math},
% \begin{math}B\end{math},
% and
% \begin{math}C\end{math}.
% \end{dseries}
% \end{verbatim}
% then conversion to inline form is simply a matter of removing the
% \verb'\begin{dseries}' and \verb'\end{dseries}' lines; the contents of the
% display need no alterations.
%
% It would be nice to provide the same feature for \verb'$' notation but
% there is no easy way to do that because the \verb'$' function has no
% entry point to allow changing what happens before math mode is entered.
% Making it work would therefore require turning \verb'$' into an active
% character, something that I hesitate to do in a \LaTeXe\ context.
%
% \item[\env{dseries*}] Unnumbered variant of \env{dseries}
%
% \item[\env{dgroup}] Like the \env{align} environment of \pkg{amsmath},
% but with each constituent equation wrapped in a \env{dmath},
% \env{dmath*}, \env{dseries}, or \env{dseries*} environment instead of being
% separated by \dbslash/. The equations are numbered with a group number.
% When the constituent environments are the numbered forms (\env{dmath} or
% \env{dseries}) they automatically switch to `subequations'-style
% numbering, i.e., something like (3a), (3b), (3c), \dots, depending on
% the current form of non-grouped equation numbers. See also
% \env{dgroup*}.
%
% \item[\env{dgroup*}] Unnumbered variant of \env{dgroup}. If the
% constituent environments are the numbered forms, they get normal
% individual equation numbers, i.e., something like (3), (4), (5), \dots~.
%
% \item[\env{darray}] Similar to \env{eqnarray} but with an argument like
% \env{array} for giving column specs. Automatic line breaking is not
% done here.
%
% \item[\env{darray*}] Unnumbered variant of \env{darray}, rather like
% \env{array} except in using \cs{displaystyle} for all column
% entries.
%
% \item[\env{dsuspend}] Suspend the current display in order to print
%   some text, without loss of the alignment. There is also a command
%   form of the same thing, \cs{intertext}.
% \end{description}
%
% \subsection{Commands}
%
% The commands provided by \thepkg are:
% \begin{description}
% \item[\cs{condition}] This command is used for
% a part of a display which functions as a condition on the main
% assertion. For example:
% \begin{verbatim}
% \begin{dmath}
% f(x)=\frac{1}{x} \condition{for $x\neq 0$}
% \end{dmath}.
% \end{verbatim}
% \begin{dmath}
% f(x)=\frac{1}{x} \condition{for $x\neq 0$}
% \end{dmath}.
% The \cs{condition} command automatically switches to text mode (so that
% interword spaces function the way they should), puts in a comma, and
% adds an appropriate amount of space. To facilitate promotion\slash
% demotion of formulas, \cs{condition} \qq{does the right thing} if used
% outside of display math.
%
% To substitute a different punctuation mark instead of the default comma,
% supply it as an optional argument for the \cs{condition} command:
% \begin{verbatim}
% \condition[;]{...}
% \end{verbatim}
% (Thus, to get no punctuation: \verb'\condition[]{...}'.)
%
% For conditions that contain no text, you can use the starred form of the
% command, which means to stay in math mode:
% \begin{verbatim}
% \begin{dmath}
% f(x)=\frac{1}{x} \condition*{x\neq 0}
% \end{dmath}.
%
% If your material contains a lot of conditions like these, you might like
% to define shorter abbreviations, e.g.,
% \begin{verbatim}
% \newcommand{\mc}{\condition*}% math condition
% \newcommand{\tc}{\condition}%  text condition
% \end{verbatim}
% But \thepkg refrains from predefining such abbreviations in order that
% they may be left to the individual author's taste.
%
% \item[\cs{hiderel}] In a compound equation it is sometimes desired to
% use a later relation symbol as the alignment point, rather than the
% first one. To do this, mark all the relation symbols up to the desired
% one with \cs{hiderel}:
% \begin{verbatim}
% T(n) \hiderel{\leq} T(2^n) \leq c(3^n - 2^n) ...
% \end{verbatim}
% \end{description}
%
% \section{Various environment options}\label{envopts}
%
% The following options are recognized for the \env{dmath}, \env{dgroup},
% \env{darray}, and \env{dseries} environments; some of the options do not
% make sense for all of the environments, but if an option is used where
% not applicable it is silently ignored rather than treated as an error.
% \changes{v0.98b}{2010/08/27}{replaced relindent with indentstep}
%
% \begin{verbatim}
% \begin{dmath}[style={\small}]
% \begin{dmath}[number={BV}]
% \begin{dmath}[labelprefix={eq:}]
% \begin{dmath}[label={xyz}]
% \begin{dmath}[indentstep={2em}]
% \begin{dmath}[compact]
% \begin{dmath}[spread={1pt}]
% \begin{dmath}[frame]
% \begin{dmath}[frame={1pt},framesep={2pt}]
% \begin{dmath}[background={red}]
% \begin{dmath}[color={purple}]
% \begin{dmath}[breakdepth={0}]
% \end{verbatim}
%
% Use the \opt{style} option to change the type size of an individual
% equation. This option can also serve as a catch-all option for
% altering the equation style in other ways; the contents are simply
% executed directly within the context of the equation.
%
% Use the \opt{number} option if you want the number for a particular
% equation to fall outside of the usual sequence. If this option is used
% the equation counter is not incremented. If for some reason you need to
% increment the counter and change the number at the same time, use the
% \opt{style} option in addition to the \opt{number} option:
% \begin{verbatim}
% style={\refstepcounter{equation}}
% \end{verbatim}
%
% Use of the normal \cs{label} command instead of the \opt{label} option
% works, I think, most of the time (untested). \opt{labelprefix} prepends
% its argument to the label (only useful as a global option, really),
% and must be called before \opt{label}.
%
% \changes{v0.98b}{2010/08/27}{replaced relindent with indentstep}
% Use the \opt{indentstep} option to specify something other than the
% default amount for the indention of relation symbols. The default is
% 8pt.
%
% Use the \opt{compact} option in compound equations to inhibit line
% breaks at relation symbols. By default a line break will be taken before
% each relation symbol except the first one. With the \opt{compact} option
% \LaTeX\ will try to fit as much material as possible on each line, but
% breaks at relation symbols will still be preferred over breaks at binary
% operator symbols.
%
% Use the \opt{spread} option to increase (or decrease) the amount of
% interline space in an equation. See the example given above.
%
% Use the \opt{frame} option to produce a frame around the body of the
% equation. The thickness of the frame can optionally be specified by
% giving it as an argument of the option. The default thickness is
% \cs{fboxrule}.
%
% Use the \opt{framesep} option to change the amount of space separating
% the frame from what it encloses. The default space is \cs{fboxsep}.
%
% Use the \opt{background} option to produce a colored background for the
% equation body. The \pkg{breqn} package doesn't automatically load the
% \pkg{color} package, so this option won't work unless you remember
% to load the \pkg{color} package yourself.
%
% Use the \opt{color} option to specify a different color for the contents
% of the equation. Like the \opt{background} option, this doesn't work if
% you forgot to load the \pkg{color} package.
%
% Use the \opt{breakdepth} option to change the level of delimiter nesting
% to which line breaks are allowed. To prohibit line breaks within
% delimiters, set this to 0:
% \begin{verbatim}
% \begin{dmath}[breakdepth={0}]
% \end{verbatim}
% The default value for breakdepth is 2. Even when breaks are allowed
% inside delimiters, they are marked as less desirable than breaks outside
% delimiters. Most of the time a break will not be taken within delimiters
% until the alternatives have been exhausted.
%
% Options for the \env{dgroup} environment: all of the above, and also
% \begin{verbatim}
% \begin{dgroup}[noalign]
% \begin{dgroup}[brace]
% \end{verbatim}
%
% By default the equations in a \env{dgroup} are mutually aligned on their
% relation symbols ($=$, $<$, $\geq$, and the like). With the
% \opt{noalign} option each equation is placed individually without
% reference to the others.
%
% The \opt{brace} option means to place a large brace encompassing the
% whole group on the same side as the equation number.
%
% Options for the \env{darray} environment: all of the above (where
% sensible), and also
% \begin{verbatim}
% \begin{darray}[cols={lcr@{\hspace{2em}}lcr}]
% \end{verbatim}
% The value of the \opt{cols} option for the darray environment should be
% a series of column specs as for the \env{array} environment, with the
% following differences:
% \begin{itemize}
% \item For l, c, and r what you get is not text, but math, and
% displaystyle math at that. To get text you must use a 'p' column
% specifier, or put an \cs{mbox} in each of the individual cells.
%
% \item Vertical rules don't connect across lines.
% \end{itemize}
%
% \section{The \pkg{flexisym} package}\label{flexisym}
%
% The \pkg{flexisym} package does some radical changes in the setup for
% math symbols to allow their definitions to change dynamically throughout
% a document. The \pkg{breqn} package uses this to make symbols of classes
% 2, 3, 4, 5 run special functions inside an environment such as
% \env{dmath} that provide the necessary support for automatic line
% breaking.
%
% The method used to effect these changes is to change the definitions of
% \cs{DeclareMathSymbol} and \cs{DeclareMathDelimiter}, and then
% re-execute the standard set of \LaTeX\ math symbol definitions.
% Consequently, additional mathrel and mathbin symbols defined by other
% packages will get proper line-breaking behavior if the other package is
% loaded after the \pkg{flexisym} package and the symbols are defined
% through the standard interface.
%
%
%
% \section{Caution! Warning!}
% Things to keep in mind when writing documents with \thepkg:
% \begin{itemize}
%
% \item The notation $:=$ must be written with the command \cs{coloneq}.
%   Otherwise the $:$ and the $=$ will be treated as two separate relation
%   symbols with an \qq{empty RHS} between them, and they will be printed
%   on separate lines.
%
% \item Watch out for constructions like \verb'^+' where a single binary
% operator or binary relation symbol is subscripted or superscripted. When
% the \pkg{breqn} or \pkg{flexisym} package is used, braces are mandatory
% in such constructions: \verb'^{+}'. This applies for both display and
% in-line math.
%
% \item If you want \LaTeX\ to make intelligent decisions about line
% breaks when vert bars are involved, use proper pairing versions of the
% vert-bar symbols according to context: \verb'\lvert n\rvert' instead of
% \verb'|n|'. With the nondirectional \verb'|' there is no way for \LaTeX\
% to reliably deduce which potential breakpoints are inside delimiters
% (more highly discouraged) and which are not.
%
% \item If you use the \pkg{german} package or some other package that
% turns double quote \verb'"' into a special character, you may encounter
% some problems with named math symbols of type mathbin, mathrel,
% mathopen, or mathclose in moving arguments. For example, \cs{leq} in a
% section title will be written to the \fn{.aux} file as something like
% \verb'\mathchar "3214'. This situation probably ought to be improved,
% but for now use \cs{protect}.
%
% \item Watch out for the \texttt{[} character at the beginning of a
% \env{dmath} or similar environment, if it is supposed to be interpreted
% as mathematical content rather than the start of the environment's
% optional argument.
%
% This is OK:
% \begin{verbatim}
% \begin{dmath}
% [\lambda,1]...
% \end{dmath}
% \end{verbatim}
% This will not work as expected:
% \begin{verbatim}
% \begin{dmath}[\lambda,1]...\end{dmath}
% \end{verbatim}
%
% \item Watch out for unpaired delimiter symbols (in display math only):
% \begin{verbatim}
% ( ) [ ] \langle \rangle \{ \} \lvert \rvert ...
% \end{verbatim}
% If an open delimiter is used without a close delimiter, or vice versa,
% it is normally harmless but may adversely affect line breaking. This is only
% for symbols that have a natural left or right directionality. Unpaired
% \cs{vert} and so on are fine.
%
% When a null delimiter is used as the other member of the pair
% (\verb'\left.' or \verb'\right.') this warning doesn't apply.
%
% \item If you inadvertently apply \cs{left} or \cs{right} to something
% that is not a delimiter, the error messages are likely to be a bit
% more confusing than usual. The normal \LaTeX\ response to an error such
% as
% \begin{verbatim}
% \left +
% \end{verbatim}
% is an immediate message
% \begin{verbatim}
% ! Missing delimiter (. inserted).
% \end{verbatim}
% When \thepkg is in use, \LaTeX\ will fail to realize anything is wrong
% until it hits the end of the math formula, or a closing delimiter
% without a matching opening delimiter, and then the first message is an
% apparently pointless
% \begin{verbatim}
% ! Missing \endgroup inserted.
% \end{verbatim}
%
% \end{itemize}
%
% \section{Examples}
%
% \renewcommand\theequation{\thesection.\arabic{equation}}
% % Knuth, SNA p74
% \begin{xio}
% Replace $j$ by $h-j$ and by $k-j$ in these sums to get [cf.~(26)]
% \begin{dmath}[label={sna74}]
% \frac{1}{6} \left(\sigma(k,h,0) +\frac{3(h-1)}{h}\right)
%   +\frac{1}{6} \left(\sigma(h,k,0) +\frac{3(k-1)}{k}\right)
% =\frac{1}{6} \left(\frac{h}{k} +\frac{k}{h} +\frac{1}{hk}\right)
%   +\frac{1}{2} -\frac{1}{2h} -\frac{1}{2k},
% \end{dmath}
% which is equivalent to the desired result.
% \end{xio}
%
% % Knuth, SNA 4.6.2, p387
% \begin{xio}
% \newcommand\mx[1]{\begin{math}#1\end{math}}% math expression
% %
% Now every column which has no circled entry is completely zero;
% so when $k=6$ and $k=7$ the algorithm outputs two more vectors,
% namely
% \begin{dseries}[frame]
% \mx{v^{[2]} =(0,5,5,0,9,5,1,0)},
% \mx{v^{[3]} =(0,9,11,9,10,12,0,1)}.
% \end{dseries}
% From the form of the matrix $A$ after $k=5$, it is evident that
% these vectors satisfy the equation $vA =(0,\dotsc,0)$.
% \end{xio}
%
% \begin{xio}
% \begin{dmath*}
% T(n) \hiderel{\leq} T(2^{\lceil\lg n\rceil})
%   \leq c(3^{\lceil\lg n\rceil}
%     -2^{\lceil\lg n\rceil})
%   <3c\cdot3^{\lg n}
%   =3c\,n^{\lg3}
% \end{dmath*}.
% \end{xio}
%
% \begin{xio}
% The reduced minimal Gr\"obner basis for $I^q_3$ consists of
% \begin{dgroup*}
% \begin{dmath*}
% H_1^3 = x_1 + x_2 + x_3
% \end{dmath*},
% \begin{dmath*}
% H_2^2 = x_1^2 + x_1 x_2 + x_2^2 - q_1 - q_2
% \end{dmath*},
% \begin{dsuspend}
% and
% \end{dsuspend}
% \begin{dmath*}
% H_3^1 = x_1^3 - 2x_1 q_1 - x_2 q_1
% \end{dmath*}.
% \end{dgroup*}
% \end{xio}
%
%
%
%\section{Technical notes on tag placement}
%
%The method used by the breqn package to place the equation number is
%rather more complicated than you might think, and the whole reason is
%to allow the number to stay properly centered on the total height even
%when the height fluctuates due to stretching or shrinking of the page.
%
%
%Consider the following equation:
%\begin{dmath}[number={3.15}]
%  N_{0} \simeq \left( \frac{\nu}{\lVert u\rVert_{H^{i}}} \right)
%  \lvert I\rvert^{-1/2}
%\end{dmath}
%It will have only one line, if the column width is not too narrow.
%
%Scrutinizing the vertical list will shed light on some of the basic properties
%shared by all breqn equations. After that we will look at what would happen if
%two or more lines were needed. The numbers added on the left in the following
%\cs{showlists} output mark the points of interest.
%\begin{verbatim}
%[1] \penalty 10000
%    \glue(\abovedisplayskip) 0.0
%    \penalty 10000
%    \glue(\belowdisplayskip) 0.0
%[2] \glue 4.0 plus 4.0
%    \glue(\lineskip) 1.0
%[3] \vbox(16.53902+0.0)x0.0, glue set 16.53902fil
%    .\glue 0.0 plus 1.0fil minus 1.0fil
%    \penalty 10000
%[4] \glue -8.51945
%[5] \hbox(7.5+2.5)x25.55563
%    .\OT1/cmr/m/n/10 (
%    .\OT1/cmr/m/n/10 3
%    .\OT1/cmr/m/n/10 .
%    .\OT1/cmr/m/n/10 1
%    .\OT1/cmr/m/n/10 5
%    .\kern 0.0
%    .\OT1/cmr/m/n/10 )
%    \penalty 10000
%[6] \glue(\parskip) -18.01956
%[7] \hbox(16.53902+9.50012)x360.0, glue set 1.78647
%\end{verbatim}
%\begin{enumerate}
%\item These four lines are a hidden display structure from \TeX's
%  primitive \texttt{\$\$} mechanism. It is used only to get the value
%  of \cs{predisplaysize} so that we can later calculate by hand
%  whether to use the short display skips or the regular ones. (The
%  reason that we have to do it by hand traces back to the fact that
%  \TeX\ 3.x does not allow unhboxing in math mode.) The penalties come
%  from \cs{predisplaypenalty} and \cs{postdisplaypenalty}, which were
%  locally set to 10000 to ensure there would be no unintended page
%  breaks at these glue nodes.
%
%\item These two glue nodes are the ones that would normally have been
%  produced at the top of a display; the first one is the above-display
%  skip node (though we had to put it in by hand with \cs{vskip}) and
%  the second one is the usual baselineskip/lineskip node.
%
%\item This is a dummy copy of the equation's first line, which is
%  thrown in here to get the proper value of baselineskip (or lineskip
%  in this case). Why do we need this? Because this ensures that we get
%  the top spacing right before we fiddle with the glue nodes
%  surrounding the equation number. And if the equation has a frame,
%  this box is a good place to add it from.
%
%\item This is a special glue node that brings us to the right vertical
%  position for adding the equation number. Its value is calculated
%  from the variables that you would expect, given the presence of the
%  dummy first line above the num- ber: starting position of the
%  equation, height of first line, total height of equation body. If
%  the equation body had more than one line, with stretchable glue
%  between the lines, half of the stretch would be added in this glue
%  node.
%
%\item The hbox containing the equation number.
%
%\item Backspace to bring the equation body to the right starting point. We use
%  \cs{parskip} to put this glue in place because we're going to get a
%  parskip node here in any case when we add the equation body with (in
%  essence). If we didn't do this we'd get two glue nodes instead of
%  one, to no purpose.
%
%\cs{\noindent} \cs{unhbox}\cs{EQ@box}.
%
%
%\item And lastly we see here the first line of the equation body,
%  which appears to have height 16.5pt and depth 9.5pt.
%\end{enumerate}
%
%For comparison, the vertical list produced from the above equation in
%standard \LaTeX\ would look like this, if the same values of
%columnwidth and abovedisplayskip are used:
%\begin{verbatim}
%[1] \penalty 10000
%[2] \glue(\abovedisplayskip) 4.0 plus 4.0
%    \glue(\lineskip) 1.0
%    \hbox(16.53902+9.50012)x232.94844
%[3] .\hbox(7.5+2.5)x25.55563
%    ..\hbox(7.5+2.5)x25.55563
%    ...\OT1/cmr/m/n/10 (
%    ...\OT1/cmr/m/n/10 3
%    ...\OT1/cmr/m/n/10 .
%    ...\OT1/cmr/m/n/10 1
%    ...\OT1/cmr/m/n/10 5
%    ...\kern 0.0
%    ...\OT1/cmr/m/n/10 )
%    .\kern101.49591
%[4] .\hbox(16.53902+9.50012)x105.8969
%    ...
%[5] \penalty 0
%[6] \glue(\belowdisplayskip) 4.0 plus 4.0
%    \glue(\lineskip) 1.0
%    \hbox(6.94444+1.94444)x345.0, glue set 62.1106fil
%\end{verbatim}
%
%\begin{enumerate}
%  \item \cs{predisplaypenalty}
%  \item \cs{abovedisplayskip}
%  \item  equation number box
%  \item equation body
%  \item \cs{postdisplaypenaltly}
%  \item \cs{belowdisplayskip}
%\end{enumerate}
%
%
% \section{Technical notes on Equation Layouts}
% \textbf{MJD [1998/12/28]}
%
% \providecommand{\qq}[1]{\textquotedblleft#1\textquotedblright}
% \providecommand{\mdash}{\textemdash}
% \providecommand{\ndash}{\textendash}
%
% \newcommand{\ititle}[1]{\textit{#1}}
%
% \newcommand{\LR}[2][.4]{%
%   \framebox[#1\displaywidth]{$\displaystyle{}#2$}%
% }
%
% \newcommand{\LHS}[1]{\LR[\relifactor]{#1}}
%
% \newdimen\relindent \newdimen\rhswd
%
% \newcommand{\dwline}{%
%   \hbox to\curdw{\vrule height1ex
%     \leaders\hrule height.55ex depth-.45ex\hfil
%     \tiny \space display width
%     \leaders\hrule height.55ex depth-.45ex\hfil
%     \vrule height1ex}%
% }
%
% \newenvironment{layout}[1][.15]{%
%   \noindent
%   $$\edef\curdw{\the\displaywidth}%
%     \def\relifactor{#1}%
%     \gdef\layoutcr{\cr}\def\\{\layoutcr}%
%     \binoppenalty 10000 \relpenalty 10000
%   \setbox8\vbox\bgroup
%     \advance\baselineskip .35\baselineskip
%     \advance\lineskip .35\baselineskip \lineskiplimit\lineskip
%     \relindent=#1\displaywidth
%     \rhswd=\displaywidth \advance\rhswd-\relindent
%     \global\row 0 \gdef\rhsskew{}%
%     \halign\bgroup \global\advance\row 1 $\hfil\displaystyle{}##$&%
%       \ifnum\row>1 \rhsskew \fi $\displaystyle{}##\hfil$\cr
% }{%
%   \crcr\egroup\egroup
%   \vcenter{\halign{\hfil##\hfil\cr
%     \hbox{\hss\dwline\hss}\cr\noalign{\vskip.6\baselineskip}\box8 \cr}}%
%   $$\relax
%   \ignorespacesafterend
% }
%
% \newcommand{\stagger}{\omit\span\gdef\layoutcr{\cr\omit\span}}
%
% \newcount\row
%
% \newcommand{\rhsskew}{}
% \newcommand{\skewleft}[1]{\gdef\rhsskew{\kern-#1\relax}}
%
%
% \subsection{Misc examples}
%
% Let us consider which of these have 50\% or more of wasted whitespace
% \emph{within the bounding box of the visible material}.
% \begin{layout}[.4]
% \LHS{L}&=\LR[.35]{R_{1}}\\
% &=\LR[.25]{R_{1}}
% \end{layout}
%
% \subsection{Ladder and step layouts}
%
% \subsubsection{Straight ladder layout}
% This is distinguished by a relatively short LHS and one or more RHS's of
% any length.
% \begin{layout}
% \LHS{L} &= \LR[.5]{R_{1}}\\
% &=\LR[.3]{R_{2}}\\
% &=\LR[.25]{R_{3}}\\
% &\qquad\ldots
% \end{layout}
% The simplest kind of equation that fits on one line and has only one RHS
% may be viewed as a trivial subcase of the straight ladder layout:
% \begin{layout}
% \LHS{L} &= \LR[.5]{R}
% \end{layout}
% If some of the RHS's are too wide to fit on a single line they may be
% broken at binary operator symbols such as plus or minus. This is still
% classified as a straight ladder layout if none of the fragments intrude
% into the LHS column, because the underlying parshape is the same.
% \begin{layout}
% \LHS{L} &= \LR[.4]{R_{1a}}\\
% &\quad +\LR[.6]{R_{1b}}\\
% &=\LR[.3]{R_{2}}\\
% &=\LR[.25]{R_{3a}}\\
% &\quad +\LR[.45]{R_{3b}}\\
% &\quad +\LR[.54]{R_{3c}}\\
% &\qquad\ldots
% \end{layout}
%
% \subsubsection{Skew ladder layout}
% \begin{layout}[.5]
% \skewleft{.35\displaywidth}
% \LHS{L}&= \LR[.3]{R_{1}}\\
% &=\LR[.6]{R_{2}}\\
% &=\LR[.25]{R_{3}}\\
% &\qquad\ldots
% \end{layout}
% In a skew ladder layout, the combined LHS width plus width of $R_{1}$
% does not exceed the available width, but one of the other RHS's is so
% wide that aligning its relation symbol with the others cannot be done
% without making it run over the right margin: $\mbox{width}(L) +
% \mbox{width}_{\mathrm{max}}(R_{i})>\mbox{width}_{\mathrm{avail}}$. In
% that case we next try aligning all but the first relation symbol,
% allowing all the $R_{i}$ after $R_1$ to shift leftward.
%
% \subsubsection{Drop ladder layout}
% \begin{layout}[.6]
% \makebox[.15\displaywidth][l]{\LHS{L}}\\
% &= \LR[.6]{R_{1}}\\
% &=\LR[.3]{R_{2}}\\
% &=\LR[.25]{R_{3}}\\
% &\qquad\ldots
% \end{layout}
% The drop ladder layout is similar to the skew ladder layout but with the
% width of $R_1$ too large for it to fit on the same line as the LHS. Then
% we move $R_1$ down to a separate line and try again to align all the
% relation symbols. Note that this layout consumes more vertical space
% than the skew ladder layout.
%
% \subsubsection{Step layout}
% \begin{layout}[.6]
% \stagger
% \LHS{R_{a}}\\
% \qquad + \LR[.7]{R_{b}}\\
% \qquad\qquad + \LR[.6]{R_{c}}\\
% \qquad\qquad\qquad + \LR[.45]{R_{d}}\\
% \qquad\qquad\qquad\qquad\ldots
% \end{layout}
% The chief characteristic of the step layout is that there is no relation
% symbol, so that the available line breaks are (usually) all at binary
% operator symbols. Let $w_1$ and $w_l$ be the widths of the first and
% last fragments. We postulate that the ideal presentation is as follows:
% Choose a small stairstep indent $I$ (let's say 1 or 2 em). We want the
% last fragment to be offset at least $I$ from the start of the first
% fragment, and to end at least $I$ past the end of the first fragment. If
% there are only two lines these requirements determine a target width
% $w_T=\max(w_1+I,w_l+I)$. If there are more than two lines ($l>2$) then
% use $w_T = \max(w_1 + (l-1)I, w_l+I, w_{\mathrm{avail}}$ and reset $I$
% to $w_T/(l-1)$ if $w_T = w_{\mathrm{avail}}$.
%
% Furthermore, we would like the material to be distributed as evenly as
% possible over all the lines rather than leave the last line exceedingly
% short. If the total width is $1.1(\mbox{width}_{\mathrm{avail}})$, we
% don't want to have .9 of that on line 1 and .2 of it on line 2:
% \begin{layout}[.9]
% \stagger
% \LHS{R_{a}\hfil+\hfil R_{b}\hfil+\hfil R_{c}}\\
% \qquad + \LR[.1]{R_{d}}
% \end{layout}
% Better to split it as evenly as possible, if the available breakpoints
% permit.
% \begin{layout}[.5]
% \stagger
% \LHS{R_{a}\hfil+\hfil R_{b}}\\
% \qquad + \LR[.5]{R_{c}\hfil+\hfil R_d}
% \end{layout}
% A degenerate step layout may arise if an unbreakable fragment of
% the equation is so wide that indenting it to its appointed starting
% point would cause it to run over the right margin. In that case, we want
% to shift the fragment leftward just enough to bring it within the right
% margin:
% \begin{layout}[.6]
% \stagger
% \LHS{L_{a}}\\
% \quad + \LR[.8]{L_{b}}\\
% \qquad + \LR[.7]{L_{c}}\\
% \; + \LR[.87]{L_{d}}\\
% \qquad\ldots
% \end{layout}
% And then we may want to regularize the indents as in the drop ladder
% layout. Let's call this a dropped step layout:
% \begin{layout}[.6]
% \stagger
% \LHS{L_{a}}\\
% \quad + \LR[.8]{L_{b}}\\
% \quad + \LR[.7]{L_{c}}\\
% \quad + \LR[.87]{L_{d}}\\
% \qquad\ldots
% \end{layout}
%
% \subsection{Strategy}
%
% Here is the basic procedure for deciding which equation layout to use,
% before complications like equation numbers and delimiter clearance come
% into the picture. Let $A$ be the available width, $w_{\mathrm{total}}$
% the total width of the equation contents, $w(L)$ the width of the
% left-hand side, $w_{\max}(R)$ the max width of the right-hand sides, $I$
% the standard indent for step layout, and $O$ the standard offset for
% binary operators if a break occurs in the middle of an RHS. Also let
% $t_L$ and $t_R$ represent certain thresholds for the width of the LHS or
% the RHS at which a layout decision may change, as explained below.
%
% \begin{small}
% \begin{enumerate}
% \renewcommand{\labelenumi}{(\theenumi)}
% \item \ititle{Does everything fit on one line?}\label{i:LR}
%   $w_{\mathrm{total}}\leq A$?
%
% Yes: print the equation on a single line (done).
%
% No: Check whether the equation has both LHS and RHS (\ref{i:lhs-check}).
%
% \item \ititle{Is there a left-hand side?}\label{i:lhs-check}
% Are there any relation symbols in the equation?
%
% Yes: Try a ladder layout (\ref{i:ladder}).
%
% No: Try a step layout (\ref{i:step}).
%
% \item\ititle{Does the LHS leave room to fit the widest RHS?}\label{i:ladder}
%   $w(L)+w_{\max}(R)<A$?
%
% Yes: Use a straight ladder layout (\ref{i:straight-ladder}).
%
% No: Check the width of the LHS (\ref{i:check-lhs}).
%
% \item\ititle{Is the LHS relatively short?}\label{i:check-lhs}
% $w(L)\leq t_L$? (where $t_L$ is typically $0.4A$).
%
% Yes: Subdividing one or more of the RHS's may permit us to use a
% straight ladder layout (\ref{i:straight-ladder}).
%
% No: The straight ladder layout is unlikely to work.
% Try a skew or drop ladder layout (\ref{i:skew-drop}).
%
% \item\ititle{Straight ladder layout}\label{i:straight-ladder}
% Set up a straight ladder parshape [0pt $A$ $w(L)$ $A-w(L)$] and run
% a trial break. If the combined width of the LHS plus the longest RHS is
% no greater than $A$ then we should get a satisfactory layout with all
% line breaks occurring at major division points (relation symbols).
% Otherwise, we hope, some additional line breaks at minor division points
% will allow everything to fit within the text column.
%
% \ititle{Line breaks OK?}
%
% \begingroup \hbadness=9999
% Yes: The straight ladder layout succeeded
%   (done).\par\endgroup
%
% No: Try a skew or drop ladder layout (\ref{i:skew-drop}).
%
% \item\ititle{Do the LHS and the first RHS fit on one
%     line?}\label{i:skew-drop} $w(L)+w(R_1) \leq A$?
%
% Yes: Try a skew ladder layout (\ref{i:skew}).
%
% No: Try a drop ladder layout (\ref{i:drop}).
%
% \item\ititle{Skew ladder layout}\label{i:skew}
% Set up a parshape [0pt $A$ $I$ $A-I$] and run a trial break.
%
% \ititle{Line breaks OK?}
%
% Yes: Skew ladder layout succeeded (done).
%
% No: One of the unbreakable fragments of the $R_i$ ($i>1$) is wider than
% $A-I$; try an almost-columnar layout (\ref{i:almost-columnar}).
%
% \item\ititle{Drop ladder layout}\label{i:drop}
% Set up a parshape [0pt $w(L)$ $I$ $A-I$] and run a trial break.
% This is the same parshape as for a skew ladder layout except that the
% width of the first line is limited to the LHS width, so that the RHS is
% forced to drop down to the next line.
%
% \ititle{Line breaks OK?}
%
% Yes: Drop ladder layout succeeded (done).
%
% No: One of the unbreakable fragments of the $R_i$ ($i>1$) is wider than
% $A-I$; try an almost-columnar layout (\ref{i:almost-columnar}).
%
% \item\ititle{Almost-columnar layout}\label{i:almost-columnar}
% This presupposes a trial break that yielded a series of expressions or
% fragments, one per line. Let $w(F)$ denote the width of the first
% fragment and $w(R_i)$ the widths of the remaining fragments.
% Set up a parshape [0pt $w(F)$ $A-w_{\max}(R_i)$ $w_{\max}(R_i)$]: in other
% words, set the first line flush left and the longest line flush right
% and all other lines indented to the same position as the longest line.
% But as a matter of fact there is one other refinement for extreme cases:
% if $w_{\max}(R_i)>A$ then the parshape can be simplified without loss to
% [0pt $w(F)$ 0pt $A$]\mdash for that is the net effect of substituting
% $\min(A,w_{\max})$ in stead of $w_{\max}$.
% (Done.)
%
% \item\ititle{Step layout}\label{i:step} Set target width $w_T$ to $A -
%   2I$.  Set parshape to [0pt $w_T$ $I$ $w_T -I$ $2I$ $w_T -2I$ \ldots\
%   $(l-1)I$ $w_T - (l-1)I$], where $l=\lceil w_{\mathrm{total}}/A\rceil$
%   is the expected number of lines that will be required.  Trial break
%   with that parshape in order to find out the width of the last line.
%
% \ititle{Indents OK?}
%
% Yes: Step layout succeeded (done).
%
% No: One of the fragments is too wide to fit in
% the allotted line width, after subtracting the indent specified by the
% parshape. Try a dropped step layout (\ref{i:drop-step})
%
% \item\ititle{Dropped step layout}\label{i:drop-step} Set up a parshape
%   [0pt $A$ $I$ $A-I$] and run a trial break.  Note that this is actually
%   the same parshape as for a skew ladder layout.
%
% \ititle{Line breaks OK?}
%
% Yes: Dropped step layout succeeded (done).
%
% No: One of the unbreakable fragments of the $R_i$ ($i>1$) is wider than
% $A-I$; as a last resort try an almost-columnar layout (\ref{i:almost-columnar}).
%
% \end{enumerate}
% \par\end{small}
%
% \section{To do}
%
% \begin{itemize}
% \item Handling of QED
% \item Space between \verb'\end{dmath}' and following punctuation will
% prevent the punctuation from being drawn into the equation.
% \item Overriding the equation layout
% \item Overriding the placement of the equation number
% \item \qq{alignid} option for more widely separated equations where
%   shared alignment is desired (requires two passes)
% \item Or maybe provide an \qq{alignwidths} option where you give
%   lhs/rhs width in terms of ems? And get feedback later on discrepancies
%   with the actual measured contents?
% \item \cs{intertext} not needed within dgroup! But currently there are
%   limitations on floating objects within dgroup.
% \item \verb'align={1}' or 2, 3, 4 expressing various levels of demand
%   for group-wide alignment. Level 4 means force alignment even if some
%   lines then have to run over the right margin! Level 1, the default,
%   means first break LHS-RHS equations as if it occurred by itself, then
%   move them left or right within the current line width to align them if
%   possible. Levels 2 and 3 mean try harder to align but give up if
%   overfull lines result.
% \item Need an \cs{hshift} command to help with alignment of
%   lines broken at a discretionary times sign. Also useful for adjusting
%   inside-delimiter breaks.
% \end{itemize}
%
% \StopEventually{}
% \clearpage
% \newgeometry{left=4cm}
% \part{Implementation}
%
%
% The package version here is Michael's v0.90 updated by Bruce
% Miller. Michael's changes between v0.90 and his last v0.94 will be
% incorporated where applicable.
%
%
% The original sources of \pkg{breqn} and related files exist in a
% non-dtx format devised by Michael Downes himself.
% Lars Madsen has kindly written a Perl script for transforming the
% original source files into near-perfect dtx state, requiring only
% very little hand tuning. Without his help it would have been nigh
% impossible to incorporate the original sources with Michael's
% comments. A big, big thank you to him.
%
%
%
% \section{Introduction}
% The \pkg{breqn} package provides environments
% \env{dmath}, \env{dseries}, and \env{dgroup} for
% displayed equations with \emph{automatic line breaking},
% including automatic indention of relation symbols and binary operator
% symbols at the beginning of broken lines.    These environments
% automatically pull in following punctuation so that it can be written in
% a natural way.    The \pkg{breqn} package also provides a
% \env{darray} environment similar to the \env{array}
% environment but using \cs{displaystyle} for all the array cells and
% providing better interline spacing (because the vertical ruling
% feature of \env{array} is dropped).
% These are all autonumbered environments like \env{equation}
% and have starred forms that don't add a number.    For a more
% comprehensive and detailed description of the features and intended
% usage of the \pkg{breqn} package see \fn{breqndoc.tex}.
%
%
%
%
% \section{Strategy}
% Features of particular note are the ability to have
% linebreaks even within a \cs{left} \ndash  \cs{right} pair of
% delimiters, and the automatic alignment on relations and binary
% operators of a split equation.    To make \env{dmath} handle
% all this, we begin by setting the body of the equation in a special
% paragraph form with strategic line breaks whose purpose is not to
% produce line breaks in the final printed output but rather to mark
% significant points in the equation and give us entry points for
% unpacking \cn{left} \ndash  \cn{right} boxes.
% After the initial typesetting, we take the resulting stack of line
% fragments and, working backward, splice them into a new, single-line
% paragraph; this will eventually be poured into a custom parshape, after
% we do some measuring to calculate what that parshape should be.
% This streamlined horizontal list may contain embedded material
% from user commands intended to alter line breaks, horizontal alignment,
% and interline spacing; such material requires special handling.
%
% To make the `shortskip' possibility work even for
% multiline equations, we must plug in a dummy \tex  display to give us
% the value of \cs{predisplaysize}, and calculate for ourselves when
% to apply the short skips.
%
% In order to measure the equation body and do various
% enervating calculations on whether the equation number will fit and so
% on, we have to set it in a box.    Among other things, this means
% that we can't unhbox it inside \dbldollars  \dots  \dbldollars , or
% even \verb"$" \dots  \verb"$": \tex  doesn't allow you to
% \cs{unhbox} in math mode.    But we do want to unhbox it rather
% than just call \cs{box}, otherwise we can't take advantage of
% available shrink from \cs{medmuskip} to make equations shrink to
% fit in the available width.    So even for simple one-line equations
% we are forced to fake a whole display without going through \tex 's
% primitive display mechanism (except for using it to get
% \cs{predisplaysize} as mentioned above).
%
%
% In the case of a framed equation body, the current implementation is
% to set the frame in a separate box, of width zero and height zero,
% pinned to the upper left corner of the equation body, and then print the
% equation body on top of it.
% For attaching an equation number it would be much simpler to wrap
% the equation body in the frame and from then on treat the body as a
% single box instead of multiple line boxes.
% But I had a notion that it might be possible some day to support
% vertical stretching of the frame.
%
%
%
%
% \section{Prelim}
%
%    \begin{macrocode}
%<*package>
\NeedsTeXFormat{LaTeX2e}
%    \end{macrocode}
%
% Declare package name and date.
%    \begin{macrocode}
\RequirePackage{expl3}
\ProvidesExplPackage{breqn}{2019/10/15}{0.98g}{Breaking equations}
%    \end{macrocode}
%   Regrettably, \pkg{breqn} is internally a mess, so we have to take
%   some odd steps.
%    \begin{macrocode}
\ExplSyntaxOff
%    \end{macrocode}
%
%
% \section{Package options}
%
% First we need to get the catcodes sorted out.
%    \begin{macrocode}
\edef\breqnpopcats{%
  \catcode\number`\"=\number\catcode`\"
  \relax}
\AtEndOfPackage{\breqnpopcats}%
\catcode`\^=7 \catcode`\_=8 \catcode`\"=12 \relax
\DeclareOption{mathstyleoff}{%
  \PassOptionsToPackage{mathstyleoff}{flexisym}%
}
%    \end{macrocode}
% Process options.
%    \begin{macrocode}
\ProcessOptions\relax
%    \end{macrocode}
%
%
%
%
% \section{Required packages}
%
% Although breqn can be loaded without amsmath, there's not much point, and explicitly
% loading amsmath here prevents loading-order issues later.
%    \begin{macrocode}
\RequirePackage{amsmath}
%    \end{macrocode}
%
% The \pkg{flexisym} package makes it possible to attach
% extra actions to math symbols, in particular mathbin, mathrel, mathopen,
% and mathclose symbols.
% Normally it would suffice to call \cs{RequirePackage} without
% any extra testing, but the nature of the package is such that it is
% likely to be called earlier with different (no) options.
% Then is it really helpful to be always warning the user about
% \quoted{Incompatible Package Options!}?
% I don't think so.
%    \begin{macrocode}
\@ifpackageloaded{flexisym}{}{%
  \RequirePackage{flexisym}[2009/08/07]
    \edef\breqnpopcats{\breqnpopcats
    \catcode\number`\^=\number\catcode`\^
    \catcode\number`\_=\number\catcode`\_
  }%
  \catcode`\^=7 \catcode`\_=8 \catcode`\"=12 \relax
}
%    \end{macrocode}
% The \pkg{keyval} package for handling equation options and
% \pkg{calc} to ease writing computations.
%    \begin{macrocode}
\RequirePackage{keyval,calc}\relax
%    \end{macrocode}
%
%
% \begin{macro}{\breqnsetup}
%    \begin{macrocode}
\newcommand{\breqnsetup}[1]{\setkeys{breqn}{#1}}
%    \end{macrocode}
% \end{macro}
%
%
%
%
% \section{Some useful tools}
%
% \begin{macro}{\@nx}
% \begin{macro}{\@xp}
% The comparative brevity of \cs{@nx} and \cs{@xp} is
% valuable not so much for typing convenience as for reducing visual
% clutter in code sections that require a lot of expansion control.
%    \begin{macrocode}
\let\@nx\noexpand
\let\@xp\expandafter
%    \end{macrocode}
% \end{macro}
% \end{macro}
%
%
% \begin{macro}{\@emptytoks}
% Constant empty token register, analogous to \cs{@empty}.
%    \begin{macrocode}
\@ifundefined{@emptytoks}{\newtoks\@emptytoks}{}
%    \end{macrocode}
% \end{macro}
%
%
% \begin{macro}{\f@ur}
% Constants 0\ndash 3 are provided in plain \tex , but not 4.
%    \begin{macrocode}
\chardef\f@ur=4
%    \end{macrocode}
% \end{macro}
%
%
% \begin{macro}{\inf@bad}
% \cs{inf@bad} is for testing box badness.
%    \begin{macrocode}
\newcount\inf@bad \inf@bad=1000000
%    \end{macrocode}
% \end{macro}
%
%
% \begin{macro}{\maxint}
%
% We want to use \cs{maxint} rather than coerced
% \cs{maxdimen} for \cs{linepenalty} in one place.
%    \begin{macrocode}
\newcount\maxint \maxint=2147483647
%    \end{macrocode}
% \end{macro}
%
%
% \begin{macro}{\int@a}
% \begin{macro}{\int@b}
% \begin{macro}{\int@b}
%
% Provide some shorter aliases for various scratch registers.
%    \begin{macrocode}
\let\int@a=\@tempcnta
\let\int@b=\@tempcntb
\let\int@c=\count@
%    \end{macrocode}
% \end{macro}
% \end{macro}
% \end{macro}
%
%
% \begin{macro}{\dim@a}
% \begin{macro}{\dim@b}
% \begin{macro}{\dim@c}
% \begin{macro}{\dim@d}
% \begin{macro}{\dim@e}
% \begin{macro}{\dim@A}
%
% Same for dimen registers.
%    \begin{macrocode}
\let\dim@a\@tempdima
\let\dim@b\@tempdimb
\let\dim@c\@tempdimc
\let\dim@d\dimen@
\let\dim@e\dimen@ii
\let\dim@A\dimen@i
%    \end{macrocode}
% \end{macro}
% \end{macro}
% \end{macro}
% \end{macro}
% \end{macro}
% \end{macro}
%
%
% \begin{macro}{\skip@a}
% \begin{macro}{\skip@b}
% \begin{macro}{\skip@c}
%
% Same for skip registers.
%    \begin{macrocode}
\let\skip@a\@tempskipa
\let\skip@b\@tempskipb
\let\skip@c\skip@
%    \end{macrocode}
% \end{macro}
% \end{macro}
% \end{macro}
%
%
% \begin{macro}{\toks@a}
% \begin{macro}{\toks@b}
% \begin{macro}{\toks@c}
% \begin{macro}{\toks@d}
% \begin{macro}{\toks@e}
% \begin{macro}{\toks@f}
%
% Same for token registers.
%    \begin{macrocode}
\let\toks@a\@temptokena
\let\toks@b\toks@
\toksdef\toks@c=2
\toksdef\toks@d=4
\toksdef\toks@e=6
\toksdef\toks@f=8
%    \end{macrocode}
% \end{macro}
% \end{macro}
% \end{macro}
% \end{macro}
% \end{macro}
% \end{macro}
%
%
% \begin{macro}{\abs@num}
% We need an absolute value function for comparing
% penalties.
%    \begin{macrocode}
\def\abs@num#1{\ifnum#1<\z@-\fi#1}
%    \end{macrocode}
% \end{macro}
%
%
% \begin{macro}{\@ifnext}
% \begin{macro}{\@ifnexta}
% The \cs{@ifnext} function is a variation of
% \cs{@ifnextchar} that doesn't skip over intervening whitespace.
% We use it for the optional arg of \dbslash  inside
% \env{dmath} \etc  because we don't want
% unwary users to be tripped up by an unexpected attempt on \latex 's part
% to interpret a bit of math as an optional arg:
% \begin{literalcode}
% \begin{equation}
% ...\\
% [z,w]...
% \end{equation}
% \end{literalcode}
% .
%    \begin{macrocode}
\def\@ifnext#1#2#3{%
  \let\@tempd= #1\def\@tempa{#2}\def\@tempb{#3}%
  \futurelet\@tempc\@ifnexta
}
%    \end{macrocode}
% Switch to \cs{@tempa} iff the next token matches.
%    \begin{macrocode}
\def\@ifnexta{\ifx\@tempc\@tempd \let\@tempb\@tempa \fi \@tempb}
%    \end{macrocode}
% \end{macro}
% \end{macro}
%
%
% \begin{macro}{\breqn@ifstar}
% Similarly let's remove space-skipping from \cs{@ifstar}
% because in some rare case of \dbslash  inside an equation, followed by
% a space and a \verb"*" where the \verb"*" is intended as the math
% binary operator, it would be a disservice to gobble the star as an
% option of the \dbslash  command.    In all other contexts the chance
% of having a space \emph{before} the star is extremely small: either
% the command is a control word which will get no space token after it in
% any case because of \tex 's tokenization rules; or it is a control
% symbol such as \dbslash  \verb"*" which is exceedingly unlikely to be
% written as \dbslash  \verb"*" by any one who really wants the
% \verb"*" to act as a modifier for the \dbslash  command.
%    \begin{macrocode}
\def\breqn@ifstar#1#2{%
  \let\@tempd*\def\@tempa*{#1}\def\@tempb{#2}%
  \futurelet\@tempc\@ifnexta
}
%    \end{macrocode}
% \end{macro}
%
%
% \begin{macro}{\breqn@optarg}
% Utility function for reading an optional arg
% \emph{without} skipping over any intervening spaces.
%    \begin{macrocode}
\def\breqn@optarg#1#2{\@ifnext[{#1}{#1[#2]}}
%    \end{macrocode}
% \end{macro}
%
%
% \begin{macro}{\@True}
% \begin{macro}{\@False}
% \begin{macro}{\@Not}
% \begin{macro}{\@And}
% After \verb"\let\foo\@True" the test
% \begin{literalcode}
% \if\foo
% \end{literalcode}
% evaluates to true.    Would rather avoid \cs{newif} because it
% uses three csnames per Boolean variable; this uses only one.
%    \begin{macrocode}
\def\@True{00}
\def\@False{01}
\def\@Not#1{0\ifcase#11 \or\@xp 1\else \@xp 0\fi}
\def\@And#1#2{0\ifcase#1#2 \@xp 0\else \@xp 1\fi}
\def\@Or#1#2{0\ifnum#1#2<101 \@xp 0\else \@xp 1\fi}
%    \end{macrocode}
% \end{macro}
% \end{macro}
% \end{macro}
% \end{macro}
%
%
%
%    \begin{macrocode}
\def\theb@@le#1{\if#1 True\else False\fi}
%    \end{macrocode}
% \begin{macro}{\freeze@glue}
%
% Remove the stretch and shrink from a glue register.
%    \begin{macrocode}
\def\freeze@glue#1{#11#1\relax}
%    \end{macrocode}
% \end{macro}
% \begin{macro}{\z@rule}
% \begin{macro}{\keep@glue}
% Note well
% the intentional absence of \cs{relax} at the end of the replacement
% text of \cs{z@rule}; use it with care.
%    \begin{macrocode}
\def\z@rule{\vrule\@width\z@}% no \relax ! use with care
%    \end{macrocode}
% Different ways to keep a bit of glue from disappearing at the
% beginning of a line after line breaking:
% \begin{itemize}
% \item Zero-thickness rule
% \item Null character
% \item \cs{vadjust}\verb"{}" (\texbook , Exercise ??)
% \end{itemize}
% The null character idea would be nice except it
% creates a mathord which then screws up math spacing for \eg  a following
% unary minus sign.    (the vrule \emph{is} transparent to
% the math spacing).    The vadjust is the cheapest in terms of box
% memory\mdash it vanishes after the pass through \tex 's
% paragrapher.
% It is what I would have used, except that the equation contents get
% run through two paragraphing passes, once for breaking up LR boxes and
% once for the real typesetting.
% If \cs{keep@glue} were done with an empty vadjust, it would
% disappear after the first pass and\mdash in particular\mdash the
% pre-bin-op adjustment for relation symbols would disappear at a line break.
%    \begin{macrocode}
\def\keep@glue{\z@rule\relax}
%    \end{macrocode}
% \end{macro}
% \end{macro}
%
%
% \begin{macro}{\replicate}
%
% This is a fully expandable way of making N copies of a token
% list.
% Based on a post of David Kastrup to comp.text.tex circa January
% 1999.
% The extra application of \cs{number} is needed for maximal
% robustness in case the repeat count N is given in some weird \tex  form
% such as \verb|"E9| or \verb|\count9|.
%    \begin{macrocode}
% usage: \message{H\replicate{5}{i h}ow de doo dee!}
\begingroup \catcode`\&=11
\gdef\replicate#1{%
  \csname &\expandafter\replicate@a\romannumeral\number\number#1 000q\endcsname
}
\endgroup
%    \end{macrocode}
% \end{macro}
%
%
% \begin{macro}{\replicate@a}
%    \begin{macrocode}
\long\def\replicate@a#1#2\endcsname#3{#1\endcsname{#3}#2}
%    \end{macrocode}
% \end{macro}
%
%
% \begin{macro}{\8m}% fix
%    \begin{macrocode}
\begingroup \catcode`\&=11
\long\gdef\&m#1#2{#1\csname &#2\endcsname{#1}}
\endgroup
%    \end{macrocode}
% \end{macro}
%
%
% \begin{macro}{\8q}% fix
%    \begin{macrocode}
\@xp\let\csname\string &q\endcsname\@gobble
%    \end{macrocode}
% \end{macro}
%
% \begin{macro}{\mathchars@reset}
%
% Need to patch up this function from flexisym a little, to better
% handle certain constructed symbols like \cs{neq}.
%    \begin{macrocode}
\ExplSyntaxOn
\g@addto@macro\mathchars@reset{%
  %\let\@symRel\@secondoftwo \let\@symBin\@secondoftwo
  %\let\@symDeL\@secondoftwo \let\@symDeR\@secondoftwo
  %\let\@symDeB\@secondoftwo
  \cs_set_eq:NN \math_csym_Rel:Nn \use_ii:nn
  \cs_set_eq:NN \math_csym_Bin:Nn \use_ii:nn
  \cs_set_eq:NN \math_csym_DeL:Nn \use_ii:nn
  \cs_set_eq:NN \math_csym_DeR:Nn \use_ii:nn
  \cs_set_eq:NN \math_csym_DeB:Nn \use_ii:nn
}
\ExplSyntaxOff
%    \end{macrocode}
% \end{macro}
%
%
% \begin{macro}{\eq@cons}
%
% \latex 's \cs{@cons} appends to the end of a list, but we need
% a function that adds material at the beginning.
%    \begin{macrocode}
\def\eq@cons#1#2{%
  \begingroup \let\@elt\relax \xdef#1{\@elt{#2}#1}\endgroup
}
%    \end{macrocode}
% \end{macro}
%
% \begin{macro}{\@saveprimitive}
% If some preceding package redefined one of the
% primitives that we must change, we had better do some checking to make
% sure that we are able to save the primitive meaning for internal use.
% This is handled by the \cs{@saveprimitive} function.    We
% follow the example of \cs{@@input} where the primitive meaning is
% stored in an internal control sequence with a \verb"@@" prefix.
% Primitive control sequences can be distinguished by the fact that
% \cs{string} and \cs{meaning} return the same information.
% Well, not quite all: \cs{nullfont} and \cs{topmark}
% and the other \cs{...mark} primitives being the exceptions.
%    \begin{macrocode}
\providecommand{\@saveprimitive}[2]{%
  \begingroup
  \edef\@tempa{\string#1}\edef\@tempb{\meaning#1}%
  \ifx\@tempa\@tempb \global\let#2#1%
  \else
%    \end{macrocode}%
% If [arg1] is no longer primitive, then we are in trouble unless
% [arg2] was already given the desired primitive meaning somewhere
% else.
%    \begin{macrocode}
    \edef\@tempb{\meaning#2}%
    \ifx\@tempa\@tempb
    \else \@saveprimitive@a#1#2%
    \fi
  \fi
  \endgroup
}
%    \end{macrocode}
% Aux function, check for the special cases.
% Most of the time this branch will be skipped so we can
% stuff a lot of work into it without worrying about speed costs.
%    \begin{macrocode}
\providecommand\@saveprimitive@a[2]{%
  \begingroup
  \def\@tempb##1#1##2{\edef\@tempb{##2}\@car{}}%
  \@tempb\nullfont{select font nullfont}%
    \topmark{\string\topmark:}%
    \firstmark{\string\firstmark:}%
    \botmark{\string\botmark:}%
    \splitfirstmark{\string\splitfirstmark:}%
    \splitbotmark{\string\splitbotmark:}%
    #1{\string#1}%
    \@nil % for the \@car
  \edef\@tempa{\expandafter\strip@prefix\meaning\@tempb}%
  \edef\@tempb{\meaning#1}%
  \ifx\@tempa\@tempb \global\let#2#1%
  \else
    \PackageError{breqn}%
      {Unable to properly define \string#2; primitive
      \noexpand#1no longer primitive}\@eha
    \fi
  \fi
  \endgroup
}
%    \end{macrocode}
% \end{macro}
%
%
% \begin{macro}{\@@math}
% \begin{macro}{\@@endmath}
% \begin{macro}{\@@display}
% \begin{macro}{\@@enddisplay}
% Move the math-start and math-end functions into control
% sequences.    If I were redesigning \tex  I guess I'd put these
% functions into primitive control words instead of linking them to a
% catcode.    That way \tex  would not have to do the special
% lookahead at a \verb"$" to see if there's another one coming up.
% Of course that's related to the question of how to provide user
% shorthand for common constructions: \tex , or an editing interface of
% some sort.
%    \begin{macrocode}
\begingroup \catcode`\$=\thr@@ % just to make sure
  \global\let\@@math=$ \gdef\@@display{$$}% $$$
\endgroup
\let\@@endmath=\@@math
\let\@@enddisplay=\@@display
%    \end{macrocode}
% \end{macro}
% \end{macro}
% \end{macro}
% \end{macro}
%
%
% \begin{macro}{\@@insert}
% \begin{macro}{\@@mark}
% \begin{macro}{\@@vadjust}
% Save the primitives \cs{vadjust}, \cs{insert},
% \cs{mark} because we will want to change them locally during
% equation measuring to keep them from getting in the way of our vertical
% decomposition procedures.    We follow the example of
% \cs{@@input}, \cs{@@end}, \cs{@@par} where the primitive
% meaning is stored in an internal control sequence with a \verb"@@"
% prefix.
%    \begin{macrocode}
\@saveprimitive\vadjust\@@vadjust
\@saveprimitive\insert\@@insert
\@saveprimitive\mark\@@mark
%    \end{macrocode}
% \end{macro}
% \end{macro}
% \end{macro}
%
%
%
%
% \section{Debugging}
% Debugging help.
%    \begin{macrocode}
%<*trace>
\errorcontextlines=2000\relax
\typeout{BREQN DEBUGGING MODE ACTIVE}
%    \end{macrocode}
%
% \begin{macro}{\breqn@debugmsg}
% Print a debugging message.
%    \begin{macrocode}
\long\def\breqn@debugmsg#1{\GenericWarning{||}{||=\space#1}}
%    \end{macrocode}
% \end{macro}
%
% \begin{macro}{\debugwr}
% Sometimes the newline behavior of \cs{message} is
% unsatisfactory; this provides an alternative.
%    \begin{macrocode}
\def\debugwr#1{\immediate\write\sixt@@n{||= #1}}
%    \end{macrocode}
% \end{macro}
%
%
% \begin{macro}{\debug@box}
% Record the contents of a box in the log file, without stopping.
%    \begin{macrocode}
\def\debug@box#1{%
  \batchmode{\showboxbreadth\maxdimen\showboxdepth99\showbox#1}%
  \errorstopmode
}
%    \end{macrocode}
% \end{macro}
%
%
% \begin{macro}{\eqinfo}
% Show lots of info about the material before launching into the
% trials.
%    \begin{macrocode}
\def\eqinfo{%
  \debug@box\EQ@copy
  \wlog{!! EQ@copy: \the\wd\EQ@copy\space x
    \the\ht\EQ@copy+\the\dp\EQ@copy
  }%
}
%    \end{macrocode}
% \end{macro}
%
%
% \begin{macro}{\debug@para}
% Check params that affect line breaking.
%    \begin{macrocode}
\def\debug@para{%
  \debugwr{\hsize\the\hsize, \parfillskip\the\parfillskip}%
  \breqn@debugmsg{\leftskip\the\leftskip, \rightskip\the\rightskip}%
  \breqn@debugmsg{\linepenalty\the\linepenalty, \adjdemerits\the\adjdemerits}%
  \breqn@debugmsg{\pretolerance\the\pretolerance, \tolerance\the\tolerance,
    \parindent\the\parindent}%
}
%    \end{macrocode}
% \end{macro}
%
%
%    \begin{macrocode}
%</trace>
%    \end{macrocode}
%
%
%
%
% \section{The \cs{listwidth} variable}
% The dimen variable \cs{listwidth} is \cs{linewidth}
% plus \cs{leftmargin} plus \cs{rightmargin}, which is typically
% less than \cs{hsize} if the list depth is greater than one.
% In case a future package will provide this variable, define it only
% if not yet defined.
%    \begin{macrocode}
\@ifundefined{listwidth}{\newdimen\listwidth}{}
\listwidth=\z@
%    \end{macrocode}
%
%
%
%
% \section{Parameters}
%
% Here follows a list of parameters needed.
%
% \begin{macro}{\eqfontsize}
% \begin{macro}{\eqcolor}
% \begin{macro}{\eqmargin}
% \begin{macro}{\eqindent}
% \begin{macro}{\eqbinoffset}
% \begin{macro}{\eqnumside}
% \begin{macro}{\eqnumplace}
% \begin{macro}{\eqnumsep}
% \begin{macro}{\eqnumfont}
% \begin{macro}{\eqnumform}
% \begin{macro}{\eqnumsize}
% \begin{macro}{\eqnumcolor}
% \begin{macro}{\eqlinespacing}
% \begin{macro}{\eqlineskip}
% \begin{macro}{\eqlineskiplimit}
% \begin{macro}{\eqstyle}
% \begin{macro}{\eqinterlinepenalty}
% \begin{macro}{\intereqpenalty}
% \begin{macro}{\intereqskip}
%
% Note: avoid M, m, P, p because they look like they might be the
% start of a keyword \quoted{minus} or \quoted{plus}.    Then
% \tex  looks further to see if the next letter is i or l.    And if
% the next thing is an undefined macro, the attempt to expand the macro
% results in an error message.
%    \begin{macrocode}
\def\eqfontsize{}         % Inherit from context    [NOT USED?]
\def\eqcolor{black}       % Default to black        [NOT USED?]
\newdimen\eqnumsep \eqnumsep=10pt        % Min space between equ number and body
\newdimen\eqmargin \eqmargin=8pt         % For `multline' gap emulation
%    \end{macrocode}
% The \cs{eqindent} and \cs{eqnumside} variables need to
% have their values initialized from context, actually.    But
% that takes a bit of work, which is postponed till later.
%    \begin{macrocode}
\def\eqindent{C}%         % C or I, centered or indented
\def\eqnumside{R}%        % R or L, right or left
\def\eqnumplace{M}%       % M or T or B, middle top or bottom
%    \end{macrocode}
% Typesetting the equation number is done thus:
% \begin{literalcode}
% {\eqnumcolor \eqnumsize \eqnumfont{\eqnumform{\eq@number}}}
% \end{literalcode}
% .
%    \begin{macrocode}
%d\eqnumfont{\upshape}% % Upright even when surrounding text is slanted
\def\eqnumfont{}%         % Null for easier debugging [mjd,1997/09/26]
\def\eqnumform#1{(#1\@@italiccorr)} % Add parens
\def\eqnumsize{}          % Allow numbers to have different typesize ...
%    \end{macrocode}
% Tricky questions on \cs{eqnumsize}.    Should the default
% be \cs{normalsize}?    Then the user can scale down the
% equation body with \cs{small} and not affect the equation
% number.    Or should the default be empty?    Then in large
% sections of smaller text, like the dangerous bend stuff in
% \emph{\TeX book}, the equation number size will keep in synch
% with the context.
% Maybe need an \cs{eqbodysize} param as well to allow separating
% the two cases.
%    \begin{macrocode}
\def\eqnumcolor{}         % ... or color than eq body e.g. \color{blue}
\newlength\eqlinespacing \eqlinespacing=14pt plus2pt % Base-to-base space between lines
\newlength\eqlineskip \eqlineskip=3pt plus2pt % Min space if eqlinespacing too small
\newdimen\eqlineskiplimit \eqlineskiplimit=2pt  % Threshold for switching to eqlineskip
%    \end{macrocode}
% The value of \cs{eqbinoffset} should include a negative shrink
% component that cancels the shrink component of medmuskip, otherwise
% there can be a noticeable variation in the indent of adjacent lines if
% one is shrunken a lot and the other isn't.
%    \begin{macrocode}
\newmuskip \eqbinoffset \eqbinoffset=15mu minus-3mu % Offset from mathrel alignment pt for mathbins
\newmuskip\eqdelimoffset \eqdelimoffset=2mu    % Additional offset for break inside delims
\newdimen\eqindentstep \eqindentstep=8pt     % Indent used when LHS wd is n/a or too large
\newtoks\eqstyle           % Customization hook
\newcount\eqbreakdepth \eqbreakdepth=2       % Allow breaks within delimiters to this depth
\newcount \eqinterlinepenalty \eqinterlinepenalty=10000 % No page breaks between equation lines
\newcount \intereqpenalty \intereqpenalty=1000   % Pagebreak penalty between equations [BRM: Was \@M]
\newlength \intereqskip \intereqskip=3pt plus2pt % Additional vert space between equations
\newcount\prerelpenalty \prerelpenalty=-\@M   % Linebreak penalty before mathrel symbols
\newcount\prebinoppenalty \prebinoppenalty=888  % Linebreak penalty before mathbins
%    \end{macrocode}
% When breaking equations we never right-justify, so a stretch
% component of the muskip is never helpful and sometimes it is definitely
% undesirable.    Note that thick\slash medmuskips frozen inside a
% fraction or radical may turn out noticeably larger than neighboring
% unfrozen ones.    Nonetheless I think this way is the best
% compromise short of a new \tex  that can make those built-up objects
% shrink horizontally in proportion; the alternative is to pretty much
% eliminate the shrink possibility completely in displays.
%    \begin{macrocode}
\newmuskip \Dmedmuskip \Dmedmuskip=4mu minus 3mu % medmuskip in displays
\newmuskip \Dthickmuskip \Dthickmuskip=5mu minus 2mu % thickmuskip in displays
%    \end{macrocode}
% \end{macro}
% \end{macro}
% \end{macro}
% \end{macro}
% \end{macro}
% \end{macro}
% \end{macro}
% \end{macro}
% \end{macro}
% \end{macro}
% \end{macro}
% \end{macro}
% \end{macro}
% \end{macro}
% \end{macro}
% \end{macro}
% \end{macro}
% \end{macro}
% \end{macro}
%
% And now some internal variables.    1997/10/22: some of
% these are dead branches that need to be pruned.
%
% MH: Started cleaning up a bit. No more funny loops.
%    \begin{macrocode}
\def\eq@number{}          % Internal variable
\newlength\eqleftskip \eqleftskip=\@centering  % Space on the left  [NOT USED?]
\newlength\eqrightskip \eqrightskip=\@centering % Space on the right [NOT USED?]
\newlength\eq@vspan \eq@vspan=\z@skip     % Glue used to vcenter the eq number
\newmuskip\eq@binoffset \eq@binoffset=\eqbinoffset % Roughly, \eqbinoffset + \eqdelimoffset
\newsavebox\EQ@box               % Storage for equation body
\newsavebox\EQ@copy              % For eq body sans vadjust/insert/mark material
\newsavebox\EQ@numbox            % For equation number
\newdimen\eq@wdNum         % width of number + separation [NEW]
\newsavebox\GRP@numbox            % For group number [NEW]
\newdimen\grp@wdNum         % width of number + separation [NEW]
%%B\EQ@vimbox            % Vadjust, insert, or mark material
%%B\EQ@vimcopy           % Spare copy of same
%%B\eq@impinging         % Temporary box for measuring number placement
\newcount \eq@lines          % Internal counter, actual number of lines
\newcount \eq@curline       % Loop counter
\newcount \eq@badness      % Used in testing for overfull lines
\newcount \EQ@vims          % For bookkeeping
\def\@eq@numbertrue{\let\eq@hasNumber\@True}%
\def\@eq@numberfalse{\let\eq@hasNumber\@False}%
\let\eq@hasNumber\@False
%    \end{macrocode}
% Here for the dimens, it would be advisable to do some more careful
% management to conserve dimen registers.    First of all, most of the
% dimen registers are needed in the measuring phase, which is a tightly
% contained step that happens after the contents of the equation have been
% typeset into a box and before any external functions have a chance to
% regain control\mdash  \eg , the output routine.
% Therefore it is possible to make use of the the dimen registers 0--9,
% reserved by convention for scratch use, without fear of conflict with
% other macros.    But I don't want to use them directly with the
% available names:
% \begin{literalcode}
% \dimen@ \dimen@i \dimen@ii \dimen3 \dimen4 ... \dimen9
% \end{literalcode}
% .    It would be much more useful to have names for these registers
% indicative of way they are used.
%
% Another source whence dimen registers could be borrowed is the
% \pkg{amsmath} package, which allocates six registers for
% equation-measuring purposes.    We can reuse them under different
% names since the \pkg{amsmath} functions and our functions will
% never be used simultaneously.
% \begin{literalcode}
% \eqnshift@ \alignsep@ \tagshift@ \tagwidth@ \totwidth@ \lineht@
% \end{literalcode}
%    \begin{macrocode}
\newdimen\eq@dp         % Depth of last line
\newdimen\eq@wdL        % Width of the left-hand-side
\newdimen\eq@wdT        % Total width for framing
\newdimen\eq@wdMin      % Width of narrowest line in equation
\newdimen\grp@wdL       % Max width of LHS's in a group
\newdimen\grp@wdR       % Max RHS of all equations in a group
\newdimen\grp@wdT
\newdimen\eq@wdRmax
\newdimen\eq@firstht    % Height of first line
%    \end{macrocode}
% BRM: measure the condition too.
%    \begin{macrocode}
\newdimen\eq@wdCond
\newdimen\eq@indentstep % Indent amount when LHS is not present
\newdimen\eq@linewidth  % Width actually used for display
\newdimen\grp@linewidth % Max eq@linewidth over a group
%    \end{macrocode}
% Maybe \cs{eq@hshift} could share the same register as
% \cs{mathindent} [mjd,1997/10/22].
%    \begin{macrocode}
\newdimen\eq@hshift
\let\eq@isIntertext\@False
%    \end{macrocode}
% Init \cs{eq@indentstep} to a nonzero value so that we can
% detect and refrain from clobbering a user setting of zero.
% And \cs{eq@sidespace} to \cs{maxdimen} because
% that is the right init before computing a min.
%    \begin{macrocode}
\eq@indentstep=\maxdimen
\newdimen\eq@given@sidespace
%    \end{macrocode}
%
% \begin{macro}{\eq@overrun}
%   MH: Appears to be unused.
%
%   Not a dimen register; don't need to advance it.
%    \begin{macrocode}
\def\eq@overrun{0pt}
%    \end{macrocode}
% \end{macro}
%
%
% To initialize \cs{eqnumside} and \cs{eqindent} properly,
% we may need to grub around a bit in \cs{@filelist}.    However,
% if the \pkg{amsmath} package was used, we can use its option
% data.    More trouble: if a documentclass sends an option of
% \opt{leqno} to \pkg{amsmath} by default, and it gets
% overridden by the user with a \opt{reqno} documentclass option,
% then \pkg{amsmath} believes itself to have received
% \emph{both} options.
%    \begin{macrocode}
\@ifpackagewith{amsmath}{leqno}{%
  \@ifpackagewith{amsmath}{reqno}{}{\def\eqnumside{L}}%
}{%
%    \end{macrocode}
% If the \pkg{amsmath} package was not used, the next
% method for testing the \opt{leqno} option is to see if
% \fn{leqno.clo} is present in \cs{@filelist}.
%    \begin{macrocode}
  \def\@tempa#1,leqno.clo,#2#3\@nil{%
    \ifx @#2\relax\else \def\eqnumside{L}\fi
  }%
  \@xp\@tempa\@filelist,leqno.clo,@\@nil
%    \end{macrocode}
% Even that test may fail in the case of \cls{amsart} if it does
% not load \pkg{amsmath}.    Then we have to look whether
% \cs{iftagsleft@} is defined, and if so whether it is true.
% This is tricky if you want to be careful about conditional nesting
% and don't want to put anything in the hash table unnecessarily.
%    \begin{macrocode}
  \if L\eqnumside
  \else
    \@ifundefined{iftagsleft@}{}{%
      \edef\eqnumside{%
        \if TT\csname fi\endcsname\csname iftagsleft@\endcsname
          L\else R\fi
      }%
    }
  \fi
}
%    \end{macrocode}
% A similar sequence of tests handles the \quoted{fleqn or not fleqn}
% question for the \cls{article} and \cls{amsart}
% documentclasses.
%    \begin{macrocode}
\@ifpackagewith{amsmath}{fleqn}{%
  \def\eqindent{I}%
}{%
  \def\@tempa#1,fleqn.clo,#2#3\@nil{%
    \ifx @#2\relax\else \def\eqindent{I}\fi
  }%
  \@xp\@tempa\@filelist,fleqn.clo,@\@nil
  \if I\eqindent
  \else
    \@ifundefined{if@fleqn}{}{%
      \edef\eqindent{%
        \if TT\csname fi\endcsname\csname if@fleqn\endcsname
          I\else C\fi
      }%
    }%
  \fi
}
%    \end{macrocode}
% BRM: This conditional implies we must use ALL indented or ALL centered?
%    \begin{macrocode}
%\if I\eqindent
  \@ifundefined{mathindent}{%
    \newdimen\mathindent
  }{%
    \@ifundefined{@mathmargin}{}{%
      \mathindent\@mathmargin
    }%
  }
%\fi
%    \end{macrocode}
%
%
%
%
% \section{Measuring equation components}
% Measure the left-hand side of an equation.    This
% function is called by mathrel symbols.    For the first mathrel we
% want to discourage a line break more than for following mathrels; so
% \cs{mark@lhs} gobbles the following \cs{rel@break} and
% substitutes a higher penalty.
% \begin{aside}
% Maybe the LHS should be kept in a separate box.
% \end{aside}
%
%
%
% \begin{macro}{\EQ@hasLHS}
%
% Boolean: does this equation have a \dquoted{left-hand side}?
%    \begin{macrocode}
\let\EQ@hasLHS=\@False
%    \end{macrocode}
% \end{macro}
%
%
% \begin{macro}{\EQ@QED}
%
% If nonempty: the qed material that should be incorporated into this
% equation after the final punctuation.
%    \begin{macrocode}
\let\EQ@QED=\@empty
%    \end{macrocode}
% \end{macro}
%
%
% \begin{macro}{\mark@lhs}
%
%    \begin{macrocode}
\def\mark@lhs#1{%
  \ifnum\lr@level<\@ne
    \let\mark@lhs\relax
    \global\let\EQ@hasLHS=\@True
    \global\let\EQ@prebin@space\EQ@prebin@space@a
    \mark@lhs@a
%    \end{macrocode}
% But the penalty for the first mathrel should still be lower than a
% binoppenalty.    If not, when the LHS contains a binop, the split
% will occur inside the LHS rather than at the mathrel.
% On the other hand if we end up with a multline sort of equation
% layout where the RHS is very short, the break before the relation symbol
% should be made \emph{less} desirable than the breakpoints inside
% the LHS.
% Since a lower penalty takes precedence over a higher one, we start
% by putting in the highest relpenalty; during subsequent measuring if we
% find that that RHS is not excessively short then we put in an extra
% \dquoted{normal} relpenalty when rejoining the LHS and RHS.
%    \begin{macrocode}
    \penalty9999 % instead of normal \rel@break
  % else no penalty = forbid break
  \fi
}
%    \end{macrocode}
% \end{macro}
%
%
% \begin{macro}{\mark@lhs@a}
%
% Temporarily add an extra thickmuskip to the LHS; it will be removed
% later.    This is necessary to compensate for the disappearance of
% the thickmuskip glue preceding a mathrel if a line break is taken at
% that point.    Otherwise we would have to make our definition of
% mathrel symbols more complicated, like the one for mathbins.    The
% penalty of $2$ put in with vadjust is a flag for
% \cs{eq@repack} to suggest that the box containing this line should
% be measured to find the value of \cs{eq@wdL}.    The
% second vadjust ensures that the normal prerelpenalty and thickmuskip
% will not get lost at the line break during this preliminary pass.
%
% BRM: I originally thought the \verb"\mskip\thickmuskip" was messing
% up summation limits in LHS.  But I may have fixed that problem by
% fixing other things\ldots
%    \begin{macrocode}
\def\mark@lhs@a{%
  \mskip\thickmuskip \@@vadjust{\penalty\tw@}\penalty-\@Mi\@@vadjust{}%
}
%    \end{macrocode}
% \end{macro}
%
%
% \begin{macro}{\hiderel}
% If you want the LHS to extend past the first mathrel symbol to a
% following one, mark the first one with \cs{hiderel}:
% \begin{literalcode}
% a \hiderel{=} b = c...
% \end{literalcode}
% .
% \begin{aside}
% I'm not sure now why I didn't use \cs{begingroup}
% \cs{endgroup} here \begin{dn}
% mjd,1999/01/21
% \end{dn}
% .
% \end{aside}
%
%    \begin{macrocode}
\newcommand\hiderel[1]{\mathrel{\advance\lr@level\@ne#1}}
%    \end{macrocode}
% \end{macro}
%
%
% \begin{macro}{\m@@Bin}
% \begin{macro}{\m@@Rel}
% \begin{macro}{\bin@break}
% \begin{macro}{\rel@break}
% \begin{macro}{\bin@mark}
% \begin{macro}{\rel@mark}
% \begin{macro}{\d@@Bin}
% \begin{macro}{\d@@Rel}
%
% \cf  \pkg{flexisym} handling of mathbins and mathrels.    These
% are alternate definitions of \cs{m@Bin} and \cs{m@Rel},
% activated by \cs{display@setup}.
%    \begin{macrocode}
%%%%\let\m@@Bin\m@Bin
%%%%%\let\m@@Rel\m@Rel
\let\EQ@prebin@space\relax
\def\EQ@prebin@space@a{\mskip-\eq@binoffset \keep@glue \mskip\eq@binoffset}
\def\bin@break{\ifnum\lastpenalty=\z@\penalty\prebinoppenalty\fi
  \EQ@prebin@space}
\def\rel@break{%
  \ifnum\abs@num\lastpenalty <\abs@num\prerelpenalty
    \penalty\prerelpenalty
  \fi
}
\ExplSyntaxOn
%%%\def\d@@Bin{\bin@break \m@@Bin}
%%%%\def\d@@Rel{\mark@lhs \rel@break \m@@Rel}
\cs_set:Npn \math_dsym_Bin:Nn {\bin@break\math_bsym_Bin:Nn}
\cs_set:Npn \math_dsym_Rel:Nn {\mark@lhs \rel@break \math_bsym_Rel:Nn }
\ExplSyntaxOff
%    \end{macrocode}
% The difficulty of dealing properly with the subscripts and
% superscripts sometimes appended to mathbins and mathrels is one of the
% reasons that we do not attempt to handle the mathrels as a separate
% \quoted{column} a la \env{eqnarray}.
%
% \end{macro}
% \end{macro}
% \end{macro}
% \end{macro}
% \end{macro}
% \end{macro}
% \end{macro}
% \end{macro}
%
%
% \begin{macro}{\m@@symRel}
% \begin{macro}{\d@@symRel}
% \begin{macro}{\m@@symBin}
% \begin{macro}{\d@@symBin}
% \begin{macro}{\m@@symDel}
% \begin{macro}{\d@@symDel}
% \begin{macro}{\m@@symDeR}
% \begin{macro}{\d@@symDeR}
% \begin{macro}{\m@@symDeB}
% \begin{macro}{\d@@symDeB}
% \begin{macro}{\m@@symDeA}
% \begin{macro}{\d@@symDeA}
%
% More of the same.
%    \begin{macrocode}
\ExplSyntaxOn
%%\let\m@@symRel\@symRel
%%%\def\d@@symRel{\mark@lhs \rel@break \m@@symRel}

\cs_set_protected:Npn \math_dcsym_Bin:Nn {\bin@break \math_bcsym_Bin:Nn}
\cs_set_protected:Npn \math_dcsym_Rel:Nn { \mark@lhs \rel@break \math_bcsym_Rel:Nn}


%%\let\m@@symBin\@symBin \def\d@@symBin{\bin@break \m@@symBin}
%%\let\m@@symDel\@symDel
%%\let\m@@symDeR\@symDeR
%%\let\m@@symDeB\@symDeB
%%\let\m@@symDeA\@symDeA

%    \end{macrocode}
% \end{macro}
% \end{macro}
% \end{macro}
% \end{macro}
% \end{macro}
% \end{macro}
% \end{macro}
% \end{macro}
% \end{macro}
% \end{macro}
% \end{macro}
% \end{macro}
%
%
% \begin{macro}{\display@setup}
% \begin{macro}{\everydisplay}
% Setup.    Note that \latex  reserves the primitive
% \cs{everydisplay} under the name \cs{frozen@everydisplay}.
% BRM: Disable this! It also affects non-breqn math!!!!
%    \begin{macrocode}
%\global\everydisplay\expandafter{\the\everydisplay \display@setup}
%    \end{macrocode}
% Change some math symbol function calls.
%    \begin{macrocode}
\def\display@setup{%
  \medmuskip\Dmedmuskip \thickmuskip\Dthickmuskip
   \math_setup_display_symbols:
  %%\let\m@Bin\d@@Bin \let\m@Rel\d@@Rel
  %%\let\@symRel\d@@symRel \let\@symBin\d@@symBin
  %%\let\m@DeL\d@@DeL \let\m@DeR\d@@DeR \let\m@DeB\d@@DeB
  %%\let\m@DeA\d@@DeA
  %%\let\@symDeL\d@@symDeL \let\@symDeR\d@@symDeR
  %%\let\@symDeB\d@@symDeB \let\@symDeA\d@@symDeA
  \let\left\eq@left \let\right\eq@right \global\lr@level\z@
  \global\eq@wdCond\z@          %BRM: new
%    \end{macrocode}
% If we have an embedded array environment (for example), we
% don't want to have each math cell within the array resetting
% \cs{lr@level} globally to 0\mdash not good!
% And in general I think it is safe to say that whenever we have a
% subordinate level of boxing we want to revert to a normal math setup.
%    \begin{macrocode}
  \everyhbox{\everyhbox\@emptytoks
    \let\display@setup\relax \textmath@setup \let\textmath@setup\relax
  }%
  \everyvbox{\everyvbox\@emptytoks
    \let\display@setup\relax \textmath@setup \let\textmath@setup\relax
  }%
}
%    \end{macrocode}
% The \cs{textmath@setup} function is needed for embedded inline
% math inside text inside a display.
%
% BRM: DS Experiment: Variant of \cs{display@setup} for use within
% dseries environmnents
%    \begin{macrocode}
\def\dseries@display@setup{%
  \medmuskip\Dmedmuskip \thickmuskip\Dthickmuskip
  \math_setup_display_symbols:
%%%%  \let\m@Bin\d@@Bin
%%%\let\m@Rel\d@@Rel
%%%  \let\@symRel\d@@symRel
%%% \let\@symBin\d@@symBin
%%%  \let\m@DeL\d@@DeL \let\m@DeR\d@@DeR \let\m@DeB\d@@DeB
%%%  \let\m@DeA\d@@DeA
%%%  \let\@symDeL\d@@symDeL \let\@symDeR\d@@symDeR
%%%  \let\@symDeB\d@@symDeB \let\@symDeA\d@@symDeA
  \let\left\eq@left \let\right\eq@right \global\lr@level\z@
  \everyhbox{\everyhbox\@emptytoks
    \let\display@setup\relax \textmath@setup \let\textmath@setup\relax
  }%
  \everyvbox{\everyvbox\@emptytoks
    \let\display@setup\relax \textmath@setup \let\textmath@setup\relax
  }%
 \displaystyle
}
%    \end{macrocode}
%
%    \begin{macrocode}
\def\textmath@setup{%
   \math_setup_inline_symbols:
%%%%  \let\m@Bin\m@@Bin \let\m@Rel\m@@Rel
%%%%  \let\@symRel\m@@symRel \let\@symBin\m@@symBin
%%%%  \let\m@DeL\m@@DeL \let\m@DeR\m@@DeR \let\m@DeB\m@@DeB
%%%%  \let\m@DeA\m@@DeA
%%%%  \let\@symDeL\m@@symDeL \let\@symDeR\m@@symDeR
%%%%  \let\@symDeB\m@@symDeB \let\@symDeA\m@@symDeA
  \let\left\@@left \let\right\@@right
}

\ExplSyntaxOff
%    \end{macrocode}
% \end{macro}
% \end{macro}
%
% \begin{macro}{\if@display}
% \begin{macro}{\everydisplay}
% The test \cs{ifinner} is unreliable for distinguishing
% whether we are in a displayed formula or an inline formula: any display
% more complex than a simple one-line equation typically involves the use
% of \verb"$" \cs{displaystyle} \dots  \verb"$" instead of
% \dbldollars  \dots  \dbldollars .    So we provide a more reliable
% test.    But it might have been provided already by the
% \pkg{amsmath} package.
%    \begin{macrocode}
\@ifundefined{@displaytrue}{%
  \@xp\newif\csname if@display\endcsname
  \everydisplay\@xp{\the\everydisplay \@displaytrue}%
}{}
%    \end{macrocode}
%
% \begin{aside}
% Is there any reason to maintain separate
% \cs{everydisplay} and \cn{eqstyle}?
%
% \end{aside}
%
%
% \end{macro}
% \end{macro}
%
%
%
%
% \section{The \env{dmath} and \env{dmath*} environments}
%
% Options for the \env{dmath} and \env{dmath*}
% environments.
% \begin{literalcode}
% \begin{dmath}[label={eq:xyz}]
% \begin{dmath}[labelprefix={eq:},label={xyz}]
% \end{literalcode}
% WSPR: added the option for a label prefix, designed to be used in the preamble like so:
% \begin{literalcode}
% \breqnsetup{labelprefix={eq:}}
% \end{literalcode}
%    \begin{macrocode}
\define@key{breqn}{label}{%
  \edef\next@label{\noexpand\label{\next@label@pre#1}}%
  \let\next@label@pre\@empty}
\define@key{breqn}{labelprefix}{\def\next@label@pre{#1}}
\global\let\next@label\@empty
\global\let\next@label@pre\@empty
%    \end{macrocode}
% Allow a variant number.
% \begin{literalcode}
% \begin{dmath}[number={\nref{foo}\textprime}]
% \end{literalcode}
%    \begin{macrocode}
\define@key{breqn}{number}{\def\eq@number{#1}%
  \let\@currentlabel\eq@number
}
%    \end{macrocode}
% \begin{literalcode}
% \begin{dmath}[shiftnumber]
% \begin{dmath}[holdnumber]
% \end{literalcode}
% Holding or shifting the number.
%    \begin{macrocode}
\define@key{breqn}{shiftnumber}{\let\eq@shiftnumber\@True}
\define@key{breqn}{holdnumber}{\let\eq@holdnumber\@True}
%    \end{macrocode}
% \begin{literalcode}
% \begin{dmath}[density={.5}]
% \end{literalcode}
%    \begin{macrocode}
\define@key{breqn}{density}{\def\eq@density@factor{#1}}
%    \end{macrocode}
% \begin{literalcode}
% \begin{dmath}[indentstep={1em}]
% \end{literalcode}
% To change the amount of indent for post-initial lines.    Note:
% for lines that begin with a mathbin symbol there is a fixed amount of
% indent already built in (\cs{eqbinoffset}) and it cannot be
% reduced through this option.    The indentstep amount is the indent
% used for lines that begin with a mathrel symbol.
%    \begin{macrocode}
\define@key{breqn}{indentstep}{\eqindentstep#1\relax}
%    \end{macrocode}
% \begin{literalcode}
% \begin{dmath}[compact]
% \begin{dmath}[compact=-2000]
% \end{literalcode}
% To make mathrels stay inline to the extent possible, use the compact
% option.
% Can give a numeric value in the range $-10000 \dots  10000$
% to adjust the behavior.
% $-10000$: always break at a rel symbol; $10000$: never
% break at a rel symbol.
%    \begin{macrocode}
\define@key{breqn}{compact}[-99]{\prerelpenalty=#1\relax}
%    \end{macrocode}
% \begin{literalcode}
% \begin{dmath}[layout={S}]%
% \end{literalcode}
% Specify a particular layout.
% We take care to ensure that \cs{eq@layout} ends up containing
% one and only one letter.
%    \begin{macrocode}
\define@key{breqn}{layout}[?]{%
  \edef\eq@layout{\@car#1?\@nil}%
}
%    \end{macrocode}
% \begin{literalcode}
% \begin{dmath}[spread={1pt}]
% \end{literalcode}
% To change the interline spacing in a particular equation.
%    \begin{macrocode}
\define@key{breqn}{spread}{%
  \addtolength\eqlinespacing{#1}%
  \addtolength\eqlineskip{#1}%
  \eqlineskiplimit\eqlineskip
}
%    \end{macrocode}
% To change the amount of space on the side for \dquoted{multline} layout.
%    \begin{macrocode}
\define@key{breqn}{sidespace}{%
  \setlength\eq@given@sidespace{#1}%
}
%    \end{macrocode}
% \begin{literalcode}
% \begin{dmath}[style={\small}]
% \end{literalcode}
% The \opt{style} option is mainly intended for changing the
% type size of an equation but as a matter of fact you could put arbitrary
% \latex  code here \mdash  thus the option name is \quoted{style} rather
% than just \quoted{typesize}.    In order for this option to work when
% setting options globally, we need to put the code in
% \cs{eqstyle} rather than execute it directly.
%    \begin{macrocode}
\define@key{breqn}{style}{\eqstyle\@xp{\the\eqstyle #1}}
%    \end{macrocode}
% \begin{literalcode}
% \begin{dmath}[shortskiplimit={1em}]
% \end{literalcode}
% If the line immediately preceeding a display has length $l$, the
% first line of the display is indented $i$, and a shortskip limit $s$
% is set, then the spacing above the display is equal to
% \cs{abovedisplayshortskip} if $l+s < i $ and \cs{abovedisplayskip}
% otherwise. The default shortskip limit is 2\,em which is what \TeX\
% hardcodes but this parameter overrides that.
%    \begin{macrocode}
\define@key{breqn}{shortskiplimit}{\def\eq@shortskiplimit{#1}}
\def\eq@shortskiplimit{2em}
%    \end{macrocode}
%
% \begin{literalcode}
% \begin{dmath}[frame]
% \end{literalcode}
% The \opt{frame} option merely puts a framebox around the body
% of the equation.    To change the thickness of the frame, give the
% thickness as the argument of the option.    For greater control, you
% can change the appearance of the frame by redefining
% \cs{eqframe}.    It must be a command taking two arguments, the
% width and height of the equation body.    The top left corner of the
% box produced by \cs{eqframe} will be pinned to the top-left corner
% of the equation body.
%    \begin{macrocode}
\define@key{breqn}{frame}[\fboxrule]{\def\eq@frame{T}%
  \dim@a#1\relax\edef\eq@framewd{\the\dim@a}%
%    \end{macrocode}
% Until such time as we provide a frame implementation that allows the
% frame to stretch and shrink, we'd better remove any stretch/shrink from
% the interline glue in this case.
%    \begin{macrocode}
  \freeze@glue\eqlinespacing \freeze@glue\eqlineskip
}
\define@key{breqn}{fullframe}[]{\def\eq@frame{U}%
  \freeze@glue\eqlinespacing \freeze@glue\eqlineskip
}
\def\eq@frame{F} % no frame
\def\eq@framewd{\fboxrule}
%    \end{macrocode}
% Wishful thinking?
% \begin{literalcode}
% \begin{dmath}[frame={width={2pt},color={blue},sep={2pt}}]
% \end{literalcode}
% To change the space between the frame and the equation there is a
% framesep option.
%    \begin{macrocode}
\define@key{breqn}{framesep}[\fboxsep]{%
  \if\eq@frame F\def\eq@frame{T}\fi
  \dim@a#1\relax \edef\eq@framesep{\the\dim@a}%
  \freeze@glue\eqlinespacing \freeze@glue\eqlineskip
}
\def\eq@framesep{\fboxsep}
%    \end{macrocode}
% \begin{literalcode}
% \begin{dmath}[background={red}]
% \end{literalcode}
% Foreground and background colors for the equation.    By default
% the background area that is colored is the size of the equation, plus
% fboxsep.    If you need anything fancier for the background, you'd
% better do it by defining \cs{eqframe} in terms of
% \cs{colorbox} or \cs{fcolorbox}.
%    \begin{macrocode}
\define@key{breqn}{background}{\def\eq@background{#1}%
  \freeze@glue\eqlinespacing \freeze@glue\eqlineskip
}
%   \end{macrocode}
% \begin{literalcode}
% \begin{dmath}[color={purple}]
% \end{literalcode}
%    \begin{macrocode}
\define@key{breqn}{color}{\def\eq@foreground{#1}}
%    \end{macrocode}
% \begin{literalcode}
% \begin{dmath}[center]
% \begin{dmath}[nocenter]
% \end{literalcode}
% The \opt{center} option means add leftskip stretch to make the
% individual lines be centered; this is the default for
% \env{dseries}.
%    \begin{macrocode}
\define@key{breqn}{center}[]{\let\eq@centerlines\@True}
\define@key{breqn}{nocenter}[]{\let\eq@centerlines\@False}
\let\eq@centerlines\@False
%    \end{macrocode}
% \begin{literalcode}
% \begin{dgroup}[noalign]
% \end{literalcode}
% Equation groups normally have alignment of the primary relation
% symbols across the whole group.    The \opt{noalign} option
% switches that behavior.
%    \begin{macrocode}
\define@key{breqn}{noalign}[]{\let\grp@aligned\@False}
\let\grp@aligned\@True % default
%    \end{macrocode}
% \begin{literalcode}
% \begin{dgroup}[breakdepth={2}]
% \end{literalcode}
% Break depth of 2 means that breaks are allowed at mathbin symbols
% inside two pairs of  delimiters, but not three.
%    \begin{macrocode}
\define@key{breqn}{breakdepth}{\eqbreakdepth#1\relax}
%    \end{macrocode}
% \begin{literalcode}
% \begin{darray}[cols={lcrlcr}]
% \end{literalcode}
% The \opt{cols} option only makes sense for the
% \env{darray} environment but we liberally allow all the options to
% be used with all the environments and just ignore any unsensible ones
% that happen to come along.
%    \begin{macrocode}
\define@key{breqn}{cols}{\global\let\@preamble\@empty
  \darray@mkpream#1\@percentchar
}
%    \end{macrocode}
%
% FORMAT STATUS%
% \begin{verbatim}
% \def\eq@frame{T}%
% CLM works tolerably
%  \def\eqindent{C}\def\eqnumside{L}\def\eqnumplace{M}
% CLT works tolerably
%  \def\eqindent{C}\def\eqnumside{L}\def\eqnumplace{T}
% ILM
%  \def\eqindent{I}\def\eqnumside{L}\def\eqnumplace{M}\mathindent40\p@
% ILT
%  \def\eqindent{I}\def\eqnumside{L}\def\eqnumplace{T}\mathindent40\p@
% Indended w/left number
%    work ok if mathindent is larger than number width,
%    but then equations must fit into smaller space.
%    Is shiftnumber allowed to put eqn at left, instead of indent?
% CRM
%  \def\eqindent{C}\def\eqnumside{R}\def\eqnumplace{M}
% CRB
%  \def\eqindent{C}\def\eqnumside{R}\def\eqnumplace{B}
% IRM
%  \def\eqindent{I}\def\eqnumside{R}\def\eqnumplace{M}\mathindent10\p@
% IRB
%  \def\eqindent{I}\def\eqnumside{R}\def\eqnumplace{B}\mathindent10\p@
% \end{verbatim}
%
% The main environments.
%
%BRM: The following incorporates several changes:
%  1) modifications supplied by MJD to fix the eaten \cs{paragraph} problem.
%  2) Added \cs{display@setup} here, rather than globally.
%
% \begin{macro}{\@dmath@start@hook}
% \begin{macro}{\@dgroup@start@hook}
%    \begin{macrocode}
\let\@dmath@start@hook\@empty
\let\@dgroup@start@hook\@empty
%    \end{macrocode}
% \end{macro}
% \end{macro}
%
% \begin{macro}{\dmath}
% \begin{macro}{\enddmath}
% For the \env{dmath} environment we don't want the standard
% optional arg processing because of the way it skips over whitespace,
% including newline, while looking for the \verb"[" char; which is not good
% for math material.    So we call \cs{@optarg} instead.
%    \begin{macrocode}
\newenvironment{dmath}{%
 \@dmath@start@hook
 \let\eq@hasNumber\@True \breqn@optarg\@dmath{}}{}
\def\@dmath[#1]{%
%<trace>  \breqn@debugmsg{=== DMATH ==================================================}%
  \everydisplay\expandafter{\the\everydisplay \display@setup}%
  \if@noskipsec \leavevmode \fi
  \if@inlabel \leavevmode \global\@inlabelfalse \fi
  \if\eq@group\else\eq@prelim\fi
  \setkeys{breqn}{#1}%
  \the\eqstyle
%    \end{macrocode}
% The equation number might have been overridden in \verb|#1|.
%    \begin{macrocode}
  \eq@setnumber
%    \end{macrocode}
% Start up the displayed equation by reading the contents into a
% box register.    Enclose this phase in an extra group so that
% modified \cs{hsize} and other params will be auto-restored
% afterwards.
%    \begin{macrocode}
  \begingroup
  \eq@setup@a
  \eq@startup
}
%    \end{macrocode}
% Before it finishes off the box holding the equation body,
% \cs{enddmath} needs to look ahead for punctuation (and
% \cs{qed}?).
%    \begin{macrocode}
\def\enddmath#1{%
  \check@punct@or@qed
}
\def\end@dmath{%
  \gdef\EQ@setwdL{}% Occasionally undefined ???
  \eq@capture
  \endgroup
  \EQ@setwdL
%    \end{macrocode}
% Measure (a copy of) the equation body to find the minimum width
% required to get acceptable line breaks, how many lines will be required
% at that width, and whether the equation number needs to be shifted to
% avoid overlapping.    This information will then be used by
% \cs{eq@finish} to do the typesetting of the real equation body.
%    \begin{macrocode}
  \eq@measure
%    \end{macrocode}
% Piece together the equation from its constituents, recognizing
% current constraints.    If we are in an equation group, this might
% just save the material on a stack for later processing.
%    \begin{macrocode}
  \if\eq@group \grp@push \else \eq@finish\fi
}
%    \end{macrocode}
% \end{macro}
% \end{macro}
%
% \begin{macro}{\dmath*}
% \begin{macro}{\enddmath*}
% Ah yes, now the lovely \env{dmath*} environment.
%    \begin{macrocode}
\newenvironment{dmath*}{%
  \@dmath@start@hook
  \let\eq@hasNumber\@False \breqn@optarg\@dmath{}%
}{}
\@namedef{end@dmath*}{\end@dmath}
\@namedef{enddmath*}#1{\check@punct@or@qed}
%    \end{macrocode}
% \end{macro}
% \end{macro}
%
%
% \begin{macro}{\eq@prelim}
% If \cs{everypar} has a non-null value, it's probably
% some code from \cs{@afterheading} that sets \cs{clubpenalty}
% and\slash or removes the parindent box.    Both of those actions
% are irrelevant and interfering for our purposes and need to be deflected
% for the time being.
% If an equation appears at the very beginning of a list item
% (possibly from a trivlist such as \env{proof}), we need to
% trigger the item label.
%    \begin{macrocode}
\def\eq@prelim{%
  \if@inlabel \indent \par \fi
  \if@nobreak \global\@nobreakfalse \predisplaypenalty\@M \fi
  \everypar\@emptytoks
%    \end{macrocode}
% If for some reason \env{dmath} is called between paragraphs,
% \cn{noindent} is better than \cn{leavevmode}, which would produce
% an indent box and an empty line to hold it.    If we are in a list
% environment, \cn{par} is defined as \verb"{\@@par}" to preserve
% \cs{parshape}.
%    \begin{macrocode}
  \noindent
  \eq@nulldisplay
  \par %% \eq@saveparinfo %% needs work
  \let\intertext\breqn@intertext
}
%    \end{macrocode}
% \end{macro}
% \begin{macro}{\breqn@parshape@warning}
% Warning message extracted to a separate function to streamline the
% calling function.
%    \begin{macrocode}
\def\breqn@parshape@warning{%
  \PackageWarning{breqn}{%
    Complex paragraph shape cannot be followed by this equation}%
}
%    \end{macrocode}
% \end{macro}
%
%
% \begin{macro}{\eq@prevshape}
% Storage; see \cs{eq@saveparinfo}.
%    \begin{macrocode}
\let\eq@prevshape\@empty
%    \end{macrocode}
% \end{macro}
%
%
% \begin{macro}{\eq@saveparinfo}
% Save the number of lines and parshape info for the text preceding
% the equation.
%    \begin{macrocode}
\def\eq@saveparinfo{%
  \count@\prevgraf \advance\count@-\thr@@ % for the null display
  \edef\eq@prevshape{\prevgraf\the\count@\space}%
  \ifcase\parshape
    % case 0: no action required
  \or \edef\eq@prevshape{\eq@prevshape
        \parshape\@ne\displayindent\displaywidth\relax
      }%
%    \end{macrocode}
% Maybe best to set \cs{eq@prevshape} the same in the else case
% also.    Better than nothing.
%    \begin{macrocode}
  \else
    \breqn@parshape@warning
  \fi
}
%    \end{macrocode}
% \end{macro}
%
%
% \begin{macro}{\eq@setnumber}
% If the current equation number is not explicitly given, then
% use an auto-generated number, unless the no-number switch has been
% thrown (\env{dmath*}).
% \cs{theequation} is the number form to be used for all equations,
% \cs{eq@number} is the actual value for the current equation
% (might be an exception to the usual sequence).
%    \begin{macrocode}
\def\eq@setnumber{%
  \eq@wdNum\z@
  \if\eq@hasNumber
    \ifx\eq@number\@empty
      \stepcounter{equation}\let\eq@number\theequation
    \fi
%  \fi
%    \end{macrocode}
 % This sets up numbox, etc, even if unnumbered?????
%    \begin{macrocode}
    \ifx\eq@number\@empty
    \else
%    \end{macrocode}
% Put the number in a box so we can use its measurements in our
% number-placement calculations.    The extra braces around
% \cs{eqnumform} make it possible for \cs{eqnumfont} to have
% either an \cs{itshape} (recommended) or a \cs{textit}
% value.
%    \begin{macrocode}
%<trace>      \breqn@debugmsg{Number \eq@number}%
      \set@label{equation}\eq@number
      \global\sbox\EQ@numbox{%
        \next@label \global\let\next@label\@empty
        \eqnumcolor\eqnumsize\eqnumfont{\eqnumform{\eq@number}}%
      }%
      \global\eq@wdNum\wd\EQ@numbox\global\advance\eq@wdNum\eqnumsep
%    \let\eq@hasNumber\@True % locally true
    \fi
  \fi
}
%    \end{macrocode}
% \end{macro}
%
% \begin{macro}{\eq@finish}
% The information available at this point from preliminary
% measuring includes the number of lines required, the width of the
% equation number, the total height of the equation body, and (most
% important) the parshape spec that was used in determining height and
% number of lines.
%
% Invoke the equation formatter for the requested centering/indentation
% having worked out the best parshape.
% BRM: This portion is extensively refactored to get common operations
% together (so corrections get consistently applied).
%
% MH: I've destroyed Bruce's nice refactoring a bit to get the
% abovedisplayskips correct for both groups of equations and single
% \env{dmath} environments.  I will have to redo that later.
%    \begin{macrocode}
\newcount\eq@final@linecount
\let\eq@GRP@first@dmath\@True
\def\eq@finish{%
  \begingroup
%<trace>    \breqn@debugmsg{Formatting equation}%
%<trace>    \debug@showmeasurements
    \if F\eq@frame\else
      \freeze@glue\eqlinespacing \freeze@glue\eqlineskip
    \fi
%    \eq@topspace{\vskip\parskip}% Set top spacing
    \csname eq@\eqindent @setsides\endcsname % Compute \leftskip,\rightskip
    \adjust@parshape\eq@parshape% Final adjustment of parshape for left|right skips
%    \end{macrocode}
% If we are in an a group of equations we don't want to calculate the
% top space for the first one as that will be delayed until later when
% the space for the group is calculated. However, we do need to store
% the leftskip used here as that will be used later on for calculating
% the top space.
%    \begin{macrocode}
    \if\eq@group
      \if\eq@GRP@first@dmath
        \global\let\eq@GRP@first@dmath\@False
        \xdef\dmath@first@leftskip{\leftskip=\the\leftskip\relax}%
%<trace> \breqn@debugmsg{Stored\space\dmath@first@leftskip}
      \else
        \eq@topspace{\vskip\parskip}% Set top spacing
      \fi
    \else
      \eq@topspace{\vskip\parskip}% Set top spacing
    \fi
%<trace>    \debug@showformat
%    \end{macrocode}
% We now know the final line count of the display. If it is a
% single-line display, we want to know as that greatly simplifies the
% equation tag placement (until such a time where this algorithm has
% been straightened out).
%    \begin{macrocode}
    \afterassignment\remove@to@nnil
    \eq@final@linecount=\expandafter\@gobble\eq@parshape\@nnil
%    \end{macrocode}
% Now, invoke the appropriate typesetter according to number placement
%    \begin{macrocode}
    \if\eq@hasNumber
      \if\eq@shiftnumber
        \csname eq@typeset@\eqnumside Shifted\endcsname
      \else
%    \end{macrocode}
% If there is only one line and the tag doesn't have to be shifted, we
% call a special procedure to put the tag correctly.
%    \begin{macrocode}
        \ifnum\eq@final@linecount=\@ne
          \csname eq@typeset@\eqnumside @single\endcsname
        \else
          \csname eq@typeset@\eqnumside\eqnumplace\endcsname
        \fi
      \fi
    \else
      \eq@typeset@Unnumbered
    \fi
  \endgroup
  \eq@botspace
}
%    \end{macrocode}
% \end{macro}
%
%
% These are temporary until the tag position algorithm gets
% rewritten. At least the tag is positioned correctly for single-line
% displays. The horizontal frame position is not correct but the
% problem lies elsewhere.
%    \begin{macrocode}
\def\eq@typeset@L@single{%
  \nobreak
  \eq@params\eq@parshape
  \nointerlineskip\noindent
  \add@grp@label
  \rlap{\kern-\leftskip\box\EQ@numbox}%
  \if F\eq@frame
  \else
    \rlap{\raise\eq@firstht\hbox to\z@{\eq@addframe\hss}}%
  \fi
  \eq@dump@box\unhbox\EQ@box \@@par
}
\def\eq@typeset@R@single{%
  \nobreak
  \eq@params\eq@parshape
  \nointerlineskip\noindent
  \add@grp@label
  \if F\eq@frame
  \else
    \rlap{\raise\eq@firstht\hbox to\z@{\eq@addframe\hss}}%
  \fi
  \rlap{\kern-\leftskip\kern\linewidth\kern-\wd\EQ@numbox\copy\EQ@numbox}%
  \eq@dump@box\unhbox\EQ@box
  \@@par
}
%    \end{macrocode}
%
%
%
% \section{Special processing for end-of-equation}
%
% At the end of a displayed equation environment we need to peek ahead
% for two things: following punction such as period or command that
% should be pulled in for inclusion at the end of the equation; and
% possibly also an \verb"\end{proof}" with an implied \dquoted{qed}
% symbol that is traditionally included at the end of the display rather
% than typeset on a separate line.
% We could require that the users type \cs{qed} explicitly at the
% end of the display when they want to have the display take notice of it.
% But the reason for doing that would only be to save work for the
% programmer; the most natural document markup would allow an inline
% equation and a displayed equation at the end of a proof to differ only
% in the environment name:
% \begin{literalcode}
% ... \begin{math} ... \end{math}.
% \end{proof}
% \end{literalcode}
% versus
% \begin{literalcode}
% ...
% \begin{dmath}
%  ...
% \end{dmath}.
% \end{proof}
% \end{literalcode}
% .
% The technical difficulties involved in supporting this markup within
% \latex2e  are, admittedly, nontrivial.
% Nonetheless, let's see how far we can go.
%
%
% The variations that we will support are only the most
% straightforward ones:
% \begin{literalcode}
% \end{dmath}.
% \end{proof}
% \end{literalcode}
% or
% \begin{literalcode}
% \end{dmath}.
% Perhaps a comment
% \end{proof}
% \end{literalcode}
% .
% If there is anything more complicated than a space after the
% period we will not attempt to scan any further for a possible
% \verb"\end{proof}".
% This includes material such as:
% \begin{literalcode}
% \begin{figure}...\end{figure}%
% \footnote{...}
% \renewcommand{\foo}{...}
% \par
% \end{literalcode}
% or even a blank line\mdash because in \latex  a blank line is
% equivalent to \cs{par} and the meaning of \cs{par} is
% \dquoted{end-paragraph}; in my opinion if explicit end-of-paragraph
% markup is given before the end of an element, it has to be respected,
% and the preceding paragraph has to be fully finished off before
% proceeding further, even inside an element like \dquoted{proof} whose
% end-element formatting requires integration with the end of the
% paragraph text.
% And \tex nically speaking, a \cs{par} token that comes from a
% blank line and one that comes from the sequence of characters
% \verb"\" \verb"p" \verb"a" \verb"r" are equally explicit.
% I hope to add support for \cs{footnote} in the future, as it
% seems to be a legitimate markup possibility in that context from a
% purely logical point of view, but there are additional technical
% complications if one wants to handle it in full generality
% \begin{dn}
% mjd,1999/02/08
% \end{dn}
% .
%
%
% \begin{macro}{\peek@branch}
% This is a generalized \dquoted{look at next token and choose some action
% based on it} function.
%    \begin{macrocode}
\def\peek@branch#1#2{%
  \let\peek@b#1\let\peek@space#2\futurelet\@let@token\peek@a
}
\def\peek@skipping@spaces#1{\peek@branch#1\peek@skip@space}
\def\peek@a{%
  \ifx\@let@token\@sptoken \expandafter\peek@space
  \else \expandafter\peek@b\fi
}
\lowercase{\def\peek@skip@space} {\futurelet\@let@token\peek@a}%
%    \end{macrocode}
% \end{macro}
%
%
% \begin{macro}{\check@punct}
%   \changes{v0.96a}{2007/12/17}{Insert \cs{finish@end} if no special
%     case is found.}
% For this one we need to recognize and grab for inclusion any of the
% following tokens: \verb",;.!?", both catcode 12 (standard \latex
% value) and catcode 13 (as might hold when the Babel package is
% being used).
% We do not support a space preceding the punctuation since that would
% be considered simply invalid markup if a display-math environment were
% demoted to in-line math; and we want to keep their markup as parallel as
% possible.
% If punctuation does not follow, then the \cs{check@qed} branch
% is not applicable.
%    \begin{macrocode}
\def\check@punct{\futurelet\@let@token\check@punct@a}
\def\check@punct@a{%
  \edef\@tempa{%
    \ifx\@let@token\@sptoken\@nx\finish@end
    \else\ifx\@let@token ,\@nx\check@qed
    \else\ifx\@let@token .\@nx\check@qed
    \else\check@punct@b % check the less common possibilities
    \fi\fi\fi
  }%
  \@tempa
}
\begingroup
\toks@a{%
  \ifx\@let@token ;\@nx\check@qed
  \else\ifx\@let@token ?\@nx\check@qed
  \else\ifx\@let@token !\@nx\check@qed
}
\toks@c{\fi\fi\fi}% matching with \toks@a
\catcode`\.=\active \catcode`\,=\active \catcode`\;=\active
\catcode`\?=\active \catcode`\!=\active
\toks@b{%
  \else\ifx\@let@token ,\@nx\check@qed
  \else\ifx\@let@token .\@nx\check@qed
  \else\ifx\@let@token ;\@nx\check@qed
  \else\ifx\@let@token ?\@nx\check@qed
  \else\ifx\@let@token !\@nx\check@qed
  \else\@nx\finish@end
  \fi\fi\fi\fi\fi
}
\xdef\check@punct@b{%
  \the\toks@a\the\toks@b\the\toks@c
}
\endgroup
%    \end{macrocode}
%
%    \begin{macrocode}
\let\found@punct\@empty
\def\check@qed#1{%
  \gdef\found@punct{#1}%
  \peek@skipping@spaces\check@qed@a
}
\def\check@qed@a{%
  \ifx\end\@let@token \@xp\check@qed@b
  \else \@xp\finish@end
  \fi
}
%    \end{macrocode}
% For each environment ENV that takes an implied qed at the end, the
% control sequence ENVqed must be defined; and it must include suitable
% code to yield the desired results in a displayed equation.
%    \begin{macrocode}
\def\check@qed@b#1#2{%
  \@ifundefined{#2qed}{}{%
    \toks@\@xp{\found@punct\csname#2qed\endcsname}%
    \xdef\found@punct{\the\toks@}%
  }%
  \finish@end
  \end{#2}%
}
%    \end{macrocode}
% \end{macro}
%
%
% \begin{macro}{\latex@end}
% \begin{macro}{\finish@end}
% The lookahead for punctuation following a display requires
% mucking about with the normal operation of \cn{end}.    Although
% this is not exactly something to be done lightly, on the other hand this
% whole package is so over-the-top anyway, what's a little more
% going to hurt?    And rationalizing this aspect of
% equation markup is a worthy cause.    Here is the usual
% definition of \cs{end}.
% \begin{literalcode}
% \def\end#1{
%   \csname end#1\endcsname \@checkend{#1}%
%   \expandafter\endgroup\if@endpe\@doendpe\fi
%   \if@ignore \global\@ignorefalse \ignorespaces \fi
% }
% \end{literalcode}
% We can improve the chances of this code surviving through future
% minor changes in the fundamental definition of \cs{end} by taking a
% little care in saving the original meaning.
%    \begin{macrocode}
\def\@tempa#1\endcsname#2\@nil{\def\latex@end##1{#2}}
\ifcsname end \endcsname
  % 2019: \end was made robust
  \expandafter\expandafter\expandafter\@tempa\csname end \endcsname{#1}\@nil
  \@namedef{end }#1{\csname end#1\endcsname \latex@end{#1}}%
\else
  % pre-2019: the old approach
  \expandafter\@tempa\end{#1}\@nil
  \def\end#1{\csname end#1\endcsname \latex@end{#1}}%
\fi
%    \end{macrocode}
% Why don't we call \cs{CheckCommand} here?    Because that
% doesn't help end users much; it works better to use it during package
% testing by the maintainer.
%
% Note to self: I always forget how this redefinition works. Here's a summary.
% The approach is:
% \begin{verbatim}
% \def\enddmath#1{\check@punct@or@qed}
% \end{verbatim}
% When hitting the end of the dmath environment we then have:
% \begin{verbatim}
% \csname end#1\endcsname \latex@end{#1}
% \enddmath \latex@end{dmath}
% \check@punct@or@qed {dmath}  % <- \enddmath absorbs one arg
% \xdef\found@punct{\@empty}\def\finish@end{\csname end@dmath\endcsname\latex@end{dmath}}\check@punct
% \end{verbatim}
% At which point \verb|\check@punct| is `past' the end of the environment and can look ahead.
%
%
% If a particular environment needs to call a different end action, the
% end command of the environment should be defined to gobble two args and
% then call a function like \cs{check@punct@or@qed}.
%    \begin{macrocode}
\def\check@punct@or@qed#1{%
  \xdef\found@punct{\@empty}% BRM: punctuation was being remembered past this eqn.
  % WSPR: err, why isn't that just \global\let\found@punct\@empty ?
  \def\finish@end{\csname end@#1\endcsname\latex@end{#1}}%
  \check@punct
}
%    \end{macrocode}
% \end{macro}
% \end{macro}
%
% \begin{macro}{\eqpunct}
% User-settable function for handling
% the punctuation at the end of an equation.    You could, for example,
% define it to just discard the punctuation.
%    \begin{macrocode}
\newcommand\eqpunct[1]{\thinspace#1}
%    \end{macrocode}
% \end{macro}
%
% \begin{macro}{\set@label}
% \cs{set@label} just sets \cs{@currentlabel} but it
% takes the counter as an argument, in the hope that \latex  will some
% day provide an improved labeling system that includes type info on the
% labels.
%    \begin{macrocode}
\providecommand\set@label[2]{\protected@edef\@currentlabel{#2}}
%    \end{macrocode}
% \end{macro}
%
% \begin{macro}{\eq@topspace}
% \begin{macro}{\eq@botspace}
% The action of \cs{eq@topspace} is complicated by the
% need to test whether the \quoted{short} versions of the display skips
% should be used.    This can be done only after the final parshape
% and indent have been determined, so the calls of this function are
% buried relatively deeply in the code by comparison to the calls of
% \cs{eq@botspace}.    This also allows us to optimize
% slightly by setting the above-skip with \cs{parskip} instead of
% \cs{vskip}.    \verb|#1|  is either \cs{noindent} or
% \verb"\vskip\parskip".
%
% BRM: Hmm; we need to do *@setspace BEFORE this for small skips to work!
%    \begin{macrocode}
\def\eq@topspace#1{%
  \begingroup
    \global\let\EQ@shortskips\@False
%    \end{macrocode}
% If we are in \env{dgroup} or \env{dgroup*} and not before the top
% one, we just insert \cs{intereqskip}. Otherwise we must check for
% shortskip.
%    \begin{macrocode}
    \if\@And{\eq@group}{\@Not\eq@GRP@first@dmath}%
%<trace>\breqn@debugmsg{Between lines}%
        \parskip\intereqskip \penalty\intereqpenalty
%<trace>\breqn@debugmsg{parskip=\the\parskip}%
    \else
      \eq@check@shortskip
      \if\EQ@shortskips
        \parskip\abovedisplayshortskip
        \aftergroup\belowdisplayskip\aftergroup\belowdisplayshortskip
%    \end{macrocode}
% BRM: Not exactly \TeX's approach, but seems right\ldots
%    \begin{macrocode}
        \ifdim\predisplaysize>\z@\nointerlineskip\fi
      \else
        \parskip\abovedisplayskip
      \fi
    \fi
    \if F\eq@frame
    \else
      \addtolength\parskip{\eq@framesep+\eq@framewd}%
    \fi
%<*trace>
    \breqn@debugmsg{Topspace: \theb@@le\EQ@shortskips, \parskip=\the\parskip,
      \predisplaysize=\the\predisplaysize}%
%</trace>
    #1%
  \endgroup
}
%    \end{macrocode}
% \begin{macro}{\eq@check@shortskip}
%   \changes{v0.96a}{2007/12/17}{Insert \cs{finish@end} if no special
%     case is found.}
%   \changes{v0.97a}{2007/12/22}{Use design parameter and fix
%   shortskips properly.}
%    \begin{macrocode}
\def\eq@check@shortskip {%
  \global\let\EQ@shortskips\@False
  \setlength\dim@a{\abovedisplayskip+\ht\EQ@numbox}%
%    \end{macrocode}
% Here we work around the hardwired standard TeX value and use the
% designer parameter instead.
%    \begin{macrocode}
  \ifdim\leftskip<\predisplaysize
  \else
%    \end{macrocode}
% If the display was preceeded by a blank line, \cs{predisplaysize} is
% $-\cs{maxdimen}$ and so we should insert a fairly large skip to
% separate paragraphs, i.e., no short skip. Perhaps this should be a
% third parameter \cs{abovedisplayparskip}.
%    \begin{macrocode}
    \ifdim -\maxdimen=\predisplaysize
    \else
      \if R\eqnumside
        \global\let\EQ@shortskips\@True
      \else
        \if\eq@shiftnumber
        \else
          \if T\eqnumplace
            \ifdim\dim@a<\eq@firstht
              \global\let\EQ@shortskips\@True
            \fi
          \else
            \setlength\dim@b{\eq@vspan/2}%
            \ifdim\dim@a<\dim@b
              \global\let\EQ@shortskips\@True
            \fi
          \fi
        \fi
      \fi
    \fi
  \fi
}
%    \end{macrocode}
% \end{macro}
%
% At the end of an equation, need to put in a pagebreak penalty
% and some vertical space.    Also set some flags to remove parindent
% and extra word space if the current paragraph text continues without an
% intervening \cs{par}.
%    \begin{macrocode}
\def\eq@botspace{%
  \penalty\postdisplaypenalty
%    \end{macrocode}
% Earlier calculations will have set \cs{belowdisplayskip} locally
% to \cs{belowdisplayshortskip} if applicable.    So we can just use
% it here.
%    \begin{macrocode}
  \if F\eq@frame
  \else
    \addtolength\belowdisplayskip{\eq@framesep+\eq@framewd}%
  \fi
  \vskip\belowdisplayskip
  \@endpetrue % kill parindent if current paragraph continues
  \global\@ignoretrue % ignore following spaces
  \eq@resume@parshape
}
%    \end{macrocode}
% \end{macro}
% \end{macro}
%
% \begin{macro}{\eq@resume@parshape}
% This should calculate the total height of the equation,
% including space above and below, and set prevgraf to the number it would
% be if that height were taken up by normally-spaced normal-height
% lines.    We also need to restore parshape if it had a non-null
% value before the equation.    Not implemented yet.
%    \begin{macrocode}
\def\eq@resume@parshape{}
%    \end{macrocode}
% \end{macro}
%
% \section{Preprocessing the equation body}
% \begin{macro}{\eq@startup}
% Here is the function that initially collects the equation
% material in a box.
%
%    \begin{macrocode}
\def\eq@startup{%
  \global\let\EQ@hasLHS\@False
  \setbox\z@\vbox\bgroup
    \noindent \@@math \displaystyle
    \penalty-\@Mi
}
%    \end{macrocode}
%
% This setup defines the environment for the first typesetting
% pass, note the \cs{hsize} value for example.
%    \begin{macrocode}
\def\eq@setup@a{%
  \everymath\everydisplay
  %\let\@newline\eq@newline % future possibility?
  \let\\\eq@newline
  \let\insert\eq@insert \let\mark\eq@mark \let\vadjust\eq@vadjust
  \hsize\maxdimen \pretolerance\@M
%    \end{macrocode}
% Here it is better not to use \cs{@flushglue} (0pt
% plus1fil) for \cs{rightskip}, or else a negative penalty
% (such as $-99$ for \cs{prerelpenalty}) will tempt
% \tex  to use more line breaks than necessary in the first typesetting
% pass.    Ideal values for \cs{rightskip} and
% \cs{linepenalty} are unclear to me, but they are rather sensitively
% interdependent.    Choice of 10000 pt for rightskip is derived by
% saying, let's use a value smaller than 1 fil and smaller than
% \cs{hsize}, but more than half of \cs{hsize} so that if a line
% is nearly empty, the glue stretch factor will always be less than 2.0
% and so the badness will be less than 100 and so \tex  will not issue
% badness warnings.
%    \begin{macrocode}
  \linepenalty\@m
  \rightskip\z@\@plus\@M\p@ \leftskip\z@skip \parfillskip\z@skip
  \clubpenalty\@ne \widowpenalty\z@ \interlinepenalty\z@
%    \end{macrocode}
% After a relation symbol is discovered, binop symbols should start
% including a special offset space.
% But until then \cs{EQ@prebin@space} is a no-op.
%    \begin{macrocode}
  \global\let\EQ@prebin@space\relax
%    \end{macrocode}
% Set binoppenalty and relpenalty high to prohibit line breaks
% after mathbins and mathrels.    As a matter of fact, the penalties are
% then omitted by \tex , since bare glue without a penalty is
% \emph{not} a valid breakpoint if it occurs within
% mathon\ndash mathoff items.
%    \begin{macrocode}
  \binoppenalty\@M \relpenalty\@M
}
%    \end{macrocode}
% \end{macro}
%
%
% \begin{figure}
%   \centering
% The contents of an equation after the initial typesetting pass,
% as shown by \cs{showlists}.    This is the material on which the
% \cs{eq@repack} function operates.    The equation was
% \begin{literalcode}
% a=b +\left(\frac{c\sp 2}{2} -d\right) +(e -f) +g
% \end{literalcode}
% .    The contents are shown in four parts in this figure and the next
% three.    The first part contains two line boxes, one for the mathon
% node and one for the LHS.
% \begin{literalcode}
% \hbox(0.0+0.0)x16383.99998, glue set 1.6384
% .\mathon
% .\penalty -10000
% .\glue(\rightskip) 0.0 plus 10000.0
% \penalty 1
% \glue(\baselineskip) 7.69446
% \hbox(4.30554+0.0)x16383.99998, glue set 1.63759
% .\OML/cmm/m/it/10 a
% .\glue 2.77771 minus 1.11108
% .\penalty -10001
% .\glue(\rightskip) 0.0 plus 10000.0
% \penalty 2
% \glue(\lineskip) 1.0
% ...
% \end{literalcode}
%   \caption{Preliminary equation contents, part 1}
% \end{figure}
% \begin{figure}\centering
% This is the first part of the RHS, up to the
% \cs{right}, where a line break has been forced so that we can break
% open the left-right box.
% \begin{literalcode}
% ...
% \penalty 2
% \glue(\lineskip) 1.0
% \hbox(14.9051+9.50012)x16383.99998, glue set 1.63107
% .\penalty -99
% .\glue(\thickmuskip) 2.77771 minus 1.11108
% .\OT1/cmr/m/n/10 =
% .\glue(\thickmuskip) 2.77771 minus 1.11108
% .\OML/cmm/m/it/10 b
% .\penalty 888
% .\glue -10.5553
% .\rule(*+*)x0.0
% .\penalty 10000
% .\glue 10.5553
% .\glue(\medmuskip) 2.22217 minus 1.66663
% .\OT1/cmr/m/n/10 +
% .\glue(\medmuskip) 2.22217 minus 1.66663
% .\hbox(14.9051+9.50012)x43.36298
% ..\hbox(0.39998+23.60025)x7.36115, shifted -14.10013
% ...\OMX/cmex/m/n/5 \hat \hat R
% ..\hbox(14.9051+6.85951)x11.21368
% ...\hbox(14.9051+6.85951)x11.21368
%   ... [fraction contents, elided]
% ..\penalty 5332
% ..\glue -10.5553
% ..\rule(*+*)x0.0
% ..\penalty 10000
% ..\glue 10.5553
% ..\glue(\medmuskip) 2.22217 minus 1.66663
% ..\OMS/cmsy/m/n/10 \hat \hat @
% ..\glue(\medmuskip) 2.22217 minus 1.66663
% ..\OML/cmm/m/it/10 d
% ..\hbox(0.39998+23.60025)x7.36115, shifted -14.10013
% ...\OMX/cmex/m/n/5 \hat \hat S
% .\penalty -10000
% .\glue(\rightskip) 0.0 plus 10000.0
% \penalty 3
% \glue(\lineskip) 1.0
% ...
% \end{literalcode}
% \caption{Preliminary equation contents, part 2}
% \end{figure}
%
% \begin{figure}
%   \centering
% This is the remainder of the RHS after the post-\cs{right}
% split.
% \begin{literalcode}
% ...
% \penalty 3
% \glue(\lineskip) 1.0
% \hbox(7.5+2.5)x16383.99998, glue set 1.63239
% .\penalty 888
% .\glue -10.5553
% .\rule(*+*)x0.0
% .\penalty 10000
% .\glue 10.5553
% .\glue(\medmuskip) 2.22217 minus 1.66663
% .\OT1/cmr/m/n/10 +
% .\glue(\medmuskip) 2.22217 minus 1.66663
% .\OT1/cmr/m/n/10 (
% .\OML/cmm/m/it/10 e
% .\penalty 5332
% .\glue -10.5553
% .\rule(*+*)x0.0
% .\penalty 10000
% .\glue 10.5553
% .\glue(\medmuskip) 2.22217 minus 1.66663
% .\OMS/cmsy/m/n/10 \hat \hat @
% .\glue(\medmuskip) 2.22217 minus 1.66663
% .\OML/cmm/m/it/10 f
% .\kern1.0764
% .\OT1/cmr/m/n/10 )
% .\penalty 888
% .\glue -10.5553
% .\rule(*+*)x0.0
% .\penalty 10000
% .\glue 10.5553
% .\glue(\medmuskip) 2.22217 minus 1.66663
% .\OT1/cmr/m/n/10 +
% .\glue(\medmuskip) 2.22217 minus 1.66663
% .\OML/cmm/m/it/10 g
% .\kern0.35878
% .\penalty -10000
% .\glue(\rightskip) 0.0 plus 10000.0
% \glue(\baselineskip) 9.5
% ...
% \end{literalcode}
% \caption{Preliminary equation contents, part 3}
% \end{figure}
%
% \begin{figure}
%   \centering
% This is the mathoff fragment.
% \begin{literalcode}
% ...
% \glue(\baselineskip) 9.5
% \hbox(0.0+0.0)x16383.99998, glue set 1.6384
% .\mathoff
% .\penalty 10000
% .\glue(\parfillskip) 0.0
% .\glue(\rightskip) 0.0 plus 10000.0
% \end{literalcode}
% \caption{Preliminary equation contents, part 4}
% \end{figure}
%
% \begin{macro}{\eq@capture}
% \begin{macro}{\eq@punct}
% If an equation ends with a \cs{right} delim, the last thing
% on the math list will be a force-break penalty.    Then don't
% redundantly add another forcing penalty.    (question: when does a
% penalty after a linebreak not disappear?    Answer: when you have
% two forced break penalties in a row).    Ending punctuation, if
% any, goes into the last box with the mathoff kern.    If the math list
% ends with a slanted letter, then there will be an italic correction
% added after it by \tex .    Should we remove it?    I guess
% so.
%
%
% \subsection{Capturing the equation}
%
% BRM: There's a problem here (or with \cs{ss@scan}).  If the LHS has
% \cs{left} \cs{right} pairs, \cs{ss@scan} gets involved.  It seems to produce
% a separate box marked w/\cs{penalty} 3.  But it appears that \cs{eq@repack}
% is only expecting a single box for the LHS; when it measures that
% box it's missing the (typically larger) bracketted section,
% so the LHS is measured => 0pt (or very small).
%  I'm not entirely clear what Michael had in mind for this case;
% whether it's an oversight, or whether I've introduced some other bug.
% At any rate, my solution is to measure the RHS (accumulated in \cs{EQ@box}),
% at the time of the relation, and subtract that from the total size.
%    \begin{macrocode}
\newdimen\eq@wdR\eq@wdR\z@%BRM
\def\eq@capture{%
  \ifnum\lastpenalty>-\@M \penalty-\@Mi \fi
%    \end{macrocode}
% We want to keep the mathoff kern from vanishing at the line break,
% so that we can reuse it later.
%    \begin{macrocode}
  \keep@glue\@@endmath
  \eq@addpunct
  \@@par
  \eq@wdL\z@
%    \end{macrocode}
% First snip the last box, which contains the mathoff node, and put it
% into \cs{EQ@box}.    Then when we call \cs{eq@repack} it
% will recurse properly.
%    \begin{macrocode}
  \setbox\tw@\lastbox
  \global\setbox\EQ@box\hbox{\unhbox\tw@\unskip\unskip\unpenalty}%
  \unskip\unpenalty
  \global\setbox\EQ@copy\copy\EQ@box
%%  \global\setbox\EQ@vimcopy\copy\EQ@vimbox
  \clubpenalty\z@
%\batchmode\showboxbreadth\maxdimen\showboxdepth99\showlists\errorstopmode
  \eq@wdR\z@%BRM: eq@wdL patch
  \eq@repack % recursive
%    \end{macrocode}
% Finally, add the mathon item to \cs{EQ@box} and \cs{EQ@copy}.
%    \begin{macrocode}
  \setbox\tw@\lastbox
  \global\setbox\EQ@box\hbox{\unhcopy\tw@\unskip\unpenalty \unhbox\EQ@box}%
  \global\setbox\EQ@copy\hbox{\unhbox\tw@\unskip\unpenalty \unhbox\EQ@copy}%
%\batchmode\showbox\EQ@copy \showthe\eq@wdL\errorstopmode
  \ifdim\eq@wdR>\z@% BRM:  eq@wdL patch
    \setlength\dim@a{\wd\EQ@box-\eq@wdR
    % Apparently missing a \thickmuskip = 5mu = 5/18em=0.27777777777.. ?
       + 0.2777777777777em}% FUDGE??!?!?!
    \ifdim\dim@a>\eq@wdL
%<*trace>
      \breqn@debugmsg{Correcting LHS from \the\eq@wdL\space to
                \the\dim@a = \the\wd\EQ@box - \the\eq@wdR}%
%</trace>
      \eq@wdL\dim@a
      \xdef\EQ@setwdL{\eq@wdL\the\eq@wdL\relax}%
    \fi
  \fi
%<*trace>
  \breqn@debugmsg{Capture: total length=\the\wd\EQ@box \MessageBreak
           ==== has LHS=\theb@@le\EQ@hasLHS, \eq@wdL=\the\eq@wdL, \eq@wdR=\the\eq@wdR,
           \MessageBreak
           ==== \eq@wdCond=\the\eq@wdCond}%
%</trace>
  \egroup % end vbox started earlier
%<*trace>
%\debugwr{EQ@box}\debug@box\EQ@box
%\debugwr{EQ@copy}\debug@box\EQ@copy
%</trace>
}
%    \end{macrocode}
% Now we have two copies of the equation, one in \cs{EQ@box},
% and one in \cs{EQ@copy} with inconvenient stuff like inserts and
% marks omitted.
%
% \cs{eq@addpunct} is for tacking on text punctuation at the end
% of a display, if any was captured by the \quoted{gp} lookahead.
%    \begin{macrocode}
\def\eq@addpunct{%
  \ifx\found@punct\@empty
  \else \eqpunct{\found@punct}%
  \fi
  % BRM: Added; the punctuation kept  getting carried to following environs
  \xdef\found@punct{\@empty}%
  \EQ@afterspace
}
%    \end{macrocode}
% Needed for the \env{dseries} environment, among other things.
%    \begin{macrocode}
\global\let\EQ@afterspace\@empty
%    \end{macrocode}
% \end{macro}
% \end{macro}
%
% \begin{macro}{\eq@repack}
% The \cs{eq@repack} function looks at the information at hand
% and proceeds accordingly.
%
% TeX Note: this scans BACKWARDS from the end of the math.
%    \begin{macrocode}
\def\eq@repack{%
% A previous penalty of 3 on the vertical list means that we need
% to break open a left-right box.
%    \begin{macrocode}
  \ifcase\lastpenalty
     % case 0: normal case
    \setbox\tw@\lastbox
    \eq@repacka\EQ@copy \eq@repacka\EQ@box
    \unskip
  \or % case 1: finished recursing
%    \end{macrocode}
% Grab the mathon object since we need it to inhibit line breaking at
% bare glue nodes later.
%    \begin{macrocode}
    \unpenalty
    \setbox\tw@\lastbox
    \eq@repacka\EQ@copy \eq@repacka\EQ@box
    \@xp\@gobble
  \or % case 2: save box width = LHS width
%    \end{macrocode}
% Don't need to set \cs{EQ@hasLHS} here because it was set earlier
% if applicable.
%    \begin{macrocode}
    \unpenalty
    \setbox\tw@\lastbox
    \setbox\z@\copy\tw@ \setbox\z@\hbox{\unhbox\z@\unskip\unpenalty}%
    \addtolength\eq@wdL{\wd\z@}
    \setlength\eq@wdR{\wd\EQ@box}% BRM:  eq@wdL patch
    \xdef\EQ@setwdL{\eq@wdL\the\eq@wdL\relax}%
%    \end{macrocode}
% At this point, box 2 typically ends with
% \begin{literalcode}
% .\mi10 a
% .\glue 2.77771 plus 2.77771
% .\penalty -10001
% .\glue(\rightskip) 0.0 plus 10000.0
% \end{literalcode}
% and we want to ensure that the thickmuskip glue gets removed.
% And we now arrange for \cs{EQ@copy} and \cs{EQ@box} to
% keep the LHS in a separate subbox; this is so that we can introduce a
% different penalty before the first relation symbol if necessary,
% depending on the layout decisions that are made later.
%    \begin{macrocode}
    \global\setbox\EQ@copy\hbox{%
      \hbox{\unhcopy\tw@\unskip\unpenalty\unskip}%
      \box\EQ@copy
    }%
    \global\setbox\EQ@box\hbox{%
      \hbox{\unhbox\tw@\unskip\unpenalty\unskip}%
      \box\EQ@box
    }%
    \unskip
  \or % case 3: unpack left-right box
    \unpenalty
    \eq@lrunpack
  \else
    \breqn@repack@err
  \fi
  \eq@repack % RECURSE
}
%    \end{macrocode}
% Error message extracted to streamline calling function.
%    \begin{macrocode}
\def\breqn@repack@err{%
  \PackageError{breqn}{eq@repack penalty neq 0,1,2,3}\relax
}
%    \end{macrocode}
% \end{macro}
%
%
% \begin{macro}{\eq@repacka}
% We need to transfer each line into two separate boxes, one
% containing everything and one that omits stuff like \cs{insert}s
% that would interfere with measuring.
%    \begin{macrocode}
\def\eq@repacka#1{%
  \global\setbox#1\hbox{\unhcopy\tw@ \unskip
    \count@-\lastpenalty
    \ifnum\count@<\@M \else \advance\count@-\@M \fi
    \unpenalty
%    \end{macrocode}
% If creating the measure copy, ignore all cases above case 3 by
% folding them into case 1.
%    \begin{macrocode}
    \ifx\EQ@copy#1\ifnum\count@>\thr@@ \count@\@ne\fi\fi
    \ifcase\count@
        % case 0, normal line break
      \penalty-\@M % put back the linebreak penalty
    \or % case 1, do nothing (end of equation)
      \relax
    \or % case 2, no-op (obsolete case)
    \or % case 3, transfer vspace and/or penalty
      \ifx#1\EQ@box \eq@revspace \else \eq@revspaceb \fi
    \or % case 4, put back an insert
      \eq@reinsert
    \or % case 5, put back a mark
      \eq@remark
    \or % case 6, put back a vadjust
      \eq@readjust
    \else % some other break penalty
      \penalty-\count@
    \fi
    \unhbox#1}%
}
%    \end{macrocode}
% \end{macro}
%
%
% \begin{macro}{\eq@nulldisplay}
% Throw in a null display in order to get predisplaysize \etc .
% My original approach here was to start the null display, then measure
% the equation, and set a phantom of the equation's first line before
% ending the null display.    That would allow finding out if \tex  used
% the short displayskips instead of the normal ones.    But because of
% some complications with grouping and the desirability of omitting
% unnecessary invisible material on the vertical list, it seems better to
% just collect information about the display (getting \cs{prevdepth}
% requires \cs{halign}) and manually perform our own version of
% \TeX's shortskip calculations.    This approach also gives greater
% control, \eg , the threshold amount of horizontal space between
% predisplaysize and the equation's left edge that determines when the
% short skips kick in becomes a designer-settable parameter rather than
% hardwired into \TeX .
%    \begin{macrocode}
\def\eq@nulldisplay{%
  \begingroup \frozen@everydisplay\@emptytoks
  \@@display
  \predisplaypenalty\@M \postdisplaypenalty\@M
  \abovedisplayskip\z@skip \abovedisplayshortskip\z@skip
  \belowdisplayskip\z@skip \belowdisplayshortskip\z@skip
  \xdef\EQ@displayinfo{%
    \prevgraf\the\prevgraf \predisplaysize\the\predisplaysize
    \displaywidth\the\displaywidth \displayindent\the\displayindent
    \listwidth\the\linewidth
%    \end{macrocode}
% Not sure how best to test whether leftmargin should be
% added.    Let's do this for now [mjd,1997/10/08].
%    \begin{macrocode}
    \ifdim\displayindent>\z@
      \advance\listwidth\the\leftmargin
      \advance\listwidth\the\rightmargin
    \fi
    \relax}%
%    \end{macrocode}
% An \cs{halign} containing only one \cs{cr} (for the
% preamble) puts no box on the vertical list, which means that no
% \cs{baselineskip} will be added (so we didn't need to set it to
% zero) and the previous value of prevdepth carries through.    Those
% properties do not hold for an empty simple equation without
% \cs{halign}.
%    \begin{macrocode}
  \halign{##\cr}%
  \@@enddisplay
  \par
  \endgroup
  \EQ@displayinfo
}
%    \end{macrocode}
% \end{macro}
%
%
% \begin{macro}{\eq@newline}
% \begin{macro}{\eq@newlinea}
% \begin{macro}{\eq@newlineb}
% Here we use \cs{@ifnext} so that in a sequence like
% \begin{literalcode}
% ...\\
% [a,b]
% \end{literalcode}
% \latex  does not attempt to interpret the \verb"[a,b]" as a
% vertical space amount.    We would have used \cs{eq@break} in the
% definition of \cs{eq@newlineb} except that it puts in a
% \cs{keep@glue} object which is not such a good idea if a mathbin
% symbol follows \mdash  the indent of the mathbin will be wrong because
% the leading negative glue will not disappear as it should at the line
% break.
%    \begin{macrocode}
\def\eq@newline{%
  \breqn@ifstar{\eq@newlinea\@M}{\eq@newlinea\eqinterlinepenalty}}
\def\eq@newlinea#1{%
  \@ifnext[{\eq@newlineb{#1}}{\eq@newlineb{#1}[\maxdimen]}}
\def\eq@newlineb#1[#2]{\penalty-\@M}
%    \end{macrocode}
% \end{macro}
% \end{macro}
% \end{macro}
%
%
% \begin{macro}{\eq@revspace}
% \begin{macro}{\eq@revspaceb}
% When \cs{eq@revspace} (re-vspace) is called, we are the
% end of an equation line; we need to remove the existing penalty of
% $-10002$ in order to put a vadjust object in front of it, then put
% back the penalty so that the line break will still take place in the
% final result.
%    \begin{macrocode}
\def\eq@revspace{%
  \global\setbox\EQ@vimbox\vbox{\unvbox\EQ@vimbox
    \unpenalty
    \global\setbox\@ne\lastbox}%
  \@@vadjust{\unvbox\@ne}%
  \penalty-\@M
}
%    \end{macrocode}
% The b version is used for the \cs{EQ@copy} box.
%    \begin{macrocode}
\def\eq@revspaceb{%
  \global\setbox\EQ@vimcopy\vbox{\unvbox\EQ@vimcopy
    \unpenalty
    \global\setbox\@ne\lastbox}%
  \@@vadjust{\unvbox\@ne}%
  \penalty-\@M
}
%    \end{macrocode}
% \end{macro}
% \end{macro}
%
%
% \begin{macro}{\eq@break}
% The function \cs{eq@break} does a preliminary linebreak with
% a flag penalty.
%    \begin{macrocode}
\def\eq@break#1{\penalty-1000#1 \keep@glue}
%    \end{macrocode}
% \end{macro}
%
%
%
%
% \section{Choosing optimal line breaks}
% The question of what line width to use when breaking an
% equation into several lines is best examined in the light of an extreme
% example.    Suppose we have a two-column layout and a displayed
% equation falls inside a second-level list with nonzero leftmargin and
% rightmargin.    Then we want to try in succession a number of
% different possibilities.    In each case if the next possibility is
% no wider than the previous one, skip ahead to the one after.
% \begin{enumerate}
% \item First try linewidth(2), the linewidth for the current
% level-2 list.
%
%
% \item If we cannot find adequate linebreaks at that width, next try
% listwidth(2), the sum of leftmargin, linewidth, and rightmargin for
% the current list.
%
%
% \item If we cannot find linebreaks at that width, next try linewidth
% (1) (skipping this step if it is no larger then
% listwidth(2)).
%
%
% \item If we cannot find linebreaks at that width, next try
% listwidth(1).
%
%
% \item If we cannot find linebreaks at that width, next try column
% width.
%
%
% \item If we cannot find linebreaks at that width, next try text
% width.
%
%
% \item If we cannot find linebreaks at that width, next try equation
% width, if it exceeds text width (\ie , if the style allows equations
% to extend into the margins).
%
%
% \end{enumerate}
%
%
% \begin{figure}
%   \centering
%   needs work
% \caption{first-approximation parshape for equations}\label{f:parshape-1}
% \end{figure}
%
% At any given line width, we run through a series of parshape
% trials and, essentially, use the first one that gives decent line
% breaks.
% But the process is a bit more complicated in fact.
% In order to do a really good job of setting up the parshapes, we
% need to know how many lines the equation will require.
% And of course the number of lines needed depends on the parshape!
% So as our very first trial we run a simple first-approximation
% parshape (Figure~\vref{f:parshape-1}) whose
% main purpose is to get an estimate on the number of lines that will be
% needed; it chooses a uniform indent for all lines after the first one
% and does not take any account of the equation number.
% A substantial majority of equations only require one line anyway,
% and for them this first trial will succeed.
% In the one-line case if there is an equation number and it doesn't
% fit on the same line as the equation body, we don't go on to other
% trials because breaking up the equation body will not gain us
% anything\mdash we know that we'll have to use two lines in any case
% \mdash  so we might as well keep the equation body together on one line
% and shift the number to a separate line.
%
% If we learn from the first trial that the equation body
% requires more than one line, the next parshape trial involves adjusting
% the previous parshape to leave room for the equation number, if
% present.    If no number is present, again no further trials are
% needed.
%
% Some remarks about parshape handling.    The \tex
% primitive doesn't store the line specs anywhere, \verb"\the\parshape"
% only returns the number of line specs.    This makes it well nigh
% impossible for different packages that use \cs{parshape} to work
% together.    Not that it would be terribly easy for the package
% authors to make inter-package collaboration work, if it were
% possible.    If we optimistically conjecture that
% someone some day may take on such a task, then the thing to do,
% obviously, is provide a parshape interface that includes a record of all
% the line specs.    For that we designate a macro \cs{@parshape}
% which includes not only the line specs, but also the line count and even
% the leading \cs{parshape} token.
% This allows it to be directly executed without an auxiliary if-empty
% test.    It should include a trailing \cs{relax} when it has a
% nonempty value.
%    \begin{macrocode}
\let\@parshape\@empty
%    \end{macrocode}
%
%
% The function \cs{eq@measure} runs line-breaking trials
% on the copy of the equation body that is stored in the box register
% \cs{EQ@copy}, trying various possible layouts in order of
% preference until we get successful line breaks, where \quoted{successful}
% means there were no overfull lines.    The result of the trials is,
% first, a parshape spec that can be used for typesetting the real
% equation body in \cs{EQ@box}, and second, some information that
% depends on the line breaks such as the depth of the last line, the
% height of the first line, and positioning information for the equation
% number.   The two main variables in the equation layout are the line
% width and the placement of the equation number, if one is present.
%
%
% \begin{macro}{\eq@measure}
% Run linebreak trials on the equation contents and measure the
% results.
%    \begin{macrocode}
\def\eq@measure{%
%    \end{macrocode}
% If an override value is given for indentstep in the env options, use
% it.
%    \begin{macrocode}
  \ifdim\eq@indentstep=\maxdimen \eq@indentstep\eqindentstep \fi
%    \end{macrocode}
% If \cs{eq@linewidth} is nonzero at this point, it means that
% the user specified a particular target width for this equation.
% In that case we override the normal list of trial widths.
%    \begin{macrocode}
  \ifdim\eq@linewidth=\z@ \else \edef\eq@linewidths{{\the\eq@linewidth}}\fi
  \begingroup \eq@params
  \leftskip\z@skip
%    \end{macrocode}
% Even if \cs{hfuzz} is greater than zero a box whose contents
% exceed the target width by less then hfuzz still has a reported badness
% value of 1000000 (infinitely bad).    Because we use inf-bad
% to test whether a particular trial succeeds or fails, we want to make
% such boxes return a smaller badness.    To this end we include an
% \cs{hfuzz} allowance in \cs{rightskip}.    In fact,
% \cs{eq@params} ensures that \cs{hfuzz} for equations is at
% least 1pt.
%    \begin{macrocode}
  \rightskip\z@\@plus\columnwidth\@minus\hfuzz
%  \eqinfo
  \global\EQ@continue{\eq@trial}%
  \eq@trial % uses \eq@linewidths
  \eq@failout % will be a no-op if the trial succeeded
  \endgroup
%    \end{macrocode}
% \quoted{local} parameter settings are passed outside the endgroup through
% \cs{EQ@trial}.
%    \begin{macrocode}
  \EQ@trial
}
%    \end{macrocode}
% \end{macro}
%    \begin{macrocode}
%<*trace>
\def\debug@showmeasurements{%
  \breqn@debugmsg{=> \number\eq@lines\space lines}%
  \begingroup
  \def\@elt##1X##2{\MessageBreak==== \space\space##1/##2}%
  \let\@endelt\@empty
  \breqn@debugmsg{=> trial info:\eq@measurements}%
  \breqn@debugmsg{=> bounding box: \the\eq@wdT x\the\eq@vspan, badness=\the\eq@badness}%
  \let\@elt\relax \let\@endelt\relax
  \endgroup
}
\def\debug@showmeasurements{%
  \begingroup
  \def\@elt##1X##2{\MessageBreak====   ##1/##2}%
  \let\@endelt\@empty
  \breqn@debugmsg{===> Measurements: \number\eq@lines\space lines
           \eq@measurements
           \MessageBreak
           ==== bounding box: \the\eq@wdT x\the\eq@vspan, badness=\the\eq@badness
           \MessageBreak
           ==== \leftskip=\the\leftskip, \rightskip=\the\rightskip}%
 \endgroup
}
%</trace>
%    \end{macrocode}
%
% Layout Trials Driver
% Basically, trying different sequences of parshapes.
%
%
% \begin{macro}{\EQ@trial}
% Init.
%    \begin{macrocode}
\let\EQ@trial\@empty
%    \end{macrocode}
% \end{macro}
%
%
% \begin{macro}{\EQ@continue}
% This is a token register used to carry trial info past a
% group boundary with only one global assignment.
%    \begin{macrocode}
\newtoks\EQ@continue
%    \end{macrocode}
% \end{macro}
%
%
% \begin{macro}{\EQ@widths}
% This is used for storing the actual line-width info of the equation
% contents after breaking.
%    \begin{macrocode}
\let\EQ@widths\@empty
%    \end{macrocode}
% \end{macro}
% \begin{macro}{\EQ@fallback}
%    \begin{macrocode}
\let\EQ@fallback\@empty
%    \end{macrocode}
% \end{macro}
% \begin{macro}{\eq@linewidths}
% This is the list of target widths for line breaking.
%
%========================================
% BRM: Odd; I don't think I've seen this use anything but \cs{displaywidth}...
%    \begin{macrocode}
\def\eq@linewidths{\displaywidth\linewidth\columnwidth}
%    \end{macrocode}
% \end{macro}
%
%
% \begin{macro}{\eq@trial}
% The \cs{eq@trial} function tries each candidate
% line width in \cs{eq@linewidths} until an equation layout is found
% that yields satisfactory line breaks.
%    \begin{macrocode}
\def\eq@trial{%
  \ifx\@empty\eq@linewidths
    \global\EQ@continue{}%
  \else
    \iffalse{\fi \@xp\eq@trial@a \eq@linewidths}%
  \fi
  \the\EQ@continue
}
%    \end{macrocode}
% \end{macro}
%
%
% \begin{macro}{\eq@trial@a}
% The \cs{eq@trial@a} function reads the leading line
% width from \cs{eq@linewidths}; if the new line width is greater
% than the previous one, start running trials with it; otherwise do
% nothing with it.
% Finally, run a peculiar \cs{edef} that leaves
% \cs{eq@linewidths} redefined to be the tail of the list.
% If we succeed in finding satisfactory line breaks
% for the equation, we will reset \cs{EQ@continue} in such a
% way that it will terminate the current trials.
% An obvious branch here would be to check whether the width of
% \cs{EQ@copy} is less than \cs{eq@linewidth} and go immediately
% to the one-line case if so.
% However, if the equation contains more than one RHS, by
% default each additional RHS starts on a new line\mdash \ie , we want
% the ladder layout anyway.
% So we choose the initial trial on an assumption of multiple lines
% and leave the one-line case to fall out naturally at a later point.
%    \begin{macrocode}
\def\eq@trial@a#1{%
  \dim@c#1\relax
  \if T\eq@frame \eq@frame@adjust\dim@c \fi
  \ifdim\dim@c>\eq@linewidth
    \eq@linewidth\dim@c
%<trace>    \breqn@debugmsg{Choose Shape for width(#1)=\the\eq@linewidth}%
    \let\eq@trial@b\eq@trial@d
    \csname eq@try@layout@\eq@layout\endcsname
%<trace>  \else
%<trace>    \breqn@debugmsg{Next width (#1) is shorter; skip it}%
  \fi
  \edef\eq@linewidths{\iffalse}\fi
}
\def\eq@frame@adjust#1{%
  %\addtolength#1{-2\eq@framewd-2\eq@framesep}%
  \dim@a\eq@framewd \advance\dim@a\eq@framesep
  \advance#1-2\dim@a
}
%    \end{macrocode}
% \end{macro}
%========================================
% Note curious control structure.
% Try to understand interaction of \cs{EQ@fallback}, \cs{EQ@continue},
% \cs{eq@failout}
%    \begin{macrocode}
\def\eq@trial@succeed{%
  \aftergroup\@gobbletwo % cancel the \EQ@fallback code; see \eq@trial@c (?)
  \global\EQ@continue{\eq@trial@done}%
}
%    \end{macrocode}
% \begin{macro}{\eq@trial@done}
% Success.
%    \begin{macrocode}
\def\eq@trial@done{%
%<trace>  \breqn@debugmsg{End trial: Success!}%
  \let\eq@failout\relax
}
%    \end{macrocode}
% \end{macro}
%
% \begin{macro}{\eq@trial@init}
% This is called from \cs{eq@trial@b} to initialize or
% re-initialize certain variables as needed when running one or more
% trials at a given line width.
% By default assume success, skip the fallback code.
%    \begin{macrocode}
\def\eq@trial@init{\global\let\EQ@fallback\eq@nextlayout}
%    \end{macrocode}
% \end{macro}
% \begin{macro}{\eq@nextlayout}
%
% In the fallback case cancel the current group to avoid unnecessary
% group nesting (with associated save-stack cost, \etc ).
%    \begin{macrocode}
\def\eq@nextlayout#1{%
  \endgroup
%<trace>  \breqn@debugmsg{Nope ... that ain't gonna work.}%
  \begingroup #1%
}
%    \end{macrocode}
% \end{macro}
% \begin{macro}{\eq@failout}
%
% .
%    \begin{macrocode}
\def\eq@failout{%
%<trace>\breqn@debugmsg{End trial: failout}%
  \global\let\EQ@trial\EQ@last@trial
}
%    \end{macrocode}
% \end{macro}
% \begin{macro}{\eq@trial@save}
%
% Save the parameters of the current trial.
%    \begin{macrocode}
\def\eq@trial@save#1{%
%<*trace>
%  \begingroup \def\@elt##1X##2{\MessageBreak==== \space\space##1/##2}\let\@endelt\@empty\breqn@debugmsg{=> trial info:\eq@measurements}%
%         \breqn@debugmsg{=> bounding box: \the\eq@wdT x\the\eq@vspan, badness=\the\eq@badness\MessageBreak}%
%         \let\@elt\relax \let\@endelt\relax
%  \endgroup
%</trace>
  \xdef#1{%
    \eq@linewidth\the\eq@linewidth
    % save info about the fit
    \eq@lines\the\eq@lines \eq@badness\the\eq@badness \def\@nx\eq@badline{\eq@badline}%
    % save size info
    \eq@wdT\the\eq@wdT \eq@wdMin\the\eq@wdMin
    \eq@vspan\the\eq@vspan \eq@dp\the\eq@dp \eq@firstht\the\eq@firstht
    % save info about the LHS
    \eq@wdL\the\eq@wdL \def\@nx\EQ@hasLHS{\EQ@hasLHS}%
    % save info about the numbering
    \def\@nx\eq@hasNumber{\eq@hasNumber}%
    % save info about the chosen layout
    \def\@nx\eq@layout{\eq@layout}%
    \def\@nx\eq@parshape{\@parshape}%
    \def\@nx\eq@measurements{\eq@measurements}%
    \def\@nx\adjust@rel@penalty{\adjust@rel@penalty}%
    \def\@nx\eq@shiftnumber{\eq@shiftnumber}%
    \def\@nx\eq@isIntertext{\@False}%
  }%
}
%    \end{macrocode}
% \end{macro}
% \begin{macro}{\eq@trial@b}
%
% By default this just runs \cs{eq@trial@c}; \cf
% \cs{eq@trial@d}.
%    \begin{macrocode}
\def\eq@trial@b{\eq@trial@c}
%    \end{macrocode}
% \end{macro}
%
%
% \begin{macro}{\eq@trial@c}
%
% Run the equation contents through the current parshape.
%    \begin{macrocode}
\def\eq@trial@c#1#2{%
%<trace>  \breqn@debugmsg{Trying layout "#1" with\MessageBreak==== parshape\space\@xp\@gobble\@parshape}%
  \begingroup
  \eq@trial@init
  \def\eq@layout{#1}%
  \setbox\z@\vbox{%
    \hfuzz\maxdimen
    \eq@trial@p % run the given parshape
    \if\@Not{\eq@badline}%
      \eq@trial@save\EQ@trial
%    \end{macrocode}
% If there is a number, try the same parshape again with adjustments
% to make room for the number.
%
% This is an awkward place for this:
% It only allows trying to fit the number w/the SAME layout shape!
%    \begin{macrocode}
      \if\eq@hasNumber\eq@retry@with@number\fi
      \if L\eq@layout \eq@check@density
      \else
        \if\@Not{\eq@badline}%
           \eq@trial@succeed
        \fi
      \fi
    \else
      \eq@trial@save\EQ@last@trial
    \fi
  }%
  \EQ@fallback{#2}%
  \endgroup
}
%    \end{macrocode}
% \end{macro}
% \begin{macro}{\eq@trial@d}
%    \begin{macrocode}
\def\eq@trial@d#1#2{\eq@trial@c{#1}{}}
%    \end{macrocode}
% \end{macro}
%
% \begin{macro}{\eq@check@density}
%    \begin{macrocode}
\def\eq@check@density{%
%<trace>  \breqn@debugmsg{Checking density for layout L}%
  \if\@Or{\@Not\EQ@hasLHS}{\eq@shortLHS}%
%<trace>    \breqn@debugmsg{Density check: No LHS, or is short; OK}%
    \eq@trial@succeed
  \else\if\eq@dense@enough
    \eq@trial@succeed
  \fi\fi
}
%    \end{macrocode}
% \end{macro}
%
% \begin{macro}{\eq@shortLHS}
% Test to see if we need to apply the \cs{eq@dense@enough} test.
%    \begin{macrocode}
\def\eq@shortLHS{\ifdim\eq@wdL>.44\eq@wdT 1\else 0\fi 0}
%    \end{macrocode}
% \end{macro}
%
%\verb|\def\eq@shortLHS{\@False}|
%========================================
% \begin{macro}{\eq@trial@p}
% Run a trial with the current \cs{@parshape} and measure it.
%    \begin{macrocode}
\def\eq@trial@p{%
  \@parshape %
  \eq@dump@box\unhcopy\EQ@copy
  {\@@par}% leave \parshape readable
  \eq@lines\prevgraf
  \eq@fix@lastline
  \let\eq@badline\@False
  \if i\eq@layout \ifnum\eq@lines>\@ne \let\eq@badline\@True \fi\fi
  \eq@curline\eq@lines % loop counter for eq@measure@lines
  \let\eq@measurements\@empty
  \eq@ml@record@indents
  \eq@measure@lines
  \eq@recalc
%<trace>  \debug@showmeasurements
}
%    \end{macrocode}
% \end{macro}
%
%
% \begin{macro}{\adjust@rel@penalty}
% Normally this is a no-op.
%    \begin{macrocode}
\let\adjust@rel@penalty\@empty
%    \end{macrocode}
% \end{macro}
%
% \begin{macro}{\eq@fix@lastline}
%  Remove parfillskip from the last line box.
%    \begin{macrocode}
\def\eq@fix@lastline{%
  \setbox\tw@\lastbox \dim@b\wd\tw@
  \eq@dp\dp\tw@
%    \end{macrocode}
% Remove \cs{parfillskip} but retain \cs{rightskip}.
% Need to keep the original line width for later shrink testing.
%    \begin{macrocode}
  \nointerlineskip\hbox to\dim@b{\unhbox\tw@
    \skip@c\lastskip \unskip\unskip\hskip\skip@c
  }%
}
%    \end{macrocode}
% \end{macro}
%
% \begin{macro}{\eq@recalc}
% Calculate \cs{eq@wdT} et cetera.
%    \begin{macrocode}
\def\eq@recalc{%
  \eq@wdT\z@ \eq@wdMin\maxdimen \eq@vspan\z@skip \eq@badness\z@
  \let\@elt\eq@recalc@a \eq@measurements \let\@elt\relax
}
%    \end{macrocode}
% \end{macro}
%
%
% \begin{macro}{\eq@recalc@a}
%    \begin{macrocode}
\def\eq@recalc@a#1x#2+#3\@endelt{%
  \eq@firstht#2\relax
  \let\@elt\eq@recalc@b
  \@elt#1x#2+#3\@endelt
}
%    \end{macrocode}
% \end{macro}
%
%
% \begin{macro}{\eq@recalc@b}
%    \begin{macrocode}
\def\eq@recalc@b#1X#2,#3x#4+#5@#6\@endelt{%
  \setlength\dim@a{#2+#3}%
  \ifdim\dim@a>\eq@wdT \eq@wdT\dim@a \fi
  \ifdim\dim@a<\eq@wdMin \eq@wdMin\dim@a \fi
  \eq@dp#5\relax
  \addtolength\eq@vspan{#1+#4+\eq@dp}%
%    \end{macrocode}
% Record the max badness of all the lines in \cs{eq@badness}.
%    \begin{macrocode}
  \ifnum#6>\eq@badness \eq@badness#6\relax\fi
}
%    \end{macrocode}
% \end{macro}
%
% \begin{macro}{\eq@layout}
% A value of \verb"?" for \cs{eq@layout} means that we should
% deduce which layout to use by looking at the size of the components.
% Any other value means we have a user-specified override on the
% layout.
%
% Layout Definitions.
% Based on initial equation measurements, we can choose a sequence of
% candidate parshapes that the equation might fit into.
% We accept the first shape that `works', else fall to next one.
% [The sequence is hardcoded in the \cs{eq@try@layout@}<shape>
%  Would it be useful be more flexible? (eg. try layouts LDA, in order...)]
%    \begin{macrocode}
\def\eq@layout{?}
%    \end{macrocode}
% \end{macro}
%
%
% \begin{macro}{\eq@try@layout@?}
% This is a branching function used to choose a suitable layout if
% the user didn't specify one in particular.
%
% Default layout:
%  Try Single line layout first, else try Multiline layouts
%    \begin{macrocode}
\@namedef{eq@try@layout@?}{%
  \let\eq@trial@b\eq@trial@c
  \edef\@parshape{\parshape 1 0pt \the\eq@linewidth\relax}%
%  \eq@trial@b{i}{\eq@try@layout@multi}%
  \setlength\dim@a{\wd\EQ@copy-2em}% Fudge; can't shrink more than this?
  % if we're in a numbered group, try hard to fit within the numbers
  \dim@b\eq@linewidth
  \if\eq@shiftnumber\else\if\eq@group
    \if\eq@hasNumber\addtolength\dim@b{-\wd\EQ@numbox-\eqnumsep}%
    \else\if\grp@hasNumber\addtolength\dim@b{-\wd\GRP@numbox-\eqnumsep}%
  \fi\fi\fi\fi
  \ifdim\dim@a<\dim@b% Do we even have a chance of fitting to one line?
%<trace>    \breqn@debugmsg{Choose Shape: (\the\wd\EQ@copy) may fit in \the\dim@b}%
%    \end{macrocode}
% BRM: assuming it might fit, don't push too hard
%    \begin{macrocode}
    \setlength\dim@b{\columnwidth-\dim@a+\eq@wdCond}%
    \rightskip\z@\@plus\dim@b\@minus\hfuzz
    \eq@trial@b{i}{\eq@try@layout@multi}%
  \else
%<*trace>
  \breqn@debugmsg{Choose Shape: Too long (\the\wd\EQ@copy) for one line
            (free width=\the\dim@b)}%
%</trace>
   \eq@try@layout@multi
  \fi
}
%    \end{macrocode}
% Layout Multiline layout:
%  If no LHS, try Stepped(S) layout
%  Else try Stepped(S), Ladder(L), Drop-ladder(D) or Stepladder(l), depending on LHS length.
%    \begin{macrocode}
\def\eq@try@layout@multi{%
  \if\EQ@hasLHS
    \ifdim\eq@wdL>\eq@linewidth
%<trace>       \breqn@debugmsg{Choose Shape: LHS \the\eq@wdL > linewidth}%
%    \end{macrocode}
% Find the total width of the RHS.
% If it is relatively short, a step layout is the thing to try.
%    \begin{macrocode}
       \setlength\dim@a{\wd\EQ@copy-\eq@wdL}%
       \ifdim\dim@a<.25\eq@linewidth \eq@try@layout@S
       \else \eq@try@layout@l
       \fi
    % BRM: Originally .7: Extreme for L since rhs has to wrap within the remaining 30+%!
    \else\ifdim\eq@wdL>.50\eq@linewidth
%<*trace>
      \breqn@debugmsg{Choose Shape: LHS (\the\eq@wdL) > .50 linewidth (linewidth=\the\eq@linewidth)}%
%</trace>
      \eq@try@layout@D
    \else
%<trace>      \breqn@debugmsg{Choose Shape: LHS (\the\eq@wdL) not extraordinarily wide}%
      \eq@try@layout@L
    \fi\fi
  \else
%<trace>    \breqn@debugmsg{Choose Shape: No LHS here}%
%    \end{macrocode}
% Try one-line layout first, then step layout.
%    \begin{macrocode}
    \eq@try@layout@S % (already checked case i)
  \fi
}
%    \end{macrocode}
% \end{macro}
%
% \begin{macro}{\eq@try@layout@D}
% Change the penalty before the first mathrel symbol to encourage a
% break there.
%
% Layout D=Drop-Ladder Layout, for wide LHS.
% \begin{literalcode}
%   LOOOOOOOONG LHS
%    = RHS
%    = ...
% \end{literalcode}
% If fails, try Almost-Columnar layout
%    \begin{macrocode}
\def\eq@try@layout@D{%
  \setlength\dim@a{\eq@linewidth -\eq@indentstep}%
  \edef\@parshape{\parshape 2
    0pt \the\eq@wdL\space \the\eq@indentstep\space \the\dim@a\relax
  }%
  \def\adjust@rel@penalty{\penalty-99 }%
  \eq@trial@b{D}{\eq@try@layout@A}%
}
%    \end{macrocode}
% \end{macro}
% \begin{macro}{\eq@try@layout@L}
% Try a straight ladder layout.
% Preliminary filtering ensures that \cs{eq@wdL} is less than 70%
% of the current line width.
% \begin{literalcode}
% Layout L=Ladder layout
%  LHS = RHS
%      = RHS
%      ...
% \end{literalcode}
% If fails, try Drop-ladder layout.
% NOTE: This is great for some cases (multi relations?), but
% tends to break really badly when it fails....
%    \begin{macrocode}
\def\eq@try@layout@L{%
  \setlength\dim@b{\eq@linewidth-\eq@wdL}%
  \edef\@parshape{\parshape 2 0pt \the\eq@linewidth\space
    \the\eq@wdL\space \the\dim@b\relax
  }%
  \eq@trial@b{L}{\eq@try@layout@D}%
}
%    \end{macrocode}
% \end{macro}
%
% \begin{macro}{\eq@try@layout@S}
% In the \dquoted{stepped} layout there is no LHS, or LHS
% is greater than the line width and RHS is small.
% Then we want to split up the equation into lines of roughly equal
% width and stagger them downwards to the right, leaving a small amount of
% whitespace on both sides.
% But also, if there is an equation number, we want to try first a
% layout that leaves room for the number.
% Otherwise it would nearly always be the case that the number would
% get thrown on a separate line.
%
% Layout S=Stepped layout, typically no LHS or very long, variations on
% \begin{literalcode}
%  STUFF ....
%    + MORE STUFF ...
%      + MORE STUFF ...
% \end{literalcode}
% If fails, try Almost-Columnar layout
%    \begin{macrocode}
\def\eq@try@layout@S{%
  \setlength\dim@b{\eq@linewidth-2\eqmargin}% \advance\dim@b-1em%
%    \end{macrocode}
% About how many lines will we need if dim@b is the line width?
%    \begin{macrocode}
  \int@a\wd\EQ@copy \divide\int@a\dim@b
%    \end{macrocode}
% Adjust the target width by number of lines times indentstep.
% We don't need to decrement \cs{int@a} because \tex
% division is integer division with truncation.
%    \begin{macrocode}
  \addtolength\dim@b{-\int@a\eq@indentstep}%
%    \end{macrocode}
% Adjust for equation number.
% But try not to leave too little room for the equation body.
%    \begin{macrocode}
  \if\eq@hasNumber
    \ifdim\dim@b>15em%
%      \advance\dim@b-\eqnumsep \advance\dim@b-\wd\EQ@numbox
      \addtolength\dim@b{-\eq@wdNum}%
    \fi
  \fi
%    \end{macrocode}
% Now some hand-waving to set up the parshape.
%    \begin{macrocode}
  \int@b\z@
  \def\@tempa{\dim}%
  \edef\@parshape{\parshape 2 0pt \the\dim@b\space
    \the\eqmargin\space\the\dim@b\relax}%
  \eq@trial@b{S}{\eq@try@layout@A}%
}
%    \end{macrocode}
% \end{macro}
%
% \begin{macro}{\eq@try@layout@l}
% This is the \dquoted{step-ladder} layout: similar to the drop-ladder
% layout but the LHS is too wide and needs to be broken up.
%
% Layout l = Stepladder
% Similar to Drop-Ladder, but LHS is long and needs to be broken up.
% If fails, try Almost-Columnar layout
%    \begin{macrocode}
\def\eq@try@layout@l{%
  \setlength\dim@a{\eq@linewidth -\eq@indentstep}%
  \int@a\eq@wdL \divide\int@a\dim@a
  \advance\int@a\tw@
  \edef\@parshape{\parshape \number\int@a\space
    0pt \the\eq@linewidth
  }%
  \advance\int@a-\tw@
  \setlength\dim@b{2\eq@indentstep}%
  \setlength\dim@c{\eq@linewidth -\dim@b}%
  \edef\@parshape{\@parshape
    \replicate{\int@a}{\space\the\eq@indentstep\space\the\dim@a}%
    \space\the\dim@b\space\the\dim@c\relax
  }%
  \eq@trial@b{l}{\eq@try@layout@A}%
}
%    \end{macrocode}
% \end{macro}
%
% \begin{macro}{\eq@try@layout@A}
% In the \dquoted{almost-columnar} layout, which is the layout of last
% resort, we let all lines run to the full width and leave the adjusting
% of the indents to later.
%
% Layout A = Almost-Columnar layout.
% Pretty much straight full width, more of a last-resort.
% If fails, give up.
%    \begin{macrocode}
\def\eq@try@layout@A{%
  \edef\@parshape{\parshape 1 0pt \the\eq@linewidth\relax}%
  \if\EQ@hasLHS \def\adjust@rel@penalty{\penalty-99 }\fi
  \eq@trial@b{A}{}%
}
%    \end{macrocode}
% \end{macro}
%
% \begin{macro}{\eq@shiftnumber}
%  MH: Should be moved to a section where all keys are set to defaults.
%    \begin{macrocode}
\let\eq@shiftnumber\@False
%    \end{macrocode}
% \end{macro}
%
% \begin{macro}{\eq@retry@with@number@a}
% Number placement adjustments
%    \begin{macrocode}
\def\eq@retry@with@number{%
 \if\eq@shiftnumber
%<trace>   \breqn@debugmsg{Place number: Shifted number requested}%
 \else
%    \end{macrocode}
% Condition and right numbers? We're just going to have to shift.
%    \begin{macrocode}
    \ifdim\eq@wdCond>\z@\if R\eqnumside
%<trace>      \breqn@debugmsg{Place number: Condition w/Right number => Shift number}%
      \let\eq@shiftnumber\@True
    \fi\fi
%    \end{macrocode}
% Compute free space.
%    \begin{macrocode}
%    \dim@b\eqnumsep\advance\dim@b\wd\EQ@numbox
    \dim@b\eq@wdNum
    \if L\eqnumside
      \ifdim\@totalleftmargin>\dim@b\dim@b\@totalleftmargin\fi
    \else
      \addtolength\dim@b{\@totalleftmargin}%
    \fi
    \setlength\dim@a{\eq@linewidth-\dim@b}%\advance\dim@a1em\relax% Allowance for shrink?
%    \end{macrocode}
% Set up test against 1-line case only if not in a group
%    \begin{macrocode}
    \int@a\@ne\if\eq@group\int@a\maxint\fi
%    \end{macrocode}
% Now check for cases.
%    \begin{macrocode}
    \if\eq@shiftnumber               % Already know we need to shift
    \else\ifdim\eq@wdT<\dim@a % Fits!
%    \end{macrocode}
% left \& right skips will be done later, and parshape adjusted if
% needed.
%    \begin{macrocode}
%<trace>      \breqn@debugmsg{Place number: eqn and number fit together}%
%    \else\ifnum\eq@lines=\int@a %  Shift, if single line, unless inside a dgroup.
%    \end{macrocode}
% NOTE: this is too strong for dgroup!
%    \begin{macrocode}
%<*trace>
%      \breqn@debugmsg{Place number: single line too long with number => Shift number \the\int@a}%
%</trace>
%      \let\eq@shiftnumber\@True
    \else
%    \end{macrocode}
%      % Retry: use leftskip for space for number(for now; whether
%      % right/left) \& adjust parshape
%    \begin{macrocode}
%      \leftskip\wd\EQ@numbox\advance\leftskip\eqnumsep
      \setlength\leftskip{\eq@wdNum}%
      \setlength\rightskip{\z@\@plus\dim@a}%
      \adjust@parshape\@parshape
%<*trace>
      \breqn@debugmsg{Place number: Try with \leftskip=\the\leftskip, \rightskip=\the\rightskip,
                \MessageBreak==== parshape\space\@xp\@gobble\@parshape}%
%</trace>
      \nointerlineskip
      \edef\eq@prev@lines{\the\eq@lines}%
      \edef\eq@prev@badness{\the\eq@badness}% BRM
      \eq@trial@p
      \int@a\eq@prev@badness\relax\advance\int@a 50\relax%?
      \int@b\eq@prev@lines  \if\eq@group\advance\int@b\@ne\fi% Allow 1 extra line in group
      \ifnum\eq@lines>\int@b % \eq@prev@lines
%<trace>        \breqn@debugmsg{Adjustment causes more breaks => Shift number}%
        \let\eq@shiftnumber\@True
      \else\if\eq@badline
%<trace>        \breqn@debugmsg{Adjustment causes bad lines (\the\eq@badness) => Shift}%
        \let\eq@shiftnumber\@True
      \else\ifnum\eq@badness>\int@a % BRM: New case
%<*trace>
        \breqn@debugmsg{Adjustment is badder than previous
                  (\the\eq@badness >> \eq@prev@badness) => Shift}%
%</trace>
        \let\eq@shiftnumber\@True
      \else
%<trace>        \breqn@debugmsg{Adjustment succeeded}%
      \fi\fi%\fi
   \fi\fi\fi
%    \end{macrocode}
%       If we got shifted, restore parshape, etc,
%    \begin{macrocode}
   \if\eq@shiftnumber
     \EQ@trial% Restore parshape & other params,
     \leftskip\z@\let\eq@shiftnumber\@True % But set shift & leftskip
     \edef\@parshape{\eq@parshape}% And copy saved parshape back to `working copy' !?!?
   \fi
   \eq@trial@save\EQ@trial      % Either way, save the trial state.
 \fi
}
%    \end{macrocode}
% \end{macro}
%
% \begin{macro}{\adjust@parshape}
% Varies depending on the layout.
%
% Adjust a parshape variable for a given set of left\textbar right skips.
% Note that the fixed part of the left\textbar right skips effectively
% comes out of the parshape widths (NOT in addition to it).
% We also must trim the widths so that the sum of skips, indents
% and widths add up to no more than the \cs{eq@linewidth}.
%    \begin{macrocode}
\def\adjust@parshape#1{%
  \@xp\adjust@parshape@a#1\relax
  \edef#1{\temp@a}%
}
%    \end{macrocode}
% \end{macro}
%
%
% \begin{macro}{\adjust@parshape@a}
% \begin{macro}{\adjust@parshape@b}
%    \begin{macrocode}
\def\adjust@parshape@a#1 #2\relax{%
  \setlength\dim@a{\leftskip+\rightskip}%
  \edef\temp@a{#1}%
  \adjust@parshape@b#2 @ @ \relax
}
\def\adjust@parshape@b#1 #2 {%
  \ifx @#1\edef\temp@a{\temp@a\relax}%
    \@xp\@gobble
  \else
    \dim@b#1\relax
    \dim@c#2\relax
    \addtolength\dim@c{\dim@a+\dim@b}%
    \ifdim\dim@c>\eq@linewidth\setlength\dim@c{\eq@linewidth}\fi
     \addtolength\dim@c{-\dim@b}%
    \edef\temp@a{\temp@a\space\the\dim@b\space\the\dim@c}%
  \fi
  \adjust@parshape@b
}
%    \end{macrocode}
% \end{macro}
% \end{macro}
%
% \begin{macro}{\eq@ml@record@indents}
% Plunk the parshape's indent values into an array for easy access
% when constructing \cs{eq@measurements}.
%    \begin{macrocode}
\def\eq@ml@record@indents{%
  \int@a\z@
  \def\@tempa{%
    \advance\int@a\@ne
    \@xp\edef\csname eq@i\number\int@a\endcsname{\the\dim@a}%
    \ifnum\int@a<\int@b \afterassignment\@tempb \fi
    \dim@a
  }%
  \def\@tempb{\afterassignment\@tempa \dim@a}%
  \def\@tempc##1##2 {\int@b##2\afterassignment\@tempa\dim@a}%
  \@xp\@tempc\@parshape
}
%    \end{macrocode}
% \end{macro}
%
% \begin{macro}{\@endelt}
% This is a scan marker.
% It should get a non-expandable definition.
% It could be \cs{relax}, but let's try a chardef instead.
%    \begin{macrocode}
\chardef\@endelt=`\?
%    \end{macrocode}
% \end{macro}
%
%
% \begin{macro}{\eq@measurements}
% This is similar to a parshape spec but for each line we record more
% info: space above, indent, width x height + dp, and badness.
%    \begin{macrocode}
\def\eq@measurements{%
  \@elt 4.5pt/5.0pt,66.0ptx6.8pt+2.4pt@27\@endelt
  ...
}
%    \end{macrocode}
% \end{macro}
%
%
% \begin{macro}{\eq@measure@lines}
% Loop through the list of boxes to measure things like total
% height (including interline stretch), \etc .    We check the
% actual width of the current line against the natural width \mdash
% after removing rightskip \mdash  in case the former is
% \emph{less} than the latter because of shrinkage.    In that
% case we do not want to use the natural width for RHS-max-width because
% it might unnecessarily exceed the right margin.
%    \begin{macrocode}
\def\eq@measure@lines{%
  \let\eq@ml@continue\eq@measure@lines
  \setbox\tw@\lastbox \dim@b\wd\tw@ % find target width of line
  \setbox\z@\hbox to\dim@b{\unhbox\tw@}% check for overfull
  \eq@badness\badness
  \ifnum\eq@badness<\inf@bad \else \let\eq@badline\@True \fi
  \eq@ml@a \eq@ml@continue
}
%    \end{macrocode}
% \end{macro}
%
%
% \begin{macro}{\eq@ml@a}
%
%    \begin{macrocode}
\def\eq@ml@a{%
  \setbox\tw@\hbox{\unhbox\z@ \unskip}% find natural width
%<*trace>
  \ifnum\eq@badness<\inf@bad\else\breqn@debugmsg{!?! Overfull: \the\wd\tw@ >\the\dim@b}\fi
%</trace>
%    \end{macrocode}
% Is actual width less than natural width?
%    \begin{macrocode}
  \ifdim\dim@b<\wd\tw@ \setlength\dim@a{\dim@b}% shrunken line
  \else                \setlength\dim@a{\wd\tw@}% OK to use natural width
  \fi
  \addtolength\dim@a{-\leftskip}% BRM: Deduct the skip if we're retrying w/number
%    \end{macrocode}
% If there's no aboveskip, assume we've reached the top of the
% equation.
%    \begin{macrocode}
  \skip@a\lastskip \unskip \unpenalty
  \ifdim\skip@a=\z@
    \let\eq@ml@continue\relax % end the recursion
  \else
    % Sum repeated vskips if present
    \def\@tempa{%
      \ifdim \lastskip=\z@
      \else \addtolength\skip@a{\lastskip}\unskip\unpenalty \@xp\@tempa
      \fi
    }%
  \fi
  \edef\eq@measurements{\@elt
    \the\skip@a\space X% extra space to facilitate extracting only the
                        % dimen part later
    \csname eq@i%
      \ifnum\eq@curline<\parshape \number\eq@curline
      \else\number\parshape
      \fi
    \endcsname,\the\dim@a x\the\ht\tw@+\the\dp\tw@ @\the\eq@badness\@endelt
    \eq@measurements
  }%
  \advance\eq@curline\m@ne
  \ifnum\eq@curline=\z@ \let\eq@ml@continue\relax\fi
}
%    \end{macrocode}
% \end{macro}
%
%
% \begin{macro}{\eq@ml@vspace}
%
% Handle an embedded vspace.
%    \begin{macrocode}
\def\eq@ml@vspace{%
  \global\advance\eq@vspan\lastskip \unskip\unpenalty
  \ifdim\lastskip=\z@ \else \@xp\eq@ml@vspace \fi
}
%    \end{macrocode}
% \end{macro}
%
%
% \begin{macro}{\eq@dense@enough}
%
%    \begin{macrocode}
\def\eq@dense@enough{%
  \ifnum\eq@lines<\thr@@
%<trace>    \breqn@debugmsg{Density check: less than 3 lines; OK}%
    \@True
  \else
    \ifdim\eq@wdL >.7\eq@wdT
%<trace>     \breqn@debugmsg{Density check: LHS too long; NOT OK}%
      \@False
    \else \@xp\@xp\@xp\eq@dense@enough@a
    \fi
  \fi
}
%    \end{macrocode}
% \end{macro}
%
%
% \begin{macro}{\true@true@true}
% \begin{macro}{\true@false@true}
% \begin{macro}{\false@true@false}
% \begin{macro}{\false@false@false}
%    \begin{macrocode}
\def\true@true@true   {\fi\fi\iftrue \iftrue \iftrue }
\def\true@false@true  {\fi\fi\iftrue \iffalse\iftrue }
\def\false@true@false {\fi\fi\iffalse\iftrue \iffalse}
\def\false@false@false{\fi\fi\iffalse\iffalse\iffalse}
%    \end{macrocode}
% \end{macro}
% \end{macro}
% \end{macro}
% \end{macro}
%
%
% \begin{macro}{\eq@density@factor}
%
% This number specifies, for the ladder layout, how much of the
% equation's bounding box should contain visible material rather than
% whitespace.
% If the amount of visible material drops below this value, then we
% switch to the drop-ladder layout.
% The optimality of this factor is highly dependent on the equation
% contents; .475 was chosen as the default just because it worked well
% with the sample equation, designed to be as average as possible, that I
% used for testing.
%    \begin{macrocode}
\def\eq@density@factor{.475}
%    \end{macrocode}
% \end{macro}
%
%
% \begin{macro}{\eq@dense@enough@a}
%
% Calculate whether there is more
% visible material than whitespace within the equation's bounding box.
% Sum up the actual line widths and compare to the total
% \dquoted{area} of the bounding box.
% But if we have an extremely large number of lines, fall back to an
% approximate calculation that is more conservative about the danger of
% exceeding \cs{maxdimen}.
%    \begin{macrocode}
\def\eq@dense@enough@a{%
  \@True \fi
  \ifnum\eq@lines>\sixt@@n
    \eq@dense@enough@b
  \else
    \dim@b\z@ \let\@elt\eq@delt \eq@measurements
    \dim@c\eq@density@factor\eq@wdT \multiply\dim@c\eq@lines
%<trace>    \breqn@debugmsg{Density check: black \the\dim@b/\eq@density@factor total \the\dim@c}%
    \ifdim\dim@b>\dim@c \true@false@true \else \false@false@false \fi
  \fi
}
%    \end{macrocode}
% \end{macro}
%
%
% \begin{macro}{\eq@delt}
% Args are space-above, indent, width, height, depth, badness.
%    \begin{macrocode}
\def\eq@delt#1X#2,#3x#4+#5@#6\@endelt{\addtolength\dim@b{#3}}%
%    \end{macrocode}
% \end{macro}
%
%
% \begin{macro}{\eq@dense@enough@b}
% This is an approximate calculation used to keep from going over
% \cs{maxdimen} if the number of lines in our trial break is large
% enough to make that a threat.
% If l, t, n represent left-side-width, total-width, and number of
% lines, the formula is
% \begin{literalcode}
% l/t < .4n/(.9n-1)
% \end{literalcode}
% or equivalently, since rational arithmetic is awkward in \tex :
% b
% \begin{literalcode}
% l/t < 4n/(9n-10)
% \end{literalcode}
% .
%    \begin{macrocode}
\def\eq@dense@enough@b{%
  \int@b\eq@wdT \divide\int@b\p@
  \dim@b\eq@wdL \divide\dim@b\int@b
  \dim@c\eq@lines\p@ \multiply\dim@c\f@ur
  \int@b\eq@lines \multiply\int@b 9 \advance\int@b -10%
  \divide\dim@c\int@b
%<trace>  \breqn@debugmsg{Density check: l/t \the\dim@b\space< \the\dim@c\space 4n/(9n-10)?}%
  \ifdim\dim@b<\dim@c \true@true@true \else \false@true@false \fi
}
%    \end{macrocode}
% \end{macro}
%
% \begin{macro}{\eq@parshape}
%    \begin{macrocode}
\let\eq@parshape\@empty
%    \end{macrocode}
% \end{macro}
%
%
% \begin{macro}{\eq@params}
% The interline spacing and penalties in \cs{eq@params}
% are used during both preliminary line breaking and final typesetting.
%    \begin{macrocode}
\def\eq@params{%
  \baselineskip\eqlinespacing
  \lineskip\eqlineskip \lineskiplimit\eqlineskiplimit
%    \end{macrocode}
% Forbid absolutely a pagebreak that separates the first line or last
% line of a multiline equation from the rest of it.    Or in other
% words: no equation of three lines or less will be broken at the bottom
% of a page; instead it will be moved whole to the top of the next
% page.    If you really really need a page break that splits the
% first or last line from the rest of the equation, you can always fall
% back to\cs{pagebreak}, I suppose.
%
%    \begin{macrocode}
  \clubpenalty\@M \widowpenalty\@M \interlinepenalty\eqinterlinepenalty
  \linepenalty199 \exhyphenpenalty5000 % was 9999: make breaks at, eg. \* a bit easier.
%    \end{macrocode}
% For equations, hfuzz should be at least 1pt.
% But we have to fake it a little because we are running the equation
% through \tex 's paragrapher.
% In our trials we use minus 1pt in the rightskip rather than hfuzz;
% and we must do the same during final breaking of the equation, otherwise
% in borderline cases \tex  will use two lines instead of one when our
% trial indicated that one line would be enough.
%    \begin{macrocode}
  \ifdim\hfuzz<\p@ \hfuzz\p@ \fi
%\hfuzz=2pt
%  \ifdim\hfuzz<2pt\relax \hfuzz2pt \fi
  \parfillskip\z@skip
%  \hfuzz\z@
%    \end{macrocode}
% Make sure we skip \tex 's preliminary line-breaking pass to save
% processing time.
%    \begin{macrocode}
  \tolerance9999 \pretolerance\m@ne
}
%    \end{macrocode}
% \end{macro}
%
%
%
%
% \section{Equation layout options}
% Using the notation C centered, I indented (applied to
% the equation body), T top, B bottom, M
% middle, L left, R right (applied to the equation number),
% the commonly used equation types are C, CRM, CRB, CLM, CLT,
% I, IRM, IRB, ILM, ILT.    In other words, CLM stands for Centered equation
% body with Left-hand Middle-placed equation number, and IRB stands for
% Indented equation with Right-hand Bottom-placed equation number.
%
% Here are some general thoughts on how to place an equation
% tag. Currently it does not work as desired: the L option positions
% the tag app. 10 lines below the math expression, the RM doesn't
% position the tag on the baseline for single-line math
% expressions. Therefore I am going to first write what I think is
% supposed to happen and then implement it.
%
% Below is a small list where especially the two three specifications
% should be quite obvious, I just don't want to forget anything and it
% is important to the implementation.
% \begin{description}
% \item[Definition 1] If a display consists of exactly one line, the
%   tag should always be placed on the same baseline as the math
%   expression.
% \end{description}
% The remaining comments refer to multi-line displays.
% \begin{description}
% \item[Definition 2] If a tag is to be positioned at the top (T), it
%   should be placed such that the baseline of the tag aligns with the
%   baseline of the top line of the display.
%
% \item[Definition 3] If a tag is to be positioned at the bottom (B),
%   it should be placed such that the baseline of the tag aligns with
%   the baseline of the bottom line of the display.
%
% \item[Definition 4] If a tag is to be positioned vertically centered
%   (M), it should be placed such that the baseline of the tag is
%   positioned exactly halfway between the baseline of the top line of
%   the display and the baseline of the bottom line of the display.
% \end{description}
%
% Definitions 1--3 are almost axiomatic in their
% simplicity. Definition~4 is different because I saw at least two
% possibilities for which area to span:
% \begin{itemize}
% \item Calculate distance from top of top line to the bottom of the
%   bottom line, position the vertical center of the tag exactly
%   halfway between those two extremes.
%
% \item Calculate the distance from the baseline of the top line to
%   the baseline of the bottom line, position the baseline of the tag
%   exactly halfway between these two extremes.
% \end{itemize}
% Additional combinations of these methods are possible but make
% little sense in my opinion. I have two reasons for choosing the
% latter of these possibilities: Firstly, two expressions looking
% completely identical with the exception of a superscript in the
% first line or a subscript in the last line will have the tag
% positioned identically. Secondly, then M means halfway between T and
% B positions which makes good sense and then also automatically
% fulfills Definition~1.
%
% From an implementation perspective, these definitions should also
% make it possible to fix a deficiency in the current implementation,
% namely that the tag does not influence the height of a display, even
% if the display is a single line. This means that two single-line
% expressions in a \env{dgroup} can be closer together than
% \cs{intereqskip} if the math expressions are (vertically) smaller
% than the tag.
%
% \section{Centered Right-Number Equations}
%
% \begin{macro}{\eq@dump@box}
% \arg1  might be \cs{unhbox} or \cs{unhcopy}; \arg2  is
% the box name.
%    \begin{macrocode}
\def\eq@dump@box#1#2{%
%\debug@box#1%
  \noindent #1#2\setbox\f@ur\lastbox \setbox\tw@\lastbox
%    \end{macrocode}
% If the LHS contains shrinkable glue, in an L layout the alignment
% could be thrown off if the first line is shrunk noticeably.
% For the time being, disable shrinking on the left-hand side.
% The proper solution requires more work \begin{dn}
% mjd,1999/03/17
% \end{dn}
% .
%    \begin{macrocode}
  \if L\eq@layout \box\tw@ \else\unhbox\tw@\fi
  \adjust@rel@penalty \unhbox\f@ur
}
%    \end{macrocode}
% \end{macro}
%
% Various typesetting bits, invoked from \cs{eq@finish}
% BRM: This has been extensively refactored from the original breqn,
% initially to get left\textbar right skips and parshape used consistently,
% ultimately to get most things handled the same way, in the same order.
%
% Given that left and right skips have been set,
% typeset the frame, number and equation with the
% given number side and placement
%
%    \begin{macrocode}
\def\eq@typeset@Unnumbered{%
  \eq@typeset@frame
  \eq@typeset@equation
}
\def\eq@typeset@LM{%
  \setlength\dim@a{(\eq@vspan+\ht\EQ@numbox-\dp\EQ@numbox)/2}%
  \eq@typeset@leftnumber
  \eq@typeset@frame
  \eq@typeset@equation
}
%    \end{macrocode}
% Typeset equation and left-top number (and shifted)
%    \begin{macrocode}
\def\eq@typeset@LT{%
  \dim@a\eq@firstht
  \eq@typeset@leftnumber
  \eq@typeset@frame
  \eq@typeset@equation
}
%    \end{macrocode}
% Typeset equation and left shifted number
%    \begin{macrocode}
\def\eq@typeset@LShifted{%
  % place number
  \copy\EQ@numbox \penalty\@M
  \dim@a\eqlineskip
  \if F\eq@frame\else
     \setlength\dim@a{\eq@framesep+\eq@framewd}%
  \fi
  \kern\dim@a
  \eq@typeset@frame
  \eq@typeset@equation
}
%    \end{macrocode}
% Typeset equation and right middle number
%    \begin{macrocode}
\def\eq@typeset@RM{%
  \setlength\dim@a{(\eq@vspan+\ht\EQ@numbox-\dp\EQ@numbox)/2}%
  \eq@typeset@rightnumber
  \eq@typeset@frame
  \eq@typeset@equation
}
%    \end{macrocode}
% Typeset equation and right bottom number
%    \begin{macrocode}
\def\eq@typeset@RB{%
  % NOTE: is \eq@dp useful here
  \setlength\dim@a{\eq@vspan-\ht\EQ@numbox-\dp\EQ@numbox}%
  \eq@typeset@rightnumber
  \eq@typeset@frame
  \eq@typeset@equation
}
%    \end{macrocode}
% Typeset equation and right shifted number
%    \begin{macrocode}
\def\eq@typeset@RShifted{%
  % place number
  \eq@typeset@frame
  \eq@typeset@equation
  \penalty\@M
  \dim@a\eqlineskip
  \if F\eq@frame\else
    \addtolength\dim@a{\eq@framesep+\eq@framewd}%
  \fi
  \parskip\dim@a
  \hbox to\hsize{\hfil\copy\EQ@numbox}\@@par%
}
%    \end{macrocode}
%
% Debugging aid to show all relevant formatting info for a given eqn.
%    \begin{macrocode}
%<*trace>
\def\debug@showformat{%
  \breqn@debugmsg{Formatting Layout:\eq@layout\space Center/indent: \eqindent\space
    Number placement \eqnumside\eqnumplace:
    \MessageBreak==== \eq@linewidth=\the\eq@linewidth, \@totalleftmargin=\the\@totalleftmargin,
    \MessageBreak==== Centered Lines=\theb@@le\eq@centerlines, Shift Number=\theb@@le\eq@shiftnumber,
    \MessageBreak==== \eq@wdT=\the\eq@wdT, \eq@wdMin=\the\eq@wdMin,
    \MessageBreak==== LHS=\theb@@le\EQ@hasLHS: \eq@wdL=\the\eq@wdL,
    \MessageBreak==== \eq@firstht=\the\eq@firstht, \eq@vspan=\the\eq@vspan
    \MessageBreak==== \eq@wdNum=\the\eq@wdNum
    \MessageBreak==== \eq@wdCond=\the\eq@wdCond, \conditionsep=\the\conditionsep,
    \MessageBreak==== \leftskip=\the\leftskip, \rightskip=\the\rightskip,
    \MessageBreak==== \abovedisplayskip=\the\abovedisplayskip,
    \MessageBreak==== \belowdisplayskip=\the\belowdisplayskip
    \MessageBreak==== parshape=\eq@parshape}%
}
%</trace>
%    \end{macrocode}
%
% Set left \& right skips for centered equations,
% making allowances for numbers (if any, right, left) and constraint.
%
% Amazingly, I've managed to collect all the positioning logic for
% centered equations in one place, so it's more manageable.
% Unfortunately, by the time it does all it needs to do,
% it has evolved I'm (re)using so many temp variables, it's becoming
% unmanageble!
%
%    \begin{macrocode}
\def\eq@C@setsides{%
  % \dim@c = space for number, if any, and not shifted.
  \dim@c\z@
  \if\eq@hasNumber\if\eq@shiftnumber\else
    \dim@c\eq@wdNum
  \fi\fi
  % \dim@e = space for condition(on right), if any and formula is only a single line.(to center nicely)
  % but only count it as being right-aligned if we're not framing, since the frame must enclose it.
  \dim@e\z@
  \if F\eq@frame
    \ifnum\eq@lines=\@ne\ifdim\eq@wdCond>\z@
      \setlength\dim@e{\eq@wdCond+\conditionsep}%
  \fi\fi\fi
  % \dim@b = minimum needed on left max(totalleftmargin, left number space)
  \dim@b\z@
  \if L\eqnumside\ifdim\dim@b<\dim@c
    \dim@b\dim@c
  \fi\fi
  \ifdim\dim@b<\@totalleftmargin
    \dim@b\z@
  \else
    \addtolength\dim@b{-\@totalleftmargin}%
  \fi
  % \dim@d = minimum needed on right max(condition, right number space)
  \dim@d\dim@e
  \if R\eqnumside\ifdim\dim@d<\dim@c
    \dim@d\dim@c
  \fi\fi
  % \dim@a = left margin; initially half available space
  % \dim@c = right margin;  ditto
  \setlength\dim@a{(\eq@linewidth-\eq@wdT+\dim@e+\@totalleftmargin)/2}%
  \dim@c=\dim@a
  % If too far to the left
  \ifdim\dim@a<\dim@b
     \addtolength\dim@c{\dim@a-\dim@b}%
     \ifdim\dim@c<\z@\dim@c=\z@\fi
     \dim@a=\dim@b
  % Or if too far to the right
  \else\ifdim\dim@c<\dim@d
     \addtolength\dim@a{\dim@c-\dim@d}%
     \ifdim\dim@a<\z@\dim@a=\z@\fi
     \dim@c=\dim@d
  \fi\fi
  % Now, \dim@d,\dim@e is the left & right glue to center each line for centerlines
  \setlength\dim@e{\eq@wdT-\eq@wdMin}\dim@d=\z@
%    \end{macrocode}
% NOTE: Need some work here centering when there's a condition
%    \begin{macrocode}
%  \advance\dim@e-\eq@wdT\multiply\dim@e-1\relax
%  \if\eq@wdMin<\dim@e\dim@e\eq@wdMin\fi
%  \multiply\dim@e-1\relax\advance\dim@e\eq@wdT
  \dim@d\z@
  \if\eq@centerlines
    \divide\dim@e2\relax
    \dim@d=\dim@e
  \fi
  \setlength\leftskip{\dim@a\@plus\dim@d}%
  \addtolength\dim@e{\dim@c}%
  \setlength\rightskip{\z@\@plus\dim@e}%\@minus5\p@
  % Special case: if framing, reduce the stretchiness of the formula (eg. condition)
  % Or if we have a right number, FORCE space for it
  \dim@b\z@
  \if F\eq@frame\else
    \dim@b\dim@c
  \fi
  \if\@And{\eq@hasNumber}{\@Not{\eq@shiftnumber}}%
    \if R\eqnumside
      \dim@c\eq@wdNum
      \ifdim\dim@c>\dim@b
        \dim@b\dim@c
      \fi
    \fi
  \fi
  % If either of those cases requires hard rightskip, move that part from glue.
  \ifdim\dim@b>\z@
    \addtolength\dim@e{-\dim@c}%
    \rightskip\dim@b\@plus\dim@e%\@minus5\p@
  \fi
  % And peculiar further special case: in indented environs, width isn't where it would seem
  \ifdim\eq@wdCond>\z@
    \addtolength\rightskip{-\@totalleftmargin}%
  \fi
  \parfillskip\z@skip
}
%    \end{macrocode}
%
% Set the left and right side spacing for indented equations
% Some things handled by eq@C@setsides that probably apply here????
% \begin{itemize}
% \item centerlines
% \item \cs{@totalleftmargin}: SHOULD we move farther right?
% \end{itemize}
% Leftskip is normally just the requested indentation
%    \begin{macrocode}
\def\eq@I@setsides{%
  \leftskip\mathindent
%    \end{macrocode}
% But move left, if shifted number presumably because of clashed w/ number?
%    \begin{macrocode}
  \if\eq@shiftnumber
    \setlength\dim@a{\eq@linewidth-\eq@wdT-\mathindent}%
    \ifdim\dim@a<\z@
      \leftskip=\z@ % Or something minimal?
    \fi
  \fi
%    \end{macrocode}
% Push gently from right.
%    \begin{macrocode}
  \dim@a=\z@
  \setlength\dim@b{\eq@linewidth-\leftskip-\eq@wdMin}%
%    \end{macrocode}
% Special case: if framing be much more rigid(?)
%    \begin{macrocode}
  \if F\eq@frame
  \else
    \setlength\dim@a{\eq@linewidth-\leftskip-\eq@wdT}
    \addtolength\dim@b{-\dim@a}%
  \fi
  % Or force the space for right number, if needed
%    \begin{macrocode}
  \if\@And{\eq@hasNumber}{\@Not{\eq@shiftnumber}}%
    \if R\eqnumside
      \dim@c=\eq@wdNum
      \if\dim@c>\dim@a
        \addtolength\dim@b{-\dim@c}%
        \dim@a=\dim@c
      \fi
    \fi
  \fi
  \setlength\rightskip{\dim@a\@plus\dim@b \@minus\hfuzz }%\hfuzz\z@
  \parfillskip\z@skip
}
%    \end{macrocode}
% \paragraph{Typesetting pieces: frame, equation and number (if any)}
% \cs{dim@a} should contain the downward displacement of number's baseline
%    \begin{macrocode}
\def\eq@typeset@leftnumber{%
  \setlength\skip@c{\dim@a-\ht\EQ@numbox}%
  \vglue\skip@c% NON discardable
  \copy\EQ@numbox \penalty\@M
  \kern-\dim@a
}
\def\eq@typeset@rightnumber{%
  \setlength\skip@c{\dim@a-\ht\EQ@numbox}%
  \vglue\skip@c% NON discardable
  \hbox to \hsize{\hfil\copy\EQ@numbox}\penalty\@M
  \kern-\dim@a
}
\def\eq@typeset@equation{%
  \nobreak
  \eq@params\eq@parshape
  \nointerlineskip\noindent
  \add@grp@label
  \eq@dump@box\unhbox\EQ@box\@@par
}
%    \end{macrocode}
%
% \section{Framing an equation}
% \begin{macro}{\eqframe}
% The \cs{eqframe} function is called in vertical mode
% with the reference point at the top left corner of the equation, including
% any allowance for \cs{fboxsep}.    Its arguments are the width
% and height of the equation body, plus fboxsep.
% \changes{v0.95}{2007/12/03}{Made \cs{eqframe} obey the key settings
%   for frame and framesep.}
%    \begin{macrocode}
\newcommand\eqframe[2]{%
  \begingroup
  \fboxrule=\eq@framewd\relax\fboxsep=\eq@framesep\relax
  \framebox{\z@rule\@height#2\kern#1}%
  \endgroup
}
%    \end{macrocode}
% \end{macro}
% The frame is not typeset at the correct horizontal position. Will
% fix later.
%    \begin{macrocode}
\def\eq@addframe{%
  \hbox to\z@{%
    \setlength\dim@a{\eq@framesep+\eq@framewd}%
    \kern-\dim@a
    \vbox to\z@{\kern-\dim@a
      \hbox{\eqframe{\eq@wdT}{\eq@vspan}}%
      \vss
    }%
    \hss
  }%
}
\def\eq@typeset@frame{%
  \if F\eq@frame\else
   % Tricky: put before \noindent, so it's not affected by glue in \leftskip
   \nobreak\nointerlineskip
   \vbox to\eq@firstht{\moveright\leftskip\hbox to\z@{\eq@addframe\hss}\vss}%
   \kern-\eq@firstht
  \fi
}
%    \end{macrocode}
%
% \section{Delimiter handling}
% The special handling of delimiters is rather complex, but
% everything is driven by two motives: to mark line breaks inside
% delimiters as less desirable than line breaks elsewhere, and to make it
% possible to break open left-right boxes so that line breaks between
% \cs{left} and \cs{right} delimiters are not absolutely
% prohibited.    To control the extent to which line breaks will be
% allowed inside delimiters, set \cs{eqbreakdepth} to the maximum
% nesting depth.    Depth 0 means never break inside delimiters.
%
% Note: \cs{eqbreakdepth} is not implemented as a \latex
% counter because changes done by \cs{setcounter} \etc  are always
% global.
%
% It would be natural to use grouping in the implementation
% \mdash  at an open delimiter, start a group and increase mathbin
% penalties; at a close delimiter, close the group.    But this gives us
% trouble in situations like the \env{array} environment, where a
% close delimiter might fall in a different cell of the \cs{halign}
% than the open delimiter.
% Ok then, here's what we want the various possibilities to
% expand to.    Note that \cs{right} and \cs{biggr} are
% being unnaturally applied to a naturally open-type delimiter.
% \begin{literalcode}
% ( -> \delimiter"4... \after@open
% \left( ->
%   \@@left \delimiter"4... \after@open
% \right( ->
%   \@@right \delimiter"4... \after@close
% \biggl( ->
%   \mathopen{\@@left \delimiter... \vrule...\@@right.}
%   \after@open
% \biggr( ->
%   \mathclose{\@@left \delimiter... \vrule...\@@right.}
%   \after@close
% \bigg\vert ->
%   \mathord{\@@left \delimiter... \vrule...\@@right.}
% \biggm\vert ->
%   \mathrel{\@@left \delimiter... \vrule...\@@right.}
% \end{literalcode}
%
% First save the primitive meanings of \cs{left} and
% \cs{right}.
%    \begin{macrocode}
\@saveprimitive\left\@@left
\@saveprimitive\right\@@right
%    \end{macrocode}
%
% The variable \cs{lr@level} is used by the first mathrel in
% an equation to tell whether it is at top level: yes? break and measure
% the LHS, no? keep going.
%    \begin{macrocode}
\newcount\lr@level
%    \end{macrocode}
%
% It would be nice to have better error checking here if the
% argument is not a delimiter symbol at all.
%
% Ah, a small problem when renaming commands. In the original version,
% |\delimiter| is hijacked in order to remove the |\after@bidir| (or
% open or close) instruction following the delimiter declaration.
%    \begin{macrocode}
\ExplSyntaxOn
\def\eq@left{%
  \@ifnext .{\eq@nullleft}{\begingroup \let\math_delimiter:NNnNn \eq@left@a}%
}
\def\eq@right{%
  \@ifnext .{\eq@nullright}{\begingroup \let \math_delimiter:NNnNn \eq@right@a}%
}
%    \end{macrocode}
% The arguments are: \arg1  delim symbol, \arg2 .
%    \begin{macrocode}
%\def\eq@left@a#1 #2{\endgroup\@@left\delimiter#1\space \after@open}
\def\eq@left@a#1#2#3#4#5#6{\endgroup
  \@@left \math_delimiter:NNnNn #1#2{#3}#4{#5}\after@open}
\def\eq@right@a#1#2#3#4#5#6{\endgroup
  \@@right \math_delimiter:NNnNn #1#2{#3}#4{#5}\after@close \ss@scan{#1#2{#3}#4{#5}}%
}
\ExplSyntaxOff
%    \end{macrocode}
% The null versions.
%    \begin{macrocode}
\def\eq@nullleft#1{\@@left#1\after@open}
\def\eq@nullright#1{\@@right#1\after@close}
%    \end{macrocode}
%
% Here is the normal operation of \cs{biggl}, for example.
% \begin{literalcode}
% \biggl ->\mathopen \bigg
% {\mathopen}
%
% \bigg #1->{\hbox {$\left #1\vbox to14.5\p@ {}\right .\n@space $}}
% #1<-(
% \end{literalcode}
% ^^AFor paren matching: )
% Like \cs{left}, \cs{biggl} coerces its delimiter to be of
% mathopen type even if its natural inclination is towards closing.
%
% The function \cs{delim@reset} makes delimiter characters
% work just about the same as they would in normal \latex .
%    \begin{macrocode}
\def\delim@reset{%
  \let\after@open\relax \let\after@close\relax
  \let\left\@@left \let\right\@@right
}
%    \end{macrocode}
% If the \pkg{amsmath} or \pkg{exscale} package is loaded, it
% will have defined \cs{bBigg@}; if not, the macros \cs{big} and
% variants will have hard-coded point sizes as inherited through the ages
% from \fn{plain.tex}.    In this case we can kluge a little by
% setting \cs{big@size} to \cs{p@}, so that our definition of
% \cs{bBigg@} will work equally well with the different multipliers.
%    \begin{macrocode}
\@ifundefined{bBigg@}{% not defined
  \let\big@size\p@
  \def\big{\bBigg@{8.5}}\def\Big{\bBigg@{11.5}}%
  \def\bigg{\bBigg@{14.5}}\def\Bigg{\bBigg@{17.5}}%
  \def\biggg{\bBigg@{20.5}}\def\Biggg{\bBigg@{23.5}}%
}{}
\def\bBigg@#1#2{%
  {\delim@reset
   \left#2%
   \vrule\@height#1\big@size\@width-\nulldelimiterspace
   \right.
  }%
}
%    \end{macrocode}
% .
%    \begin{macrocode}
\def\bigl#1{\mathopen\big{#1}\after@open}
\def\Bigl#1{\mathopen\Big{#1}\after@open}
\def\biggl#1{\mathopen\bigg{#1}\after@open}
\def\Biggl#1{\mathopen\Bigg{#1}\after@open}
\def\bigggl#1{\mathopen\biggg{#1}\after@open}
\def\Bigggl#1{\mathopen\Biggg{#1}\after@open}

\def\bigr#1{\mathclose\big{#1}\after@close}
\def\Bigr#1{\mathclose\Big{#1}\after@close}
\def\biggr#1{\mathclose\bigg{#1}\after@close}
\def\Biggr#1{\mathclose\Bigg{#1}\after@close}
\def\bigggr#1{\mathclose\biggg{#1}\after@close}
\def\Bigggr#1{\mathclose\Biggg{#1}\after@close}

%% No change needed, I think. [mjd,1998/12/04]
%%\def\bigm{\mathrel\big}
%%\def\Bigm{\mathrel\Big}
%%\def\biggm{\mathrel\bigg}
%%\def\Biggm{\mathrel\Bigg}
%%\def\bigggm{\mathrel\biggg}
%%\def\Bigggm{\mathrel\Biggg}
%    \end{macrocode}
%
%
% \begin{macro}{\m@@DeL} \begin{macro}{\d@@DeL}
% \begin{macro}{\m@@DeR} \begin{macro}{\d@@DeR}
% \begin{macro}{\m@@DeB} \begin{macro}{\d@@DeB}
% Original definition of \cs{m@DeL} from
% \pkg{flexisym} is as follows.    \cs{m@DeR} and
% \cs{m@DeB} are the same except for the math class number.
% \begin{literalcode}
% \def\m@DeL#1#2#3{%
%   \delimiter"4\@xp\delim@a\csname sd@#1#2#3\endcsname #1#2#3 }
% \end{literalcode}
%
% Save the existing meanings of \cs{m@De[LRB]}.
%
% Define display variants of DeL, DeR, DeB
%    \begin{macrocode}
\ExplSyntaxOn
\cs_set:Npn \math_dsym_DeL:Nn #1#2{\math_bsym_DeL:Nn #1{#2}\after@open}
\cs_set:Npn \math_dsym_DeR:Nn #1#2{\math_bsym_DeR:Nn #1{#2}\after@close}
\cs_set:Npn \math_dsym_DeB:Nn #1#2{\math_bsym_DeB:Nn #1{#2}\after@bidir}

%%%%%
%%%%%\let\m@@DeL\m@DeL \let\m@@DeR\m@DeR \let\m@@DeB\m@DeB
%%%%%\def\d@@DeL#1#2#3{%
%%%%%  \delimiter"4\@xp\delim@a\csname sd@#1#2#3\endcsname #1#2#3 \after@open
%%%%%}
%%%%%\def\d@@DeR#1#2#3{%
%%%%%  \delimiter"5\@xp\delim@a\csname sd@#1#2#3\endcsname #1#2#3 \after@close
%%%%%}
%%%%%\def\d@@DeB#1#2#3{%
%%%%%  \delimiter"0\@xp\delim@a\csname sd@#1#2#3\endcsname #1#2#3 \after@bidir
%%%%%}
%    \end{macrocode}
%%BRM: These weren't defined, but apparently should be.
% Are these the right values???
%    \begin{macrocode}
%%%%%%\let\m@@DeA\m@DeA\let\d@@DeA\m@DeA%
%    \end{macrocode}
% \end{macro}
% \end{macro}
% \end{macro}
% \end{macro}
% \end{macro}
% \end{macro}
%
% \begin{macro}{\after@open}
% \begin{macro}{\after@close}
% \begin{macro}{\after@bidir}
% \begin{macro}{\zero@bop}
% \begin{macro}{\bop@incr}
% \cs{after@open} and \cs{after@close} are carefully
% written to avoid the use of grouping and to run as fast as possible.
% \cs{zero@bop} is the value used for \cs{prebinoppenalty} at
% delimiter level 0, while \cs{bop@incr} is added for each level of
% nesting.    The standard values provide that breaks will be prohibited
% within delimiters below nesting level 2.
%    \begin{macrocode}
\let\after@bidir\@empty
\mathchardef\zero@bop=888 \relax
\mathchardef\bop@incr=4444 \relax
\def\after@open{%
  \global\advance\lr@level\@ne
  \prebinoppenalty\bop@incr \multiply\prebinoppenalty\lr@level
  \advance\prebinoppenalty\zero@bop
  \ifnum\eqbreakdepth<\lr@level
     \cs_set_eq:NN \math_sym_Bin:Nn \math_isym_Bin:Nn %%%%%%\let\m@Bin\m@@Bin
%    \end{macrocode}
% Inside delimiters, add some fillglue before binops so that a broken off
% portion will get thrown flush right.    Also shift it slightly
% further to the right to ensure that it clears the opening delimiter.
%    \begin{macrocode}
  \else
    \eq@binoffset=\eqbinoffset
    \advance\eq@binoffset\lr@level\eqdelimoffset plus1fill\relax
    \def\dt@fill@cancel{\hskip\z@ minus1fill\relax}%
  \fi
  \penalty\@M % BRM: discourage break after an open fence?
}
\def\after@close{%
  \global\advance\lr@level\m@ne
  \prebinoppenalty\bop@incr \multiply\prebinoppenalty\lr@level
  \advance\prebinoppenalty\zero@bop
  \ifnum\eqbreakdepth<\lr@level
  \else \cs_set_eq:NN \math_sym_Bin:Nn \math_dsym_Bin:Nn %%%%%%\let\m@Bin\d@@Bin
  \fi
%    \end{macrocode}
% When we get back to level 0, no delimiters, remove the stretch
% component of \cs{eqbinoffset}.
%    \begin{macrocode}
  \ifnum\lr@level<\@ne \eq@binoffset=\eqbinoffset\relax \fi
}

\ExplSyntaxOff

%    \end{macrocode}
% \end{macro}
% \end{macro}
% \end{macro}
% \end{macro}
% \end{macro}
%
%
% \begin{macro}{\subsup@flag}
% \begin{macro}{\ss@scan}
% \cs{ss@scan} is called after a \cs{right} delimiter and
% looks ahead for sub and superscript tokens.
% If sub and/or superscripts are present, we adjust the line-ending
% penalty to distinguish the various cases (sub, sup, or both).
% This facilitates the later work of excising the sub/sup box and
% reattaching it with proper shifting.
%
%%%%%%%%%%%%%%%%%%%%%%%%%%%%%%%%%%%%%%%%%%%%%%%%%%%%%%%%%%%%%%%%%%%%%%
% Sub/Superscript measurement
%%%%%%%%%%%%%%%%%%%%%%%%%%%%%%%%%%%%%%%%%%%%%%%%%%%%%%%%%%%%%%%%%%%%%%
% BRM: There's possibly a problem here.
% When \cs{ss@scan} gets invoked after a \cs{left}...\cs{right} pair in the LHS
% during \cs{eq@measure}, it produces an extra box (marked with \cs{penalty} 3);
% Apparently \cs{eq@repack} expects only one for the LHS.
% The end result is \cs{eq@wdL} => 0.0pt !!! (or at least very small)
%    \begin{macrocode}
\let\subsup@flag=\count@
\def\ss@delim@a@new#1#2#3#4#5{\xdef\right@delim@code{\number"#4#5}}
%    \end{macrocode}
% The argument of \cs{ss@scan} is an expanded form of a
% right-delimiter macro.
% We want to use the last three digits in the expansion
% to define \cs{right@delim@code}.
% The assignment to a temp register is just a way to scan away the
% leading digits that we don't care about.
%    \begin{macrocode}
\def\ss@scan#1{%
%    \end{macrocode}
% This part of the code.
%    \begin{macrocode}
  \begingroup
    \ss@delim@a@new #1%
  \endgroup
  \subsup@flag\@M \afterassignment\ss@scan@a \let\@let@token=}
\def\ss@scan@a{%
  \let\breqn@next\ss@scan@b
  \ifx\@let@token\sb \advance\subsup@flag\@ne\else
  \ifx\@let@token\@@subscript \advance\subsup@flag\@ne\else
  \ifx\@let@token\@@subscript@other \advance\subsup@flag\@ne\else
  \ifx\@let@token\sp \advance\subsup@flag\tw@\else
  \ifx\@let@token\@@superscript \advance\subsup@flag\tw@\else
  \ifx\@let@token\@@superscript@other \advance\subsup@flag\tw@\else
    \ss@finish
    \let\breqn@next\relax
  \fi\fi\fi\fi\fi\fi
  \breqn@next\@let@token
}
%    \end{macrocode}
%
%    \begin{macrocode}
\ExplSyntaxOn
\def\ss@scan@b#1#2{#1{%
%    \end{macrocode}
% hack! coff!
%    \begin{macrocode}
  %%%%%\let\m@Bin\m@@Bin  \let\m@Rel\m@@Rel
  \cs_set_eq:NN \math_sym_Bin:Nn \math_isym_Bin:Nn
  \cs_set_eq:NN \math_sym_Rel:Nn \math_isym_Rel:Nn
  #2}\afterassignment\ss@scan@a \let\@let@token=}%
\ExplSyntaxOff
%    \end{macrocode}
% We need to keep following glue from disappearing
% \mdash  \eg , a thickmuskip or medmuskip from a following mathrel or
% mathbin symbol.
%    \begin{macrocode}
\def\ss@finish{%
  \@@vadjust{\penalty\thr@@}%
  \penalty\right@delim@code \penalty-\subsup@flag \keep@glue
}
%    \end{macrocode}
% \end{macro}
% \end{macro}
%
%
% \begin{macro}{\eq@lrunpack}
% For \cs{eq@lrunpack} we need to break open a left-right box and
% reset it just in case it contains any more special breaks.    After
% it is unpacked the recursion of \cs{eq@repack} will continue,
% acting on the newly created lines.
%    \begin{macrocode}
\def\eq@lrunpack{\setbox\z@\lastbox
%    \end{macrocode}
% We remove the preceding glue item and deactivate
% baselineskip for the next line, otherwise we would end up with
% three items of glue (counting parskip) at this point instead of
% the single one expected by our recursive repacking
% procedure.
%    \begin{macrocode}
  \unskip \nointerlineskip
%    \end{macrocode}
% Then we open box 0, take the left-right box at the right end of
% it, and break that open.    If the line-ending penalty is greater than
% 10000, it means a sub and/or superscript is present on the right
% delimiter and the box containing them must be taken off first.
%    \begin{macrocode}
  \noindent\unhbox\z@ \unskip
  \subsup@flag-\lastpenalty \unpenalty
  \xdef\right@delim@code{\number\lastpenalty}%
  \unpenalty
  \ifnum\subsup@flag>\@M
    \advance\subsup@flag-\@M
    \setbox\tw@\lastbox
  \else \setbox\tw@\box\voidb@x
  \fi
  \setbox\z@\lastbox
  \ifvoid\tw@ \unhbox\z@
  \else \lrss@reattach % uses \subsup@flag, box\z@, box\tw@
  \fi
%    \end{macrocode}
% The reason for adding a null last line here is that the last
% line will contain parfillskip in addition to rightskip, and a final
% penalty of $10000$ instead of $-1000N$
% ($1\leq N\leq 9$), which would interfere with the usual
% processing.    Setting a null last line and discarding it dodges
% this complication.    The penalty value $-10001$ is a no-op case
% in the case statement of \cs{eq@repacka}.
%    \begin{macrocode}
  \penalty-\@Mi\z@rule\@@par
  \setbox\z@\lastbox \unskip\unpenalty
%%{\showboxbreadth\maxdimen\showboxdepth99\showlists}%
}
%    \end{macrocode}
% \end{macro}
%
%
% \begin{macro}{\lrss@reattach}
%
% Well, for a small self-contained computation, carefully
% hand-allocated dimens should be safe enough.    But let the
% maintainer beware!    This code cannot be arbitrarily transplanted
% or shaken up without regard to grouping and interaction with other
% hand-allocated dimens.
%    \begin{macrocode}
\dimendef\sub@depth=8 \dimendef\sup@base=6
\dimendef\prelim@sub@depth=4 \dimendef\prelim@sup@base=2
\def\sym@xheight{\fontdimen5\textfont\tw@}
\def\sup@base@one{\fontdimen13\textfont\tw@}
\def\sub@base@one{\fontdimen16\textfont\tw@}
\def\sub@base@two{\fontdimen17\textfont\tw@}
%    \end{macrocode}
% Note that only \cs{sup@drop} and \cs{sub@drop} come from
% the next smaller math style.
%    \begin{macrocode}
\def\sup@drop{\fontdimen18\scriptfont\tw@}
\def\sub@drop{\fontdimen19\scriptfont\tw@}
%    \end{macrocode}
% Provide a mnemonic name for the math axis fontdimen, if it's not
% already defined.
%    \begin{macrocode}
\providecommand{\mathaxis}{\fontdimen22\textfont\tw@}
%    \end{macrocode}
%
% Assumes box 2 contains the sub/sup and box 0 contains the left-right
% box.    This is just a repeat of the algorithm in \fn{tex.web},
% with some modest simplifications from knowing that this is only going to
% be called at top level in a displayed equation, thus always mathstyle =
% uncramped displaystyle.
%    \begin{macrocode}
\def\lrss@reattach{%
  \begingroup
  % "The TeXbook" Appendix G step 18:
  \setlength\prelim@sup@base{\ht\z@-\sup@drop}%
  \setlength\prelim@sub@depth{\dp\z@ +\sub@drop}%
  \unhbox\z@
  \ifcase\subsup@flag      % case 0: this can't happen
  \or \lr@subscript   % case 1: subscript only
  \or \lr@superscript % case 2: superscript only
  \else \lr@subsup    % case 3: sub and superscript both
  \fi
  \endgroup
}
%    \end{macrocode}
%    \begin{macrocode}
\def\lr@subscript{%
  \sub@depth\sub@base@one
  \ifdim\prelim@sub@depth>\sub@depth \sub@depth\prelim@sub@depth\fi
  \setlength\dim@a{\ht\tw@  -.8\sym@xheight}%
  \ifdim\dim@a>\sub@depth \sub@depth=\dim@a \fi
  \twang@adjust\sub@depth
  \lower\sub@depth\box\tw@
}
%    \end{macrocode}
%
%    \begin{macrocode}
\def\lr@superscript{%
  \sup@base\sup@base@one
  \ifdim\prelim@sup@base>\sup@base \sup@base\prelim@sup@base\fi
  \setlength\dim@a{\dp\tw@ -.25\sym@xheight}%
  \ifdim\dim@a>\sup@base \sup@base\dim@a \fi
  \twang@adjust\sup@base
  \raise\sup@base\box\tw@
}
%    \end{macrocode}
%
%    \begin{macrocode}
\def\lr@subsup{%
  \sub@depth\sub@base@two
  \ifdim\prelim@sub@depth>\sub@depth \sub@depth\prelim@sub@depth \fi
  \twang@adjust\sub@depth
  \lower\sub@depth\box\tw@
}
%    \end{macrocode}
%
% For delimiters that curve top and bottom, the twang factor allows
% horizontal shifting of the sub and superscripts so they don't
% fall too far away (or too close for that matter).    This is
% accomplished by arranging for (\eg ) \verb"\right\rangle" to leave
% a penalty $N$ in the math list before the subsup penalty that triggers
% \cs{lrss@reattach}, where $N$ is the mathcode of
% \cs{rangle} (ignoring \dquoted{small} variant).
%    \begin{macrocode}
\def\twang@adjust#1{%
  \begingroup
    \@ifundefined{twang@\right@delim@code}{}{%
      \setlength\dim@d{#1-\mathaxis}%
      % put an upper limit on the adjustment
      \ifdim\dim@d>1em \dim@d 1em \fi
      \kern\csname twang@\right@delim@code\endcsname\dim@d
    }%
  \endgroup
}
%    \end{macrocode}
% The method used to apply a \dquoted{twang} adjustment is just an
% approximate solution to a complicated problem.
% We make the following assumptions that hold true, approximately,
% for the most common kinds of delimiters:
% \begin{enumerate}
% \item
% The right delimiter is symmetrical top to bottom.
%
%
% \item There is an upper limit on the size of the adjustment.
%
%
% \item When we have a superscript, the amount of left-skew that we
% want to apply is linearly proportional to the distance of the bottom
% left corner of the superscript from the math axis, with the ratio
% depending on the shape of the delimiter symbol.
%
%
% \end{enumerate}
% .
% By symmetry, Assumption 3 is true also for subscripts (upper left
% corner).
% Assumption 2 is more obviously true for parens and braces, where the
% largest super-extended versions consist of truly vertical parts with
% slight bending on the ends, than it is for a \cs{rangle}.
% But suppose for the sake of expediency that it is
% approximately true for rangle symbols also.
%
%
% Here are some passable twang factors for the most common types of
% delimiters in \fn{cmex10}, as determined by rough measurements from
% magnified printouts.
% \begin{literalcode}
%   vert bar, double vert:  0
%          square bracket: -.1
%             curly brace: -.25
%             parenthesis: -.33
%                  rangle: -.4
% \end{literalcode}
% Let's provide a non-private command for changing the twang factor of
% a given symbol.
%    \begin{macrocode}
\newcommand{\DeclareTwang}[2]{%
  \ifcat.\@nx#1\begingroup
    \lccode`\~=`#1\lowercase{\endgroup \DeclareTwang{~}}{#2}%
  \else
    \@xp\decl@twang#1?\@nil{#2}%
  \fi
}
%    \end{macrocode}
% Note that this is dependent on a fixed interpretation of the
% mathgroup number \arg4 .
%    \begin{macrocode}
\def\decl@twang#1#2#3#4#5#6#7\@nil#8{%
  \@namedef{twang@\number"#4#5#6}{#8}%
}
\DeclareTwang{\rangle}{-.4}
\DeclareTwang{)}{-.33}
\DeclareTwang{\rbrace}{-.25}
%    \end{macrocode}
% \end{macro}
%
%
%
% \section{Series of expressions}
% The \env{dseries} environment is for a display
% containing a series of expressions of the form \quoted{A, B} or \quoted{A and
% B} or \quoted{A, B, and C} and so on.    Typically the expressions
% are separated by a double quad of space.    If the expressions in a
% series don't all fit in a single line, they are continued onto extra
% lines in a ragged-center format.
%    \begin{macrocode}
\newenvironment{dseries}{\let\eq@hasNumber\@True \breqn@optarg\@dseries{}}{}%
\def\enddseries#1{\check@punct@or@qed}%
%    \end{macrocode}
%
% And the unnumbered version of same.
%    \begin{macrocode}
\newenvironment{dseries*}{\let\eq@hasNumber\@False \breqn@optarg\@dseries{}}{}%
\@namedef{enddseries*}#1{\check@punct@or@qed}%
\@namedef{end@dseries*}{\end@dseries}%
\def\@dseries[#1]{%
%    \end{macrocode}
% Turn off the special breaking behavior of mathrels \etc  for math
% formulas embedded in a \env{dseries} environment.
%
%BRM: DS Expermient: Use alternative display setup.
%    \begin{macrocode}
%  \def\display@setup{\displaystyle}%
  \let\display@setup\dseries@display@setup
  % Question: should this be the default for dseries???
%  \let\eq@centerlines\@True
  \global\eq@wdCond\z@
%    \end{macrocode}
% BRM: use special layout for dseries
%    \begin{macrocode}
%  \@dmath[#1]%
  \@dmath[layout={M},#1]%
  \mathsurround\z@\@@math \penalty\@Mi
  \let\endmath\ends@math
  \def\premath{%
%    \end{macrocode}
% BRM: Tricky to cleanup space OR add space ONLY BETWEEN math!
%    \begin{macrocode}
    \ifdim\lastskip<.3em \unskip
    \else\ifnum\lastpenalty<\@M \dquad\fi\fi
}%
%    \end{macrocode}
%BRM: Tricky; if a subformula breaks, we'd like to start the next on new line!
%    \begin{macrocode}
  \def\postmath{\unpenalty\eq@addpunct \penalty\intermath@penalty \dquad \@ignoretrue}%
\ignorespaces
}
\def\end@dseries{%
  \unskip\unpenalty
  \@@endmath \mathsurround\z@ \end@dmath
}
%    \end{macrocode}
% BRM: Try this layout for dseries: Essentially layout i, but w/o
% limit to 1 line.  And no fallback!
%    \begin{macrocode}
\def\eq@try@layout@M{%
  \edef\@parshape{\parshape 1 0pt \the\eq@linewidth\relax}%
  \eq@trial@b{M}{}%
}
%    \end{macrocode}
% BRM: Tricky to get right value here.
% Prefer breaks between formula if we've got to break at all.
%    \begin{macrocode}
%\def\intermath@penalty{-201}%
\def\intermath@penalty{-221}%
%    \end{macrocode}
% BRM: A bit tighter than it was ( 1em minus.25em )
%    \begin{macrocode}
%\newcommand\dquad{\hskip0.4em}
\newcommand\dquad{\hskip0.6em minus.3em}
\newcommand\premath{}\newcommand\postmath{}
%    \end{macrocode}
%
% Change the \env{math} environment to add
% \cs{premath} and \cs{postmath}.    They are no-ops except
% inside a \env{dseries} environment.
%
%%%%%%%%%%%%%%%%%%%%%%%%%%%%%%%%%%%%%%%%%%%%%%%%%%%%%%%%%%%%%%%%%%%%%%
% Redefinition of math environment to take advantage of dseries env.
%    \begin{macrocode}
\renewenvironment{math}{%
  \leavevmode \premath
  \ifmmode\@badmath\else\@@math\fi
}{%
  \ifmmode\@@endmath\else\@badmath\fi
}
\def\ends@math#1{\check@punct@or@qed}
\def\end@math{%
  \ifmmode\@@endmath\else\@badmath\fi
  \postmath
}
%    \end{macrocode}
%
%
%
%
% \section{Equation groups}
% For many equation groups the strategy is easy: just center each
% equation individually following the normal rules for a single
% equation.    In some groups, each equation gets its own number; in
% others, a single number applies to the whole group (and may need to be
% vertically centered on the height of the group).    In still other
% groups, the equations share a parent number but get individual equation
% numbers consisting of parent number plus a letter.
%
% If the main relation symbols in a group of equations are to be
% aligned, then the final alignment computations cannot be done until the
% end of the group \mdash  \ie , the horizontal positioning of the first
% $n - 1$ equations cannot be done immediately.    Yet because of
% the automatic line breaking, we cannot calculate an initial value of
% RHS-max over the whole group unless we do a trial run on each equation
% first to find an RHS-max for that equation.    Once we know RHS-group-max
% and LHS-group-max we must redo the trial set of each equation because
% they may affect the line breaks.    If the second trial for an
% equation fails (one of its lines exceeds the available width), but
% the first one succeeded, fall back to the first trial, \ie  let that
% equation fall out of alignment with the rest of the group.
%
%
% All right then, here is the general idea of the whole algorithm for
% group alignment.
% To start with, ignore the possibility of equation numbers so that
% our equation group has the form:
% \begin{literalcode}
% LHS[1] RHS[1,1] RHS[1,2] ... RHS[1,n[1]]
% LHS[2] RHS[2,1] RHS[2,2] ... RHS[2,n[2]]
%   ...
% LHS[3] RHS[3,1] RHS[3,2] ... RHS[3,n[3]]
% \end{literalcode}
% The number of RHS's might not be the same for all of the
% equations.
% First, accumulate all of the equation contents in a queue, checking
% along the way to find the maximum width of all the LHS's and the maximum
% width of all the RHS's.
% Call these widths maxwd\_L and maxwd\_R.
% Clearly if maxwd\_L + maxwd\_R is less than or equal to the available
% equation width then aligning all of the equations is going to be simple.
%
%
% Otherwise we are going to have to break at least one of the RHS's
% and/or at least one of the LHS's.
% The first thing to try is using maxwd\_L for the LHS's and breaking
% all the RHS's as needed to fit in the remaining space.
% However, this might be a really dumb strategy if one or more of the
% LHS's is extraordinarily wide.
% So before trying that we check whether maxwd\_L exceeds some
% threshold width beyond which it would be unsensible not to break the LHS.
% Such as, max(one-third of the available width; six ems), or
% something like that.
% Or how about this?
% Compare the average LHS width and RHS width and divide up the available
% width in the same ratio for line breaking purposes.
%
%
% BRM: Fairly broad changes; it mostly didn't work before (for me).
%
% \begin{description}
% \item[\cs{begin}\csarg{dgroup} produces a `numbered' group]
%   The number is the next equation number.
%   There are 2 cases:
% \begin{itemize}
% \item If ANY contained equations are numbered (|\begin{dmath}|),
%       then they will be subnumbered: eg 1.1a
%       and the group number is not otherwise displayed.
% \item If ALL contained equations are unnumbered (|\begin{dmath*}|)
%       then the group, as a whole, gets a number displayed,
%       using the same number placement as for equations.
% \end{itemize}
% \item[\cs{begin}\csarg{dgroup*} produces an unnumbered group.]
%    Contained equations are numbered, or not, as normal.
%    But note that in the mixed case, it's too late to
%    force the unnumbered eqns to \cs{retry@with@number}
%    We'll just do a simple check of dimensions, after the fact,
%    and force a shiftnumber if we're stuck.
%
% NOTE: Does this work for dseries, as well? (alignment?)
%
% NOTE: Does \cs{label} attach to the expected thing?
%
% \item[For number placement] We use shiftnumber placement on ALL equations
%    if ANY equations need it, or if an unnumbered equation is too
%    wide to be aligned, given that the group or other eqns are numbered.
%    [does this latter case interract with the chosen alignment?]
%
% \item[For Alignment]
%   As currently coded, it tries to align on relations, by default.
%   If LHS's are not all present, or too long, it switches to left-justify.
%   Maybe there are other cases that should switch?
%   Should there be a case for centered?
%
% NOTE: Should there be some options to choose alignment?
% \end{description}
%
% \begin{macro}{\eq@group}
% \begin{macro}{\GRP@top}
%
%    \begin{macrocode}
\let\eq@group\@False
\let\grp@shiftnumber\@False
\let\grp@hasNumber\@False
\let\grp@eqs@numbered\@False
\let\grp@aligned\@True
%    \end{macrocode}
% \end{macro}
% \end{macro}
%
%
% Definition of the \env{dgroup} environment.
%    \begin{macrocode}
\newenvironment{dgroup}{%
  \@dgroup@start@hook
  \let\grp@hasNumber\@True\breqn@optarg\@dgroup{}%
}{%
  \end@dgroup
}
%    \end{macrocode}
% And the.
%    \begin{macrocode}
\newtoks\GRP@queue
\newenvironment{dgroup*}{%
  \let\grp@hasNumber\@False\breqn@optarg\@dgroup{}%
}{%
  \end@dgroup
}
\def\@dgroup[#1]{%
%<trace>  \breqn@debugmsg{=== DGROUP ==================================================}%
  \let\eq@group\@True \global\let\eq@GRP@first@dmath\@True
  \global\GRP@queue\@emptytoks \global\setbox\GRP@box\box\voidb@x
  \global\let\GRP@label\@empty
  \global\grp@wdL\z@\global\grp@wdR\z@\global\grp@wdT\z@
  \global\grp@linewidth\z@\global\grp@wdNum\z@
  \global\let\grp@eqs@numbered\@False
  \global\let\grp@aligned\@True
  \global\let\grp@shiftnumber\@False
  \eq@prelim
  \setkeys{breqn}{#1}%
  \if\grp@hasNumber \grp@setnumber \fi
}
\def\end@dgroup{%
  \EQ@displayinfo \grp@finish
  \if\grp@hasNumber\grp@resetnumber\fi
}
%    \end{macrocode}
% If the \pkg{amsmath} package is not loaded the parentequation
% counter will not be defined.
%    \begin{macrocode}
\@ifundefined{c@parentequation}{\newcounter{parentequation}}{}
%    \end{macrocode}
% Init.
%    \begin{macrocode}
\global\let\GRP@label\@empty
\def\add@grp@label{%
  \ifx\@empty\GRP@label
  \else \GRP@label \global\let\GRP@label\@empty
  \fi
}
%    \end{macrocode}
% Before sending down the `equation' counter to the subordinate level,
% set the current number in \cs{EQ@numbox}.    The
% \cs{eq@setnumber} function does everything we need here.    If
% the child equations are unnumbered, \cs{EQ@numbox} will retain the
% group number at the end of the group.
%    \begin{macrocode}
\def\grp@setnumber{%
  \global\let\GRP@label\next@label \global\let\next@label\@empty
  % Trick \eq@setnumber to doing our work for us.
  \let\eq@hasNumber\@True
  \eq@setnumber
%    \end{macrocode}
% Define \cn{theparentequation} equivalent to current
% \cn{theequation}.    \cn{edef} is necessary to expand the
% current value of the equation counter.    This might in rare cases
% cause something to blow up, in which case the user needs to add
% \cn{protect}.
%    \begin{macrocode}
  \global\sbox\GRP@numbox{\unhbox\EQ@numbox}%
  \grp@wdNum\eq@wdNum
  \let\eq@hasNumber\@False
  \let\eq@number\@empty
  \eq@wdNum\z@
%
  \protected@edef\theparentequation{\theequation}%
  \setcounter{parentequation}{\value{equation}}%
%    \end{macrocode}
% And set the equation counter to 0, so that the normal incrementing
% processes will produce the desired results if the child equations are
% numbered.
%    \begin{macrocode}
  \setcounter{equation}{0}%
  \def\theequation{\theparentequation\alph{equation}}%
%<trace>  \breqn@debugmsg{Group Number \theequation}%
}
%    \end{macrocode}
% At the end of a group, need to reset the equation counter.
%    \begin{macrocode}
\def\grp@resetnumber{%
  \setcounter{equation}{\value{parentequation}}%
}
\newbox\GRP@box
\newbox\GRP@wholebox
%    \end{macrocode}
% Save data for this equation in the group
% \begin{itemize}
% \item push the trial data onto end of \cs{GRP@queue}.
% \item push an hbox onto the front of \cs{GRP@box} containing:
%   \cs{EQ@box}, \cs{EQ@copy}, \cs{penalty} 1 and \cs{EQ@numbox}.
% \end{itemize}
% \begin{macro}{\grp@push}
%
% For putting the equation on a queue.
%    \begin{macrocode}
\def\grp@push{%
  \global\GRP@queue\@xp\@xp\@xp{\@xp\the\@xp\GRP@queue
    \@xp\@elt\@xp{\EQ@trial}%
  }%
  \global\setbox\GRP@box\vbox{%
    \hbox{\box\EQ@box\box\EQ@copy\penalty\@ne\copy\EQ@numbox}%
    \unvbox\GRP@box
  }%
  \EQ@trial
  \if\eq@isIntertext\else
    \ifdim\eq@wdL>\grp@wdL \global\grp@wdL\eq@wdL \fi
    \ifdim\eq@wdT>\grp@wdT \global\grp@wdT\eq@wdT \fi
    \setlength\dim@a{\eq@wdT-\eq@wdL}%
    \ifdim\dim@a>\grp@wdR \global\grp@wdR\dim@a \fi
    \ifdim\eq@linewidth>\grp@linewidth \global\grp@linewidth\eq@linewidth\fi
    \if\eq@hasNumber
       \global\let\grp@eqs@numbered\@True
       \ifdim\eq@wdNum>\grp@wdNum\global\grp@wdNum\eq@wdNum\fi
    \fi
    \if\EQ@hasLHS\else\global\let\grp@aligned\@False\fi
    \if D\eq@layout \global\let\grp@aligned\@False\fi % Layout D (usually) puts rel on 2nd line.
    \if\eq@shiftnumber\global\let\grp@shiftnumber\@True\fi % One eq shifted forces all.
  \fi
}
%    \end{macrocode}
% \end{macro}
% \begin{macro}{\grp@finish}
%
% Set accumulated equations from a \env{dgroup} environment.
%
% BRM: Questionable patch!!
% When processing the \cs{GRP@queue}, put it into a \cs{vbox}, then \cs{unvbox} it.
% This since there's a bizarre problem when the \cs{output} routine
% gets invoked at an inopportune moment: All the not-yet-processed
% \cs{GRP@queue} ends up in the \cs{@freelist} and bad name clashes happen.
% Of course, it could be due to some other problem entirely!!!
%    \begin{macrocode}
\def\grp@finish{%
%  \debug@box\GRP@box
%  \breqn@debugmsg{\GRP@queue: \the\GRP@queue}%
%    \end{macrocode}
% == Now that we know the collective measurements, make final decision
% about alignment \& shifting.  Check if alignment is still possible
%    \begin{macrocode}
  \setlength\dim@a{\grp@wdL+\grp@wdR-4em}% Allowance for shrink?
  \if\grp@aligned
    \ifdim\dim@a>\grp@linewidth
      \global\let\grp@aligned\@False
    \fi
  \fi
%    \end{macrocode}
% If we're adding an unshifted group number that equations didn't know
% about, re-check shifting
%    \begin{macrocode}
  \addtolength\dim@a{\grp@wdNum }% Effective length
  \if\grp@shiftnumber
  \else
    \if\@And{\grp@hasNumber}{\@Not\grp@eqs@numbered}
      \ifdim\dim@a>\grp@linewidth
        \global\let\grp@shiftnumber\@True
      \fi
    \fi
  \fi
%    \end{macrocode}
% If we can still align, total width is sum of maximum LHS \& RHS
%    \begin{macrocode}
  \if\grp@aligned
     \global\grp@wdT\grp@wdL
     \global\advance\grp@wdT\grp@wdR
  \fi
%<*trace>
  \breqn@debugmsg{======= DGROUP Formatting
   \MessageBreak==== \grp@wdL=\the\grp@wdL, \grp@wdR=\the\grp@wdR
   \MessageBreak==== Shift Number=\theb@@le\grp@shiftnumber, Eqns. numbered=\theb@@le\grp@eqs@numbered
   \MessageBreak==== Aligned=\theb@@le\grp@aligned
   \MessageBreak==== \grp@wdNum=\the\grp@wdNum}%
%</trace>
%    \end{macrocode}
% BRM: Originally this stuff was dumped directly, without capturing it
% in a \cs{vbox}
%    \begin{macrocode}
  \setbox\GRP@wholebox\vbox{%
    \let\@elt\eqgrp@elt
    \the\GRP@queue
  }%
%    \end{macrocode}
% If we're placing a group number (not individual eqn numbers)
% NOTE: For now, just code up LM number
% NOTE: Come back and handle other cases.
% NOTE: Vertical spacing is off, perhaps because of inter eqn. glue
%
% A bit of a hack to get the top spacing correct. Fix this logic
% properly some day.  Also, we do the calculation in a group for
% maximum safety.
%    \begin{macrocode}
  \global\let\eq@GRP@first@dmath\@True
  \begingroup
  \dmath@first@leftskip
  \eq@topspace{\vskip\parskip}%
  \endgroup
  \if\@And{\grp@hasNumber}{\@Not{\grp@eqs@numbered}}%
%    \eq@topspace{\vskip\parskip}%
    \if\grp@shiftnumber
      \copy\GRP@numbox \penalty\@M
      \kern\eqlineskip
    \else
      \setlength\dim@a{%
        (\ht\GRP@wholebox+\dp\GRP@wholebox+\ht\GRP@numbox-\dp\GRP@numbox)/2}%
      \setlength\skip@c{\dim@a-\ht\GRP@numbox}%
      \vglue\skip@c% NON discardable
      \copy\GRP@numbox \penalty\@M
%<*trace>
\breqn@debugmsg{GROUP NUMBER: preskip:\the\skip@c,  postkern:\the\dim@a, height:\the\ht\GRP@wholebox,
  \MessageBreak==== box height:\the\ht\GRP@numbox, box depth:\the\dp\GRP@numbox}%
%</trace>
      \kern-\dim@a
      \kern-\abovedisplayskip % To cancel the topspace above the first eqn.
    \fi
  \fi
%<*trace>
%\debug@box\GRP@wholebox
%</trace>
  \unvbox\GRP@wholebox
  \let\@elt\relax
%    \end{macrocode}
% We'd need to handle shifted, right number here, too!!!
%    \begin{macrocode}
  \eq@botspace % not needed unless bottom number?
}
%    \end{macrocode}
% \end{macro}
%
% \begin{macro}{\eqgrp@elt}
%
% Mission is to typeset the next equation from the group queue.
%
% The arg is an \cs{EQ@trial}
%    \begin{macrocode}
\def\eqgrp@elt#1{%
  \global\setbox\GRP@box\vbox{%
    \unvbox\GRP@box
    \setbox\z@\lastbox
    \setbox\tw@\hbox{\unhbox\z@
      \ifnum\lastpenalty=\@ne
      \else
        \global\setbox\EQ@numbox\lastbox
      \fi
      \unpenalty
      \global\setbox\EQ@copy\lastbox
      \global\setbox\EQ@box\lastbox
    }%
  }%
  \begingroup \let\eq@botspace\relax
  #1%
  \if\eq@isIntertext
    \vskip\belowdisplayskip
    \unvbox\EQ@copy
  \else
    \grp@override
    \eq@finish
  \fi
  \endgroup
}
%    \end{macrocode}
% \end{macro}
% Override the \cs{eq@trial} data as needed for this equation in this group
% NOTE: w/ numbering variations (see above), we may need to tell
%  \cs{eq@finish} to allocate space for a number, but not actually have one
%    \begin{macrocode}
\def\grp@override{%
%    \end{macrocode}
% For aligned (possibly becomes an option?)
% For now ASSUMING we started out as CLM!!!
%    \begin{macrocode}
  \def\eqindent{I}%
%    \end{macrocode}
% compute nominal left for centering the group
%    \begin{macrocode}
  \setlength\dim@a{(\grp@linewidth-\grp@wdT)/2}%
%    \end{macrocode}
% Make sure L+R not too wide; should already have unset alignment
%    \begin{macrocode}
  \ifdim\dim@a<\z@\dim@a=\z@\fi
  \dim@b\if L\eqnumside\grp@wdNum\else\z@\fi
%    \end{macrocode}
% make sure room for number on left, if needed.
%    \begin{macrocode}
  \if\grp@shiftnumber\else
    \ifdim\dim@b>\dim@a\dim@a\dim@b\fi
  \fi
  \if\grp@aligned
    \addtolength\dim@a{\grp@wdL-\eq@wdL}%
  \fi
  \mathindent\dim@a
  \ifdim\dim@b>\dim@a
    \let\eq@shiftnumber\@True
  \fi
%    \end{macrocode}
% Could set |\def\eqnumplace{T}| (or even (m) if indentation is enough).
%
% NOTE: Work out how this should interact with the various formats!!!
% NOTE: should recognize the case where the LHS's are a bit Wild,
%  and then do simple left align (not on relation)
%    \begin{macrocode}
}
%    \end{macrocode}
%
%
%
% \section{The \env{darray} environment}
% There are two potential applications for darray.    One
% is like eqnarray where the natural structure of the material crosses the
% table cell boundaries, and math operator spacing needs to be preserved
% across cell boundaries.    And there is also the feature of
% attaching an equation number to each row.    The other application
% is like a regular array but with automatic displaystyle math in each
% cell and better interline spacing to accommodate outsize cell
% contents.    In this case it is difficult to keep the vert ruling
% capabilities of the standard \env{array} environment without
% redoing the implementation along the lines of Arseneau's
% \pkg{tabls} package.    Because the vert ruling feature is at
% cross purposes with the feature of allowing interline stretch and page
% breaks within a multiline array of equations, the \env{darray}
% environment is targeted primarily as an alternative to
% \env{eqnarray}, and does not support vertical ruling.
%
% Overall strategy for \env{darray} is to use
% \cs{halign} for the body.    In the case of a group, use a
% single halign for the whole group!
% \begin{aside}
% What about intertext?
% \end{aside}
%
% That's the most reliable way
% to get accurate column widths.    Don't spread the halign to the
% column width, just use the natural width.    Then, if we repack the
% contents of the halign into \cs{EQ@box} and \cs{EQ@copy}, as
% done for dmath, and twiddle a bit with the widths of the first and last
% cell in each row, we can use the same algorithms for centering and
% equation number placement as dmath!    As well as handling footnotes
% and vadjust objects the same way.
%
% We can't just use \cs{arraycolsep} for \env{darray}, if
% we want to be able to change it without screwing up interior arrays.
% So let's make a new colsep variable.    The initial value is
% \quoted{2em, but let it shrink if necessary}.
%    \begin{macrocode}
\newskip\darraycolsep \darraycolsep 20pt plus1fil minus12pt
%    \end{macrocode}
% Let's make a nice big default setup with eighteen columns, split up
% into six sets of lcr like \env{eqnarray}.
%    \begin{macrocode}
\newcount\cur@row \newcount\cur@col
\def\@tempa#1#2#3{%
  \cur@col#1 \hfil
  \setbox\z@\hbox{$\displaystyle####\m@th$}\@nx\col@box
  \tabskip\z@skip
  &\cur@col#2 \hfil
  \setbox\z@\hbox{$\displaystyle\mathord{}####\mathord{}\m@th$}\@nx\col@box
  \hfil
  &\cur@col#3 \setbox\z@\hbox{$\displaystyle####\m@th$}\@nx\col@box
  \hfil\tabskip\darraycolsep
}
\xdef\darray@preamble{%
  \@tempa 123&\@tempa 456&\@tempa 789%
  &\@tempa{10}{11}{12}&\@tempa{13}{14}{15}&\@tempa{16}{17}{18}%
  \cr
}
\@ifundefined{Mathstrut@}{\let\Mathstrut@\strut}{}
\def\darray@cr{\Mathstrut@\cr}
\def\col@box{%
%<*trace>
%\breqn@debugmsg{Col \number\cur@row,\number\cur@col: \the\wd\z@\space x \the\ht\z@+\the\dp\z@}%
%</trace>
  \unhbox\z@
}
\newenvironment{darray}{\breqn@optarg\@darray{}}{}
\def\@darray[#1]{%
%<trace>  \breqn@debugmsg{=== DARRAY ==================================================}%
  \if\eq@group\else\eq@prelim\fi
%    \end{macrocode}
% Init the halign preamble to empty, then unless the \quoted{cols} key is
% used to provide a non-null preamble just use the
% default darray preamble which is a multiple lcr.
%    \begin{macrocode}
  \global\let\@preamble\@empty
  \setkeys{breqn}{#1}%
  \the\eqstyle \eq@setnumber
  \ifx\@preamble\@empty \global\let\@preamble\darray@preamble \fi
  \check@mathfonts
  % \let\check@mathfonts\relax % tempting, but too risky
  \@xp\let\csname\string\ \endcsname\darray@cr
  \setbox\z@\vbox\bgroup
  \everycr{\noalign{\global\advance\cur@row\@ne}}%
  \tabskip\z@skip \cur@col\z@
  \global\cur@row\z@
  \penalty\@ne % flag for \dar@repack
  \halign\@xp\bgroup\@preamble
}
%    \end{macrocode}
% Assimilate following punctuation.
%    \begin{macrocode}
\def\enddarray#1{\check@punct@or@qed}
\def\end@darray{%
  \ifvmode\else \eq@addpunct \Mathstrut@\fi\crcr \egroup
  \dar@capture
  \egroup
}
%    \end{macrocode}
%
% The \cs{dar@capture} function steps back through the
% list of row boxes and grinds them up in the best possible way.
%    \begin{macrocode}
\def\dar@capture{%
%% \showboxbreadth\maxdimen\showboxdepth99\showlists
  \eq@wdL\z@ \eq@wdRmax\z@
  \dar@repack
}
%    \end{macrocode}
%
% The \cs{dar@repack} function is a variation of
% \cs{eq@repack}.
%    \begin{macrocode}
\def\dar@repack{%
  \unpenalty
  \setbox\tw@\lastbox
%\batchmode{\showboxbreadth\maxdimen\showboxdepth99\showbox\tw@}\errorstopmode
  \global\setbox\EQ@box\hbox{%
    \hbox{\unhcopy\tw@\unskip}\penalty-\@M \unhbox\EQ@box}%
  \global\setbox\EQ@copy\hbox{%
    \hbox{\unhbox\tw@\unskip}\penalty-\@M \unhbox\EQ@copy}%
  \unskip
  \ifcase\lastpenalty \else\@xp\@gobble\fi
  \dar@repack
}
%    \end{macrocode}
%
%
%
%
% \section{Miscellaneous}
% The \cs{condition} command.    With
% the star form, set the argument in math mode instead of text mode.
% In a series of conditions, use less space between members of the
% series than between the conditions and the main equation body.
%
% WSPR: tidied/fixed things up as it made sense to me but might have
% broken something else!
%    \begin{macrocode}
\newskip\conditionsep \conditionsep=10pt minus5pt%
\newcommand{\conditionpunct}{,}
%    \end{macrocode}
% \begin{macro}{\condition}
%    \begin{macrocode}
\newcommand\condition{%
  \begingroup\@tempswatrue
    \breqn@ifstar{\@tempswafalse \condition@a}{\condition@a}}
%    \end{macrocode}
% \end{macro}
% \begin{macro}{\condition@a}
%    \begin{macrocode}
\newcommand\condition@a[2][\conditionpunct]{%
  \unpenalty\unskip\unpenalty\unskip % BRM Added
  \hbox{#1}%
  \penalty -201\relax\hbox{}% Penalty to allow breaks here.
  \hskip\conditionsep
  \setbox\z@\if@tempswa\hbox{#2}\else\hbox{$\textmath@setup #2$}\fi
%    \end{macrocode}
% BRM's layout is achieved with this line commented out but it has the nasty side-effect of shifting the equation number to the next line:
%    \begin{macrocode}
%  \global\eq@wdCond\wd\z@
  \usebox\z@
  \endgroup}
%    \end{macrocode}
% \end{macro}
%
% The \env{dsuspend} environment.    First the old one that didn't work.
%    \begin{macrocode}
\newenvironment{XXXXdsuspend}{%
  \global\setbox\EQ@box\vbox\bgroup \@parboxrestore
%    \end{macrocode}
% If we are inside a list environment, \cs{displayindent} and
% \cs{displaywidth} give us \cs{@totalleftmargin} and
% \cs{linewidth}.
%    \begin{macrocode}
    \parshape 1 \displayindent \displaywidth\relax
    \hsize=\columnwidth \noindent\ignorespaces
}{%
  \par\egroup
%    \end{macrocode}
% Let's try giving \cs{EQ@box} the correct height for the first
% line and \cs{EQ@copy} the depth of the last line.
%    \begin{macrocode}
  \global\setbox\GRP@box\vbox{%
    \vbox{\copy\EQ@box\vtop{\unvbox\EQ@box}}%
    \unvbox\GRP@box
  }%
%    \end{macrocode}
% Need to add a dummy element to \cs{GRP@queue}.
%    \begin{macrocode}
  \global\GRP@queue\@xp{\the\GRP@queue
    \@elt{\gdef\EQ@trial{}}%
  }%
}
%    \end{macrocode}
% And then the one that does work.
%    \begin{macrocode}
\newenvironment{dsuspend}{%
  \global\setbox\EQ@box\vbox\bgroup \@parboxrestore
    \parshape 1 \displayindent \displaywidth\relax
    \hsize=\columnwidth \noindent\ignorespaces
}{%
  \par\egroup
  \global\setbox\GRP@box\vbox{%
    \hbox{\copy\EQ@box\vtop{\unvbox\EQ@box}}%
    \unvbox\GRP@box
  }%
  \global\GRP@queue\@xp{\the\GRP@queue
%    \@elt{\gdef\EQ@trial{\let\eq@isIntertext\@True}}%
     \@elt{\let\eq@isIntertext\@True}%
  }%
}
%    \end{macrocode}
% Allow \cn{intertext} as a short form of the \env{dsuspend}
% environment; it's more convenient to write, but it doesn't support
% embedded verbatim because it reads the material as a macro argument.
% To support simultaneous use of \pkg{amsmath} and
% \pkg{breqn}, the user command \cs{intertext} is left alone
% until we enter a \pkg{breqn} environment.
%    \begin{macrocode}
\newcommand\breqn@intertext[1]{\dsuspend#1\enddsuspend}
%    \end{macrocode}
%
%
% \begin{macro}{\*}
% \begin{macro}{\discretionarytimes}
% Discretionary times sign.    Standard \latex  definition
% serves only for inline math.    Should the thin space be
% included?    Not sure.
%    \begin{macrocode}
\renewcommand{\*}{%
  \if@display
%    \end{macrocode}
% Since \cs{eq@binoffset} is mu-glue, we can't use it directly
% with \cs{kern} but have to measure it separately in a box.
%    \begin{macrocode}
    \setbox\z@\hbox{\mathsurround\z@$\mkern\eq@binoffset$}%
    \discretionary{}{%
      \kern\the\wd\z@ \textchar\discretionarytimes
    }{}%
    \thinspace
  \else
    \discretionary{\thinspace\textchar\discretionarytimes}{}{}%
  \fi
}
%    \end{macrocode}
% This is only the symbol; it can be changed to some other symbol if
% desired.
%    \begin{macrocode}
\newcommand{\discretionarytimes}{\times}
%    \end{macrocode}
% \end{macro}
% \end{macro}
%
%
% \begin{macro}{\nref}
%
% This is like \cs{ref} but doesn't apply font changes or other
% guff if the reference is undefined.
% And it is fully expandable for use as a label value.
% \begin{aside}
%
% Can break with Babel if author uses active characters in label key;
% need to address that \begin{dn}
% mjd,1999/01/21
% \end{dn}
% .
% \end{aside}
%
%    \begin{macrocode}
\def\nref#1{\@xp\@nref\csname r@#1\endcsname}
\def\@nref#1#2{\ifx\relax#1??\else \@xp\@firstoftwo#1\fi}
%%%%%%%%%%%%%%%%%%%%%%%%%%%%%%%%%%%%%%%%%%%%%%%%%%%%%%%%%%%%%%%%%%%%%%
%    \end{macrocode}
% \end{macro}
%
%
% \section{Compatibility}
%
% \paragraph{lineno} (or at the very least, allow documents to compile!)
%    \begin{macrocode}
\AtBeginDocument{%
  \@ifpackageloaded{lineno}{%
    \g@addto@macro\@dmath@start@hook{\nolinenumbers}%
    \g@addto@macro\@dgroup@start@hook{\nolinenumbers}%
  }{}%
}
%    \end{macrocode}
%
%
%
% \section{Wrap-up}
% The usual endinput.
%    \begin{macrocode}
%</package>
%    \end{macrocode}
%
%
%
%
% \section{To do}
% \begin{enumerate}
% \item Alignment for equation groups.
%
%
% \item
% Use dpc's code for package options in keyval form.
%
% \item
% Encapsulate \dquoted{break math} into a subroutine taking suitable
% arguments.
%
% \item
% Need a density check for layout S when linewidth is very small.
%
% \item
% Make \verb":=" trigger a warning about using \cs{coloneq}
% instead.
%
% \item Ill-centered multiline equation (three-line case) in
% test008.
%
% \item Attaching a single group number.
%
%
% \item
% Make sure to dump out box registers after done using them.
%
% \item Do the implementation for \cs{eq@resume@parshape}.
%
%
% \item Check on stackrel and buildrel and relbar and ???.
%
%
% \item Test math symbols at the beginning of array cells.
%
% \item Test \dbslash  cmd in and out of delims.
%
% \item Framing the equation body: the parshape and number placement
% need adjusting when a frame is present.
%
%
% \item Cascading line widths in list env.
%
%
% \item Noalign option for dmath = multline arrangement?
%
%
% \item Nocompact option, suggested 1998/05/19 by Andrew
% Swann.
%
%
% \item \cs{delbreak} cmd to add discretionary space at a break
% within delimiters.
%
%
% \item Reduce above/below skip when the number is shifted.
%
%
% \item Need a \cs{middelim} command for marking a delimiter symbol
% as nondirectional if it has an innate directionality \verb"()[]" \etc .
%
%
% \item
% \cs{xrightarrow} from amsmath won't participate in line
% breaking unless something extra is done.
% Make \cs{BreakingRel} and \cs{BreakingBin} functions?
%
% \item Placement of number in an indented quotation or
% abstract.
%
% \item If $LHSwd > 2em$, it might be a good idea to try with
% eq@indentstep = 2em before shifting the number.    Currently this
% doesn't happen if the first trial pass (without the number)
% succeeds with $indentstep = LHSwd > 2em$.
%
%
% \item Read past \verb"\end{enumerate}" when checking
% for \verb"\end{proof}"?
%
% \item
% Look into using a \dquoted{qed-list} of environment names instead of
% checking the existence of \cs{proofqed}.
%
% \item Pick up the vadjust\slash footnote\slash mark handling.
%
%
% \item Forcing\slash prohibiting page breaks after\slash before
% an equation.
%
%
% \item Adding a spanner brace on the left and individual numbers on
% the right (indy-numbered cases).
%
%
% \item Provide \cs{shiftnumber}, \cs{holdnumber} to
% override the decision.
%
% \item  Provide a mechanism for adjusting the vertical position of
% the number.    Here a version-specific selection macro would be
% useful.
% \begin{literalcode}
% \begin{dmath}[
%   style={\foredition{1}{\raisenumber{13pt}}}
% ]
% \end{literalcode}
%
%
% \item
% Add an alignleft option for an equation group to mean, break and
% align to a ladder layout as usual within the equations, but for the
% group alignment used the leftmost point (for equations that don't
% have an LHS, this makes no difference).
%
% \item
% Test with Arseneau's wrapfig for parshape\slash everypar
% interaction.
%
%
% \item Fix up the macro/def elements.
%
% \item Convert the literal examples in section \quoted{Equation types and
% forms} to typeset form.
%
%
% \item Compile comparison-examples: \eg , a standard equation
% env with big left-right objects that don't shrink, versus how shrinking
% can allow it to fit.
%
%
% \item Frame the \dquoted{figures} since they are mostly
% text.
%
% \end{enumerate}
%
%
%
% Possible enhancements:
% \begin{enumerate}
% \item Provide a \opt{pull} option meaning to pull the first
% and last lines out to the margin, like the \env{multline}
% environment of the \pkg{amsmath} package.    Maybe this should
% get an optional argument, actually, to specify the amount of space left
% at the margin.
%
% \item With the draft option, one would like to see the equation
% labels in the left margin.    Need to check with the
% \pkg{showkeys} package.
%
%
% \item Options for break preferences: if there's not enough room, do
% we first shift the number, or first try to break up the equation
% body?.    In an aligned group, does sticking to the group alignment
% take precedence over minimizing the number of line breaks needed for
% individual equations?.    And the general preferences probably need
% to be overridable for individual instances.
%
% \item Extend suppress-breaks-inside-delimiters support to inline
% math (suggestion of Michael Doob).
%
% \item Use belowdisplayshortskip above a dsuspend fragment if the
% fragment is only one line and short enough compared to the equation line
% above it.
%
%
% \item Add \cs{eqfuzz} distinct from \cs{hfuzz}.
% Make use of it in the measuring phase.
%
%
% \item Provision for putting in a \quoted{continued} note.
%
% \item Conserve box mem: modify frac, sub, sup, overline, underline,
% sqrt, to turn off \cs{bin@break} and (less urgently)
% \cs{rel@break}.
%
%
% \item More explicit support for Russian typesetting conventions (cf
% Grinchuk article).
%
%
% \item With package option \opt{refnumbers},
% leave unnumbered all uncited equations, even if they are not done with
% the star form (Bertolazzi's easyeqn idea).
%
% \item In an equation group, use a vertical bracket with the
% equation number to mark the lines contained in that equation.
%
%
% \item For a two-line multline thingamabob, try to
% make sure that the lines overlap in the middle by 2 em or whatever
% (settable design variable).
%
% \item Provide a separate vertical column for the principal mathrel
% symbols and center them within the column if they aren't all the same
% width.    Maybe an option for \env{dmath}: relwidth=x, so that two
% passes are not required to get the max width of all the mathrels.
% Or, no, just require it to be an halign or provide a macro to be
% applied to all the shorter rels:
% \begin{literalcode}
% lhs \widerel{19pt}{=} ...
%     \xrightarrow{foo} ...
% \end{literalcode}
%
%
% \item try to use vadjust for keepglue
%
% \end{enumerate}
%
% \PrintIndex
%
% \Finale

