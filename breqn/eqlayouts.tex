\documentclass[twocolumn]{article}
\pagestyle{myheadings}
\raggedbottom
\makeatletter
\def\maketitle{%
  \twocolumn[%
    \centering
    \noindent
      \begingroup \Large\bfseries \@title\par\endgroup
    \medskip
      \textsc{\@author}%
    \par \bigskip
  ]\thispagestyle{plain}%
  \markboth{\@title}{\@title}%
}

\def\@oddhead{\quad\hfil\textsc{\rightmark}\hfil\quad\llap{\thepage}}
\def\@evenhead{\rlap{\thepage}\quad\hfil\textsc{\leftmark}\quad\hfil}

\providecommand{\qq}[1]{\textquotedblleft#1\textquotedblright}
\providecommand{\mdash}{\textemdash}
\providecommand{\ndash}{\textendash}

\newcommand{\ititle}[1]{\textit{#1}}

\newcommand{\LR}[2][.4]{%
  \framebox[#1\displaywidth]{$\displaystyle{}#2$}%
}

\newcommand{\LHS}[1]{\LR[\relifactor]{#1}}

\newdimen\relindent \newdimen\rhswd

\newcommand{\dwline}{%
  \hbox to\curdw{\vrule height1ex
    \leaders\hrule height.55ex depth-.45ex\hfil
    \tiny \space display width
    \leaders\hrule height.55ex depth-.45ex\hfil
    \vrule height1ex}%
}

\newenvironment{layout}[1][.15]{%
  \noindent
  $$\edef\curdw{\the\displaywidth}%
    \def\relifactor{#1}%
    \gdef\layoutcr{\cr}\def\\{\layoutcr}%
    \binoppenalty 10000 \relpenalty 10000
  \setbox8\vbox\bgroup
    \advance\baselineskip .35\baselineskip
    \advance\lineskip .35\baselineskip \lineskiplimit\lineskip
    \relindent=#1\displaywidth
    \rhswd=\displaywidth \advance\rhswd-\relindent
    \global\row 0 \gdef\rhsskew{}%
    \halign\bgroup \global\advance\row 1 $\hfil\displaystyle{}##$&%
      \ifnum\row>1 \rhsskew \fi $\displaystyle{}##\hfil$\cr
}{%
  \crcr\egroup\egroup
  \vcenter{\halign{\hfil##\hfil\cr
    \hbox{\hss\dwline\hss}\cr\noalign{\vskip.6\baselineskip}\box8 \cr}}%
  $$\relax
  \ignorespacesafterend
}

\newcommand{\stagger}{\omit\span\gdef\layoutcr{\cr\omit\span}}
  
\newcount\row

\newcommand{\rhsskew}{}
\newcommand{\skewleft}[1]{\gdef\rhsskew{\kern-#1\relax}}

\parskip\baselineskip

% Tone down the usual large font sizes of section heads.
\let\Huge\large \let\huge\large \let\LARGE\large \let\Large\large
\let\large\normalsize

\begin{document}
\title{Equation Layouts}
\author{MJD [1998/12/28]}
\date{}
\maketitle

\section{Misc examples}

Let us consider which of these have 50\% or more of wasted whitespace
\emph{within the bounding box of the visible material}.
\begin{layout}[.4]
\LHS{L}&=\LR[.35]{R_{1}}\\
&=\LR[.25]{R_{1}}
\end{layout}

\section{Ladder and step layouts}

\subsection{Straight ladder layout}
This is distinguished by a relatively short LHS and one or more RHS's of
any length.
\begin{layout}
\LHS{L} &= \LR[.5]{R_{1}}\\
&=\LR[.3]{R_{2}}\\
&=\LR[.25]{R_{3}}\\
&\qquad\ldots
\end{layout}
The simplest kind of equation that fits on one line and has only one RHS
may be viewed as a trivial subcase of the straight ladder layout:
\begin{layout}
\LHS{L} &= \LR[.5]{R}
\end{layout}
If some of the RHS's are too wide to fit on a single line they may be
broken at binary operator symbols such as plus or minus. This is still
classified as a straight ladder layout if none of the fragments intrude
into the LHS column, because the underlying parshape is the same.
\begin{layout}
\LHS{L} &= \LR[.4]{R_{1a}}\\
&\quad +\LR[.6]{R_{1b}}\\
&=\LR[.3]{R_{2}}\\
&=\LR[.25]{R_{3a}}\\
&\quad +\LR[.45]{R_{3b}}\\
&\quad +\LR[.54]{R_{3c}}\\
&\qquad\ldots
\end{layout}

\subsection{Skew ladder layout}
\begin{layout}[.5]
\skewleft{.35\displaywidth}
\LHS{L}&= \LR[.3]{R_{1}}\\
&=\LR[.6]{R_{2}}\\
&=\LR[.25]{R_{3}}\\
&\qquad\ldots
\end{layout}
In a skew ladder layout, the combined LHS width plus width of $R_{1}$
does not exceed the available width, but one of the other RHS's is so
wide that aligning its relation symbol with the others cannot be done
without making it run over the right margin: $\mbox{width}(L) +
\mbox{width}_{\mathrm{max}}(R_{i})>\mbox{width}_{\mathrm{avail}}$. In
that case we next try aligning all but the first relation symbol,
allowing all the $R_{i}$ after $R_1$ to shift leftward.

\subsection{Drop ladder layout}
\begin{layout}[.6]
\makebox[.15\displaywidth][l]{\LHS{L}}\\
&= \LR[.6]{R_{1}}\\
&=\LR[.3]{R_{2}}\\
&=\LR[.25]{R_{3}}\\
&\qquad\ldots
\end{layout}
The drop ladder layout is similar to the skew ladder layout but with the
width of $R_1$ too large for it to fit on the same line as the LHS. Then
we move $R_1$ down to a separate line and try again to align all the
relation symbols. Note that this layout consumes more vertical space
than the skew ladder layout.

\subsection{Step layout}
\begin{layout}[.6]
\stagger
\LHS{R_{a}}\\
\qquad + \LR[.7]{R_{b}}\\
\qquad\qquad + \LR[.6]{R_{c}}\\
\qquad\qquad\qquad + \LR[.45]{R_{d}}\\
\qquad\qquad\qquad\qquad\ldots
\end{layout}
The chief characteristic of the step layout is that there is no relation
symbol, so that the available line breaks are (usually) all at binary
operator symbols. Let $w_1$ and $w_l$ be the widths of the first and
last fragments. We postulate that the ideal presentation is as follows:
Choose a small stairstep indent $I$ (let's say 1 or 2 em). We want the
last fragment to be offset at least $I$ from the start of the first
fragment, and to end at least $I$ past the end of the first fragment. If
there are only two lines these requirements determine a target width
$w_T=\max(w_1+I,w_l+I)$. If there are more than two lines ($l>2$) then
use $w_T = \max(w_1 + (l-1)I, w_l+I, w_{\mathrm{avail}}$ and reset $I$
to $w_T/(l-1)$ if $w_T = w_{\mathrm{avail}}$.

Furthermore, we would like the material to be distributed as evenly as
possible over all the lines rather than leave the last line exceedingly
short. If the total width is $1.1(\mbox{width}_{\mathrm{avail}})$, we
don't want to have .9 of that on line 1 and .2 of it on line 2:
\begin{layout}[.9]
\stagger
\LHS{R_{a}\hfil+\hfil R_{b}\hfil+\hfil R_{c}}\\
\qquad + \LR[.1]{R_{d}}
\end{layout}
Better to split it as evenly as possible, if the available breakpoints
permit.
\begin{layout}[.5]
\stagger
\LHS{R_{a}\hfil+\hfil R_{b}}\\
\qquad + \LR[.5]{R_{c}\hfil+\hfil R_d}
\end{layout}
A degenerate step layout may arise if an unbreakable fragment of
the equation is so wide that indenting it to its appointed starting
point would cause it to run over the right margin. In that case, we want
to shift the fragment leftward just enough to bring it within the right
margin:
\begin{layout}[.6]
\stagger
\LHS{L_{a}}\\
\quad + \LR[.8]{L_{b}}\\
\qquad + \LR[.7]{L_{c}}\\
\; + \LR[.87]{L_{d}}\\
\qquad\ldots
\end{layout}
And then we may want to regularize the indents as in the drop ladder
layout. Let's call this a dropped step layout:
\begin{layout}[.6]
\stagger
\LHS{L_{a}}\\
\quad + \LR[.8]{L_{b}}\\
\quad + \LR[.7]{L_{c}}\\
\quad + \LR[.87]{L_{d}}\\
\qquad\ldots
\end{layout}

\section{Strategy}

Here is the basic procedure for deciding which equation layout to use,
before complications like equation numbers and delimiter clearance come
into the picture. Let $A$ be the available width, $w_{\mathrm{total}}$
the total width of the equation contents, $w(L)$ the width of the
left-hand side, $w_{\max}(R)$ the max width of the right-hand sides, $I$
the standard indent for step layout, and $O$ the standard offset for
binary operators if a break occurs in the middle of an RHS. Also let
$t_L$ and $t_R$ represent certain thresholds for the width of the LHS or
the RHS at which a layout decision may change, as explained below.

\newpage
\begin{small}
\begin{enumerate}
\renewcommand{\labelenumi}{(\theenumi)}
\item \ititle{Does everything fit on one line?}\label{i:LR}
  $w_{\mathrm{total}}\leq A$?

Yes: print the equation on a single line (done).

No: Check whether the equation has both LHS and RHS (\ref{i:lhs-check}).

\item \ititle{Is there a left-hand side?}\label{i:lhs-check}
Are there any relation symbols in the equation?

Yes: Try a ladder layout (\ref{i:ladder}).

No: Try a step layout (\ref{i:step}).

\item\ititle{Does the LHS leave room to fit the widest RHS?}\label{i:ladder}
  $w(L)+w_{\max}(R)<A$?

Yes: Use a straight ladder layout (\ref{i:straight-ladder}).

No: Check the width of the LHS (\ref{i:check-lhs}).

\item\ititle{Is the LHS relatively short?}\label{i:check-lhs}
$w(L)\leq t_L$? (where $t_L$ is typically $0.4A$).

Yes: Subdividing one or more of the RHS's may permit us to use a
straight ladder layout (\ref{i:straight-ladder}).

No: The straight ladder layout is unlikely to work.
Try a skew or drop ladder layout (\ref{i:skew-drop}).

\item\ititle{Straight ladder layout}\label{i:straight-ladder}
Set up a straight ladder parshape [0pt $A$ $w(L)$ $A-w(L)$] and run
a trial break. If the combined width of the LHS plus the longest RHS is
no greater than $A$ then we should get a satisfactory layout with all
line breaks occurring at major division points (relation symbols).
Otherwise, we hope, some additional line breaks at minor division points
will allow everything to fit within the text column.

\ititle{Line breaks OK?}

\begingroup \hbadness=9999
Yes: The straight ladder layout succeeded
  (done).\par\endgroup

No: Try a skew or drop ladder layout (\ref{i:skew-drop}).

\item\ititle{Do the LHS and the first RHS fit on one
    line?}\label{i:skew-drop} $w(L)+w(R_1) \leq A$?

Yes: Try a skew ladder layout (\ref{i:skew}).

No: Try a drop ladder layout (\ref{i:drop}).

\item\ititle{Skew ladder layout}\label{i:skew}
Set up a parshape [0pt $A$ $I$ $A-I$] and run a trial break.

\ititle{Line breaks OK?}

Yes: Skew ladder layout succeeded (done).

No: One of the unbreakable fragments of the $R_i$ ($i>1$) is wider than
$A-I$; try an almost-columnar layout (\ref{i:almost-columnar}).

\item\ititle{Drop ladder layout}\label{i:drop}
Set up a parshape [0pt $w(L)$ $I$ $A-I$] and run a trial break.
This is the same parshape as for a skew ladder layout except that the
width of the first line is limited to the LHS width, so that the RHS is
forced to drop down to the next line.

\ititle{Line breaks OK?}

Yes: Drop ladder layout succeeded (done).

No: One of the unbreakable fragments of the $R_i$ ($i>1$) is wider than
$A-I$; try an almost-columnar layout (\ref{i:almost-columnar}).

\item\ititle{Almost-columnar layout}\label{i:almost-columnar}
This presupposes a trial break that yielded a series of expressions or
fragments, one per line. Let $w(F)$ denote the width of the first
fragment and $w(R_i)$ the widths of the remaining fragments.
Set up a parshape [0pt $w(F)$ $A-w_{\max}(R_i)$ $w_{\max}(R_i)$]: in other
words, set the first line flush left and the longest line flush right
and all other lines indented to the same position as the longest line.
But as a matter of fact there is one other refinement for extreme cases:
if $w_{\max}(R_i)>A$ then the parshape can be simplified without loss to
[0pt $w(F)$ 0pt $A$]\mdash for that is the net effect of substituting
$\min(A,w_{\max})$ in stead of $w_{\max}$.
(Done.)

\item\ititle{Step layout}\label{i:step} Set target width $w_T$ to $A -
  2I$.  Set parshape to [0pt $w_T$ $I$ $w_T -I$ $2I$ $w_T -2I$ \ldots\ 
  $(l-1)I$ $w_T - (l-1)I$], where $l=\lceil w_{\mathrm{total}}/A\rceil$
  is the expected number of lines that will be required.  Trial break
  with that parshape in order to find out the width of the last line.

\ititle{Indents OK?}

Yes: Step layout succeeded (done).

No: One of the fragments is too wide to fit in
the allotted line width, after subtracting the indent specified by the
parshape. Try a dropped step layout (\ref{i:drop-step})

\item\ititle{Dropped step layout}\label{i:drop-step} Set up a parshape
  [0pt $A$ $I$ $A-I$] and run a trial break.  Note that this is actually
  the same parshape as for a skew ladder layout.

\ititle{Line breaks OK?}

Yes: Dropped step layout succeeded (done).

No: One of the unbreakable fragments of the $R_i$ ($i>1$) is wider than
$A-I$; as a last resort try an almost-columnar layout (\ref{i:almost-columnar}).

\end{enumerate}
\par\end{small}

\end{document}
